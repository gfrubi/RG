\chapter{Motivaciones para RG: Relatividad Especial y Gravitación newtoniana}
\section{Distribución de energía y momentum de la materia: tensor de energía-momentum}

Para un resumen de las definiciones básicas, la notación y convenciones usadas en la teoría de la mecánica y electrodinámica relativista, vea los apuntes de Electrodinámica (II) \cite{7}.

En la teoría newtoniana de la gravitación la \textit{masa} de los cuerpos es la fuente del campo gravitacional. Cuando se describe un sistema como un \textit{continuo}, la distribución espacial y el movimiento de su masa son caracterizados por 4 cantidades: la densidad de masa $\rho$ y (las tres componentes de) la densidad de corriente de masa\footnote{Si el sistema está constituido de una distribución de densidad $\rho$ que se mueve con velocidad $\vec{v}$, entonces $\vec{J}=\rho\vec{v}$.} $\vec{J}$. En la teoría de RE, por otro lado, la masa de los cuerpos no es más que una componente de la energía de 'estos (su energía en reposo): el concepto de masa es reemplazado por el de energía (no existe ley de conservación de la masa de un sistema, sino sólo de la energía total, etc.). Más a'un, la energía es ``sólo'' una componente del 4-momentum de un cuerpo. Los valores de las componentes del 4-momentum, y por lo tanto de la energía y del momentum lineal, se ``mezclan'' al describir un sistema desde distintos SRI's (el 4-momentum es un 4-vector bajo TL's). Por todo esto, es natural suponer que en una teoría relativista de la gravitación las fuentes del campo gravitacional est'en descritas por la distribución de energía y momentum en el espacio(tiempo) y sus flujos\footnote{Es decir, densidad de energía, densidad de flujo de energía, densidad de momentum y densidad de flujo de momentum.}. Todas estas cantidades quedan condensadas en el \textit{tensor energía-momentum de un sistema}.

Las componentes del tensor de energía-momentum\footnote{De acuerdo a nuestras convenciones este tensor tiene unidades de \textit{densidad de energía}, es decir, energía por unidad de volumen o, equivalentemente, de \textit{presión}.} $T^{\mu\nu}$ están relacionadas con las densidades y flujos de energía y momentum, de acuerdo a
\begin{equation}
 T^{00}(x)=u(x), \qquad T^{0i}(x)=\frac{1}{c}S^i(x), \qquad T^{i0}(x)=c\,\pi^i(x), \qquad T^{ij}(x)=p^{ij}(x),
\end{equation}
donde $u$ es la \textit{densidad de energía} (energía por unidad de volumen), $S^i$ la \textit{densidad de flujo de energía} (energía por unidad de tiempo y superficie), $\pi^i$ la \textit{densidad de momentum} (momentum por unidad de volumen) y $p^{ij}$ el \textit{tensor de tensiones} (momentum por unidad de tiempo y superficie).

El \textit{4-momentum total del sistema}, contenido en un volumen $V$ en un instante dado, es entonces la integral
\begin{equation}
p^\mu=\frac{1}{c}\int_V T^{\mu 0}\,dV \label{pmTmn}
\end{equation}
o, equivalentemente, la energía y el momentum adoptan la forma
\begin{equation}
E=\int_V u\,dV, \qquad P^i=\int_V \pi^i\,dV.
\end{equation}
Si el sistema está \textit{aislado}, de modo que su energía y momentum se conserven, se satisface 
\begin{equation}
 \partial_\nu T^{\mu \nu}=0
\end{equation}
que, en virtud de las identificaciones anteriores, es equivalente a las usuales ``ecuaciones de continuidad'' para la energía y el momentum:
\begin{equation}
 \partial_tu+\partial_iS^i=0, \qquad \partial_t \pi^i+\partial_jp^{ij}=0.
\end{equation}


\subsection{Campo electromagn'etico}
El campo electromagn'etico posee y transporta energía y momentum. El respectivo tensor de energía-momentum del campo electromagn'etico en el vacío es dado por
\begin{equation}
\boxed{T_\text{em}^{\mu\nu}:=\frac{1}{\mu_0}\left(  F^{\mu\lambda}F_\lambda^{\ \nu}+\frac{1}{4}F_{\rho\sigma}F^{\rho\sigma}\eta^{\mu\nu} \right).}  \label{temsimSI}
\end{equation}
Puede verificarse que las densidades de energía, momentum y sus flujos están dados por las cantidades apropiadas definidas en electrodinámica:
\begin{align}
u &:= \frac{1}{2}\left(\varepsilon_0\vec{E}^2+\frac{1}{\mu_0}\vec{B}^2\right),\label{uTheta}\\
S^i &:= \frac{1}{\mu_0}\left(\vec{E}\times\vec{B}\right)^i=c^2\pi^i, \label{STheta}\\
T^{ij} &:= \frac{1}{2}\left(\varepsilon_0\vec{E}^2+\frac{1}{\mu_0}\vec{B}^2\right)\delta_j^i-\varepsilon_0 E^i E^j -\frac{1}{\mu_0}B^i B^j , \label{TTheta}
\end{align}
Para mayores detalles, ver \cite{7}.


\subsection{Fluido perfecto}
Consideramos ahora un fluido con estructura interna descrita por una \textit{presión
isótropa} $p$. 'Esta está definida respecto al SRI comóvil $K'$ con un elemento de fluido dado, ubicado en el evento $x$, de modo que
\begin{eqnarray}
T'^{00}(x)&=&\epsilon(x) , \label{t00c}\\
T'^{0i}(x) &=&0, \label{t0ic}\\
T'^{i0}(x) &=&0, \label{ti0c}\\
T'^{ij}(x) &=&p(x)\,\delta^{ij} .\label{tijc}
\end{eqnarray}

La componente en (\ref{t00c}) es la \textit{densidad propia de energía} (es decir, $\epsilon(x)\,dV'$ es la energía del sistema contenida en el elemento de volumen $dV'$, en el SRI instantáneamente comóvil con el fluido en el evento $x$), que a menudo se expresa en t'erminos de la \textit{densidad propia de masa} $\rho(x)$, definida por $\rho(x):=\epsilon(x)/c^2$. Las componentes
mixtas (\ref{t0ic}) y (\ref{ti0c}) son nulas puesto que suponemos que en el sistema comóvil \textit{la distribución microscópica es isótropa}, por lo que no existe una dirección preferente para el flujo de energía ni para la densidad de momentum. Finalmente las componentes espaciales
en (\ref{tijc}) son diagonales puesto que \textit{suponemos que en un fluido
perfecto la fuerza sobre cada elemento de superficie es normal a esta
superficie}, de modo que $T^{ij}dS^j=T^{ij}\,\hat{n}^j\,dS=p\,\hat{n}^i dS$ y que $p>0$ corresponde a un fluido que tiende a expandirse.

A partir del tensor de energía-momentum para un fluido perfecto en su sistema local comóvil, encontraremos la expresión general respecto a cualquier
otro SRI por medio del boost de Lorentz apropiado. Para esto usaremos los resultados y convenciones contenidos en \cite{7}. Si la TL es denotada como $x'^\mu=\Lambda^\mu_{\ \nu}x^\nu$ y $v^i=c\beta^i$ es la velocidad de $K'$ respecto a $K$ entonces
\begin{equation}
 \Lambda^0_{\ 0}=\gamma, \qquad \Lambda^i_{\ 0}=\Lambda^0_{\ i}=-\gamma\beta^i, \qquad \Lambda^i_{\ j}=\delta^i_j
+\frac{(\gamma-1) }{\beta^2}\beta^i\beta^j .
\end{equation}
Por otro lado el tensor buscado $T^{\mu\nu}$ está relacionado con $T'^{\mu\nu}$ por medio de
\begin{eqnarray}
T^{\mu\nu}&=&(\Lambda^{-1})^\mu_{\ \lambda}(\Lambda^{-1})^\nu_{\ \rho}T'^{\lambda\rho}
\\
&=& (\Lambda^{-1})^\mu_{\ 0}(\Lambda^{-1})^\nu_{\ 0}T'^{00}+(\Lambda^{-1})^\mu_{\
i}(\Lambda^{-1})^\nu_{\ j}T'^{ij}\\
&=& (\Lambda^{-1})^\mu_{\ 0}(\Lambda^{-1})^\nu_{\ 0}\rho c^2+p(\Lambda^{-1})^\mu_{\
i}(\Lambda^{-1})^\nu_{\ j}\delta^{ij}.
\end{eqnarray}
Las componentes de la matriz $(\Lambda^{-1})$ están dadas por
\begin{equation}
(\Lambda^{-1})^0_{\ 0}=\gamma, \qquad (\Lambda^{-1})^i_{\ 0}=(\Lambda^{-1})^0_{\ i}=+\gamma\beta^i, \qquad (\Lambda^{-1})^i_{\ j}=\delta^i_j
+\frac{(\gamma-1) }{\beta^2}\beta^i\beta^j .
\end{equation}
Por lo tanto, obtenemos que
\begin{eqnarray}
T^{00}&=& (\Lambda^{-1})^0_{\ 0}(\Lambda^{-1})^0_{\ 0}\rho c^2+p(\Lambda^{-1})^0_{\ i}(\Lambda^{-1})^0_{\ j}\delta^{ij} \\
&=& \gamma^2\rho c^2+p\gamma^2\beta^i\beta^j\delta^{ij} \\
&=& \gamma^2\rho c^2+p\gamma^2\beta^2 .
\end{eqnarray}

Vemos que en el contexto de la teoría Especial de la Relatividad \textit{la presión de un fluido en movimiento aporta a su densidad de energía}\footnote{En su generalización a la teoría General de la Relatividad, este hecho tiene importantes consecuencias, por ejemplo, para el análisis de estabilidad (y colapso) estelar.}. Además,
\begin{eqnarray}
T^{0i}&=& (\Lambda^{-1})^0_{\ 0}(\Lambda^{-1})^i_{\ 0}\rho c^2+p\,(\Lambda^{-1})^0_{\ j}(\Lambda^{-1})^i_{\ k} \delta^{jk} \\
&=& \gamma^2\beta^i\rho c^2 +p\gamma\beta^j\left(\delta^i_k
+\frac{1}{\beta^2}\beta^i\beta^k(\gamma-1) \right) \delta^{jk} \\
&=& \gamma^2\beta^i\rho c^2 +p\gamma\left(\beta^i +\beta^i(\gamma-1) \right)
\\
&=& \gamma^2\beta^i\rho c^2 +p\gamma^2\beta^i,
\end{eqnarray}
y $T^{i0}=T^{0i}$. Note además que $T^{0i}\neq T^{00}\beta^i$ cuando $p\neq 0$. Finalmente,
\begin{eqnarray}
T^{ij}&=& (\Lambda^{-1})^i_{\ 0}(\Lambda^{-1})^j_{\ 0}\rho c^2+p\,(\Lambda^{-1})^i_{\ k}(\Lambda^{-1})^j_{\ l}\delta^{kl} \\
&=&\gamma^2\beta^i\beta^j\rho c^2+ p\left(\delta^i_k+
\frac{1}{\beta^2}\beta^i\beta^k(\gamma-1) \right)\left(\delta^j_l
+\frac{1}{\beta^2}\beta^j\beta^l(\gamma-1) \right)\delta^{kl} \\
&=&\gamma^2\beta^i\beta^j\rho c^2+p\left(\delta^{ij}
+\frac2{\beta^2}\beta^i\beta^j(\gamma-1)
+\frac{1}{\beta^2}\beta^i\beta^j(\gamma-1)^2\right) \\
&=&\gamma^2\beta^i\beta^j\rho c^2+p\left(\delta^{ij}
+\frac{1}{\beta^2}\beta^i\beta^j(\gamma^2-1)\right) \\
&=&\gamma^2\beta^i\beta^j\rho c^2+p\left(\delta^{ij}
+\gamma^2\beta^i\beta^j\right).
\end{eqnarray}
Puede verificarse fácilmente que 'estas son precisamente las componentes de
\begin{equation}\label{temfp}
\boxed{T_{\rm fp}^{\mu\nu}=\left( \rho +\frac{p}{c^2}\right) u^\mu
u^\nu-p\,\eta^{\mu\nu}.}\marginnote{energía-momentum fluido perfecto}
\end{equation}

La expresión (\ref{temfp}) constituye por tanto la expresión covariante del tensor de energía-momentum de un fluido perfecto, caracterizado por su densidad propia de masa $\rho$, y su presión isótropa $p$, movi'endose con 4-velocidad $u^\mu$. Estas tres cantidades son, en general, dependientes de la posición y del tiempo. En t'erminos de la densidad propia de energía, $\epsilon$, tenemos:
\begin{equation}\label{temfpu}
\boxed{T_{\rm fp}^{\mu\nu}=\left(\epsilon +p\right)\frac{u^\mu}{c}
\frac{u^\nu}{c}-p\,\eta^{\mu\nu}.}\marginnote{energía-momentum fluido perfecto}
\end{equation}
\subsection{Fluido simple (polvo)}

El caso particular de un fluido simple, es decir, sin presión, describe la situación en que el sistema está constituido por  un conjunto de partículas \textit{no interactuantes movi'endose todas con la misma velocidad} en un elemento de volumen dado (``polvo''). El tensor energía-momentum se reduce entonces a:
\begin{equation}
\boxed{T_{\rm polvo}^{\mu\nu}(x)=\rho\, u^\mu u^\nu.}
\end{equation}

Las respectivas densidades de energía, de flujo de energía, de momentum y el
tensor de tensiones están dadas en este caso por:
\begin{eqnarray}
u&=&\rho \gamma^2 c^2, \\ 
S^i&=&\rho \gamma^2c^2v^i=u\,v^i, \\ 
{\pi}^i&=&\rho \gamma^2v^i =\frac{u}{c^2}v^i, \\ 
p^{ij} &=&\rho \gamma^2 v^i v^j=\pi^iv^j.
\end{eqnarray}
Note que para este tipo de sistema se verifica que los respectivos flujos $S^i$ y $p^{ij}$ tienen la forma esperada de producto de la respectiva densidad (``transportada'') y la velocidad de cada elemento de fluido (``a la que es transportada''), tal como ocurre por ejemplo para una distrubución de carga de densidad $\rho_{\rm c}$ movi'endose con velocidad $\vec{v}_{\rm c}$, para la cual la densidad de corriente es dada por $\vec{J}_{\rm c}=\rho_{\rm c}\vec{v}_{\rm c}$.

\section{Ley de Gravitación universal}
En la teoría newtoniana la fuerza gravitacional de una masa muy peque\~na (``puntual'') $m_1$ sobre otra masa (tambi'en muy peque\~na) $m_2$
está dada por la ley de gravitación de Newton ($1687$),
\begin{equation}
\vec{F}_{1\rightarrow 2}=-G\frac{m_1m_2}{|\vec{r}|^2}\frac{\vec{r}}{|\vec{r}|},
\end{equation}
donde $G$ es la constante de gravitación (de Cavendish, medida por primera vez en 1897). Su valor, de acuerdo a las mediciones actuales, es\footnote{Ver por ejemplo, \url{http://www.codata.org}.}
\begin{equation}
G\stackrel{\rm SI}{=}(6,674 28\pm 0.000 67)\times 10^{-11}\, m^3kg^{-1}s^{-2}.
\end{equation}
El vector $\vec{r}:=\vec{x}_2-\vec{x}_1$ apunta desde $m_1$ hasta $m_2$, ver figura \ref{lguN}. De acuerdo al principio de acción y reacción (tercera ley de Newton),
tenemos que $\vec{F}_{1\rightarrow 2}=-\vec{F}_{2\rightarrow 1}$.
\begin{center}
\begin{figure}[H]
\centerline{\includegraphics[height=4cm]{fig/fig-ley-gravitacion-Newton.pdf}}
\caption{Ley de gravitación universal de Newton.}
\label{lguN}
\end{figure}
\end{center}
El \textit{campo gravitacional} $\vec{g}$ es definido como la fuerza por unidad de masa
que experimenta una masa de prueba. Por tanto, el campo gravitacional en la posición $\vec{x}_2$ generado por una masa (muy peque\~na) $m_1$ es
\begin{equation}
\vec{g}_1(x_2):=\frac{\vec{F}_{1\to 2}}{m_2}=-G\frac{m_1}{|\vec{r}|^2}\frac{\vec{r}}{|\vec{r}|}.
\end{equation}
En el caso anterior, es posible distinguir entre la masa $m_1$ que
\textit{genera} el campo gravitacional, que llamamos \textit{masa gravitacional
activa}, y la masa $m_2$ de la partícula que sobre la cual act'ua la fuerza, que
llamaremos \textit{masa gravitacional pasiva}.

Por el \textit{principio} de superposición, el campo gravitacional generado por una distribución continua de masa, caracterizada por su densidad de masa $\rho(\vec{x})$ será de la forma
\begin{equation}
\vec{g}(\vec{x})=-G\int_V\frac{\rho(\vec{x}')(\vec{x}-\vec{x}')}{|\vec{x}-\vec{x}'|^3}dV'.
\end{equation}
Como consecuencia, el campo gravitacional es siempre irrotacional,
\begin{equation}
\vec{\nabla}\times\vec{g}=\vec{0},
\end{equation}
y puede derivarse a partir de un \textit{potencial gravitacional} $\phi$, de modo
que
\begin{equation}
\boxed{\vec{g}(x)=-\vec{\nabla}\phi(x),}
\end{equation}
donde
\begin{equation}
\phi(\vec{x}):=-G\int_V\frac{\rho(\vec{x}')}{|\vec{x}-\vec{x}'|}dV'+\text{cte}.
\end{equation}
Este potencial satisface la \textit{ecuación de Poisson},
\begin{equation}\marginnote{Ecuación de Poisson}
\boxed{\nabla^2\phi=4\pi G \rho,} \label{Poisson}
\end{equation}
o equivalentemente, el campo gravitacional satisface la ecuación (``de Gauss")
\begin{equation}
\vec\nabla\cdot\vec{g}=-4\pi G \rho, \label{gaussg}
\end{equation}
que relaciona los \textit{gradientes} del campo gravitacional con la distribución de masa, descrita por la densidad $\rho(\vec{r})$, que lo genera.

Si bien la ecuación (\ref{Poisson}) es una ecuación para el campo $\phi$, en el
contexto newtoniano el campo gravitacional no es un genuino \textit{campo
dinámico}, es decir, no contiene grados de libertad independientes, ya que
'este queda determinado (para condiciones de borde dadas) 'unicamente por la
densidad de masa $\rho$. En otras palabras, la teoría newtoniana de la
gravitación es una teoría de \textit{acción a distancia}. De acuerdo a este modelo, si se removiese la fuente del campo ($\rho\rightarrow 0$) entonces el campo gravitacional desaparecería \textit{instantáneamente} en todo punto. El mismo Newton estaba bastante preocupado por este hecho. Claramente, esta propiedad es \textit{incompatible} con los postulados básicos de la teoría de Relatividad Especial.

\section{Masa inercial y masa gravitacional}
Una característica especial de la interacción gravitacional es que, hasta donde hemos podido observar, la aceleración de un cuerpo que cae libremente en un campo gravitacional dado \textit{no depende de su masa sino sólo de su posición} en el campo gravitacional. En el contexto newtoniano esto puede describirse a trav'es de la igualdad entre la \textbf{masa inercial} y la \textbf{masa gravitacional} de \textit{todo cuerpo}.

En este contexto, la \textbf{masa inercial} de un cuerpo, $\stackrel{\rm iner}{m}$, es definida como aquella cantidad que mide la resistencia de 'este a cambiar su estado de movimiento, de acuerdo a la segunda ley de Newton:
\begin{equation}\label{2ln}
\vec{F}=\stackrel{\rm iner}{m}\frac{d^2\vec{x}}{dt^2}.
\end{equation}
Esta masa puede ser determinada por medio de experimentos ``no-gravitacionales'', es decir, que no involucran la interacción gravitacional. Por ejemplo, puede determinarse la masa inercial de un cuerpo comparando su frecuencia de oscilación $\omega$ con la frecuencia de oscilación de un cuerpo de referencia $\omega_{\rm ref}$) cuando ambos son unidos alternativamente a un mismo resorte, de modo que se produzca una oscilación horizontal. En este caso (suponiendo la ley de Hooke) se cumple que $\stackrel{\rm iner}{m}/\stackrel{\rm iner}{m}_{\rm ref}=\omega_{\rm ref}^2/\omega^2$.

Por otro lado, se define la \textbf{masa gravitacional}, $\stackrel{\rm grav}{m}$, de un cuerpo como la cantidad que mide la magnitud de la fuerza que este cuerpo experimenta al estar en una región del espacio con campo gravitacional $\vec{g}$:
\begin{equation}\label{fg}
\vec{F}_{\rm grav}=\stackrel{\rm grav}{m}\vec{g}=-\stackrel{\rm grav}{m}\vec\nabla\phi.
\end{equation}
A partir de (\ref{2ln}) y (\ref{fg}) obtenemos que la aceleración de un cuerpo debido a un campo gravitacional $\vec{g}$ es dado por
\begin{equation}\label{amimg}
\frac{d^2\vec{x}}{dt^2}=\left(\frac{\stackrel{\rm grav}{m}}{\stackrel{\rm iner}{m}}\right)\vec{g}.
\end{equation}
Esta expresión es análoga a la que determina la aceleración de un cuerpo cargado en presencia de un campo el'ectrico externo:
\begin{equation}\label{aqE}
\frac{d^2\vec{x}}{dt^2}=\left(\frac{q}{\stackrel{\rm iner}{m}}\right)\vec{E}.
\end{equation}
En este sentido, la masa gravitacional es el análogo gravitacional a la carga el'ectrica (es decir, es la ``carga gravitacional'').


\subsection{Universalidad de la interacción gravitacional, Principio de Equivalencia D'ebil}

La experiencia muestra que la interacción electrostática causa, incluso en presencia de un mismo campo el'ectrico, que cuerpos diferentes aceleren en forma diferente. Esto se describe, de acuerdo a (\ref{aqE}), diciendo que distintos cuerpos poseen distintos valores de la relación carga-masa (inercial) $q/m$. Existen cuerpos donde esta relación es positiva, negativa, o cero. En contraste, la interacción gravitacional \textit{parece} (de acuerdo a todas observaciones realizadas hasta hoy) tener la propiedad 'unica que \textit{todos} los cuerpos aceleran en la \textit{misma} dirección y con la \textit{misma} magnitud en un campo gravitacional dado. De acuerdo a (\ref{amimg}), esto requiere que el cuociente entre la masa inercial y gravitacional de todo cuerpo sea una constante universal (es decir, que tenga siempre el mismo valor, independiente del cuerpo). Por simplicidad, usualmente se eligen las unidades de $\stackrel{\rm grav}{m}$ (o, equivalentemente, de la constante gravitacional $G$) tal que esta \textit{universalidad de la interacción gravitacional} implique que
\begin{equation}
\boxed{\stackrel{\rm iner}{m}=\stackrel{\rm grav}{m}.}
\end{equation}
En este caso, (\ref{amimg}) se reduce a
\begin{equation}\label{ag}
\frac{d^2\vec{x}}{dt^2}=\vec{g}
\end{equation}
\textit{para todo cuerpo}.

Si el \textit{postulado} de universalidad de la interacción gravitacional o, en otras palabras, de la igualdad de masa inercial y gravitacional, es \textit{siempre} válido entonces la trayectoria de los cuerpos sometidos (sólo) a la acción de la gravedad es independiente del cuerpo (de su carga, composición, temperatura, color, etc.) y sólo depende del campo gravitacional en el que se encuentra (determinado por los otros cuerpos en el sistema) y de la posición y velocidad inicial de 'este. La suposición que las trayectorias de los cuerpos sometidos a la acción de la gravedad son realmente id'enticas (dadas las mismas condiciones iniciales) es usualmente llamado \textbf{Principio de Equivalencia D'ebil} (PED). Lo importante es que la validez del PED o, equivalentemente, de la universalidad de la aceleración debido a la gravedad, o de la igualdad de las masas inerciales y gravitacionales, son características distintivas de la interacción gravitacional que se han obtenido a partir de la generalización de \textit{observaciones}, y que pueden (y deben!) ser testeadas experimentalmente. Hasta ahora, toda la evidencia observacional respalda al PED, siendo verificado con precisiones que restringen las posibles \textbf{desviaciones relativas de la aceleración}, y por lo tanto la diferencia relativa del cuociente $\stackrel{\rm iner}{m}/\stackrel{\rm grav}{m}$ entre dos cuerpos (``1'' y ``2'')
\begin{equation}
\frac{\Delta a}{a}=\frac{\left(\stackrel{\rm iner}{m}/\stackrel{\rm grav}{m}\right)_2-\left(\stackrel{\rm iner}{m}/\stackrel{\rm grav}{m}\right)_1}{\left(\stackrel{\rm iner}{m}/\stackrel{\rm grav}{m}\right)_1},
\end{equation}
a ser menores que una parte en $10^{13}$. Para más detalles, ver \cite{Will06}-\cite{STEP}.
Estas observaciones abarcan desde los experimentos originales de Galileo usando p'endulos, planos inclinados, etc. hasta los más modernos y precisos experimentos con sat'elites e incluso neutrones \cite{Koester76} y electrones. El PED tambi'en ha sido testeado con átomos, usando t'ecnicas de interferometría atómica, ver \cite{Zhou15} y las referencias ahí citadas. Recientemente, se han reportado experimentos que testean el PED con átomos con distintas orientaciones de su spin, ver \cite{Duan16}.

Como veremos más adelante, la teoría general de la relatividad de Einstein \textit{supone} como ingrediente crucial para su construcción el PED\footnote{de hecho se postula una versión generalizada y a'un más demandante, el así llamado ``principio de equivalencia fuerte'' (PEF).}. Por esta razón existe continuo inter'es en poner a prueba este principio, y mejorar cada vez más la precisión con la que se ha verificado su validez.

El siguiente gráfico resume los resultados de m'ultiples experimentos modernos que determinan cotas máximas para las aceleraciones relativas de distintos cuerpos sometidos a la acción de la gravedad, junto con el a\~no en que fueron realizados los experimentos.
\begin{center}
\begin{figure}[H]
\centerline{\includegraphics[height=5cm]{fig/fig-tests-PED.pdf}}
\caption{Límites a posibles violaciones del PED. Figura adaptada a partir de la original en \cite{Turyshev08}.}
\label{fig:equiv1}
\end{figure}
\end{center}

\section{Fuerzas de marea}
%Tal como hemos visto, un indicador de la presencia de campo gravitacional no nulo es la \textit{aceleración relativa} entre distintos SRLI's. 
En el contexto newtoniano, se usa el t'ermino \textbf{fuerzas de marea} para describir el efecto de la \textit{inhomogeneidad} de un campo gravitacional, que genera diferencias en la aceleración de la materia en distintas partes de un sistema. Considere el caso de un campo gravitacional generado por una distribución compacta de masa (ver, por ejemplo, la figura \ref{fig:SRLI}). Debido a las inhomogeneidades del campo gravitacional, dos peque\~nas masas de prueba inicialmente en reposo caerán hacia el centro de fuerzas, acercándose (o, en general, cambiando la distancia entre ellas). Similarmente, una gota de agua inicialmente esf'erica tenderá a deformarse a una forma elipsoidal debido a que la fuerza gravitacional en su parte inferior (más cerca del centro de fuerzas) es mayor que en la parte superior, que está a una distancia mayor de la fuente.

Si la distancia entre las dos masas de prueba es \emph{suficientemente peque\~na}, podemos encontrar una expresión explícita simple para su aceleración relativa. Considere que las posiciones de estas masas son $\vec{x}$ y $\vec{x}+\delta\vec{x}$.
Podemos entonces, a primer orden en $\delta\vec{x}$, escribir:
\begin{equation}
a_i(\vec{x}+\delta\vec{x})=a_i(\vec{x})+\delta x^j\partial_j a_i(\vec{x}).
\end{equation}
Esta aproximación es buena si $|\delta\vec{x}|\ll |a_i|/|\partial_j
a_i|$. Definimos el \textbf{tensor de mareas} $K_{ij}$ por
\begin{equation}\marginnote{Tensor de Mareas}
K_{ij}(\vec{x}):=-\partial_j a_i(\vec{x}),
\end{equation}
(el signo negativo es convencional) de modo que
\begin{equation}
\delta a_i:=a_i(\vec{x}+\delta\vec{x})-a_i(\vec{x})=- K_{ij}\delta x^j.
\end{equation}
Si ahora expresamos la aceleración relativa en t'erminos de la variación temporal de la posición relativa, es decir, $\delta a_i=d^2(\delta x^i)/dt^2$, encontramos 
\begin{equation}\label{desvrel}
\frac{d^2(\delta x^i)}{dt^2}+K_{ij}\delta x^j=0.
\end{equation}
El carácter irrotacional del campo gravitacional es equivalente a la simetría
del tensor de mareas, $K_{ij}=K_{ji}$. Además,
\begin{equation}
K_{ij}=\partial_i\partial_j\phi.
\end{equation}
En t'erminos del tensor de mareas, la ecuación de Poisson (\ref{Poisson}) puede escribirse como
\begin{equation}\label{Kiirho}
%\sum_{i=1}^3
K_{ii}=4\pi G\rho.
\end{equation}
Como veremos posteriormente, el tensor de mareas es el análogo newtoniano del \textbf{tensor de curvatura de Riemann}, y las relaciones \eqref{desvrel} y \eqref{Kiirho} corresponden al límite no-relativista de la \textbf{ecuación de desvío geod'esico} y de las \textbf{ecuaciones de Einstein}, respectivamente.



\section{Observadores acelerados y gravedad: Versión no-rela\-ti\-vis\-ta}
Considere, a'un en el contexto de la mecánica de Newton, una partícula movi'endose sólo bajo la acción de un campo gravitacional $\vec{g}$. Siempre es posible considerar una región suficientemente peque\~na del espacio y un intervalo de tiempo suficientemente corto que permitan aproximar a $\vec{g}$ como \textit{homog'eneo e independiente del tiempo} en aquella región. Como vimos anteriormente, suponiendo la validez del PED, la ecuación de movimiento de toda partícula respecto de un SRI $K$ será
\begin{equation}\label{enmp}
\frac{d^2\vec{x}}{dt^2}=\vec{g},
\end{equation}
cuya solución es una trayectoria parabólica:
\begin{equation}\label{tr1}
\vec{x}(t)=\vec{x}_0+\vec{v}_0t+\frac{1}2\vec{g}t^2.
\end{equation}

Considere ahora un SR $K'$ que \textit{acelera respecto a} $K$, con aceleración constante $\vec{a}$. Consideraremos que la transformación entre las coordenadas asociadas a ambos SR's es 
\begin{equation}\label{tgan}
t'=t, \qquad \vec{x}'=\vec{x}-\frac{1}2\vec{a}t^2.
\end{equation}
Verificamos que nuestra interpretación es consistente ya que los eventos sobre trayectorias en reposo respecto a $K'$, es decir, con $\vec{x}'=\vec{x}'_0$, describen un movimiento uniformemente acelerado respecto al SRI $K$: $(ct,\vec{x})=(ct,\vec{x}'_0+\vec{a}t^2/2)$.

Usando la transformación \eqref{tgan} y la aceleración \eqref{enmp} podemos calcular la aceleración de la partícula respecto a $K'$, obteniendo
\begin{equation}\label{acelprima}
\frac{d^2\vec{x}'}{dt'^2}=\vec{g}-\vec{a}.
\end{equation}
Equivalentemente, la trayectoria respecto a $K'$ es determinada transformando  \eqref{tr1}:
\begin{equation}
\vec{x}'(t)=\vec{x}_0+\vec{v}_0t+\frac{1}2(\vec{g}-\vec{a})t^2.
\end{equation}
\begin{center}
\begin{figure}[H]
\centerline{\includegraphics[height=4cm]{fig/fig-equivalencia.pdf}}
\caption{Equivalencia entre un SRI en ausencia de gravedad y un SR en caida libre.}
\label{fig:equiv2}
\end{figure}
\end{center}

Este resultado establece cierta relación de \textit{equivalencia} en lo que respecta a la mecánica (es decir, al movimiento de cuerpos) en campos gravitacionales \textit{estacionarios y homog'eneos} y en \textit{SR's acelerados}. Por ejemplo, en un sistema de referencia \textit{en caída libre}, es decir, con $\vec{a}=\vec{g}$, tendremos que cada partícula describirá una trayectoria rectilínea con velocidad constante respecto a $K'$, \textit{tal como lo haría en un SRI en el que no existiese campo gravitacional}. Ver figura \ref{fig:equiv1}.
En otras palabras, parece posible ``eliminar'' los efectos de un campo gravitacional (estacionario y homog'eneo) sobre el movimiento de cuerpos, refiriendo estos movimientos a un SR en caída libre\footnote{Ver, por ejemplo, el siguiente \href{http://youtu.be/1ieR8hIXUIg}{video} y tambi'en \href{http://youtu.be/xsNFqMtNZvI}{\'este}.}.
Análogamente, si consideramos una región muy lejos de todo cuerpo masivo que genere un campo gravitacional ($\vec{g}=\vec{0}$), entonces es posible ``simular'' (los efectos mecánicos de) un campo gravitacional (estacionario y homog'eneo) describiendo los movimientos desde un SR acelerado (por ejemplo, con aceleración $\vec{a}=-\vec{g}$). Ver figura\footnote{Adaptada a partir de  \href{http://commons.wikimedia.org/wiki/File:Elevator_gravity.svg}{esta} figura original.} \ref{fig:gya}.
\begin{figure}[H]
 \begin{center}
\includegraphics[height=4cm]{fig/fig-gravedad-y-aceleracion.pdf}
\caption{Equivalencia entre campo gravitacional y aceleración.}
\label{fig:gya}
\end{center}
\end{figure}
Note que la existencia de esta equivalencia entre efectos gravitacionales y efectos inerciales depende crucialmente del carácter universal de la interacción gravitacional (en otras palabras, de la validez del PED).

\section{Principio de Equivalencia de Einstein y Sistemas de Referencia Localmente inerciales}
Tal como hemos discutido, la experiencia suministra evidencia a favor de que la interacción gravitacional posee las siguientes características:
\begin{quotation}
En una región (espacio-temporal) suficientemente peque\~na (donde las inhomogeneidades del campo puedan ser despreciadas), y respecto a un SR \textit{en caída libre}, las trayectorias de \textit{todo} (peque\~no) cuerpo, libre de fuerzas no-gravitacionales, \textit{son líneas rectas con velocidad constante}.
\end{quotation}

Note además que suficientemente lejos de otras distribuciones de masa, donde el campo gravitacional pueda considerarse como nulo, los SR's en caída libre son los usuales SRI's. En este sentido, al menos en lo que respecta a la trayectoria de cuerpos no sometidos a fuerzas no-gravitacionales, los SR's en caída libre juegan el mismo rol físico que los SRI's en ausencia de gravedad. Einstein supuso que estos SR's en caída libre son \textit{en toda situación}, es decir\marginnote{Principio de Equivalencia de Einstein}, \textit{para todo tipo de fenómeno físico} (no sólo mecánico, tambi'en por ejemplo, electromagn'etico), equivalentes a los SRI's en ausencia de gravitación. Por esta razón, y dada la extensión finita de estos SR's en caída libre, 'estos son tambi'en llamados \textbf{Sistemas de Referencia Localmente Inerciales} (SRLI's). Este supuesto es tambi'en llamado \textbf{Principio de Equivalencia Fuerte} (PEF), en contraste al PED, que se refiere sólo a la \emph{mecánica} de los cuerpos (macroscópicos).

La primera referencia conocida al PEF se encuentra en el artículo de 1907 de Einstein \cite{Einstein07}, cuyo título podría traducirse ``Sobre el Principio de Relatividad y las consecuencias que de 'el se desprenden".
 Casi al final de este trabajo (página 454) Einstein escribe 
\begin{quotation}
``Wir... wollen ... in folgenden die v\"ollige physikalische Gleichwertigkeit von Gravitationsfeld und entsprechender Beschleunigung des Bezugssystems annehmen",
\end{quotation}
que se traduciría como ``queremos suponer la completa equivalencia física de un campo gravitacional y la correspondiente aceleración del sistema de referencia". Acto seguido (en el resto del paper) Einstein estudia las primeras consecuencias de esta suposición.
 
El PEF obliga a repensar la existencia y el rol de los Sistemas de Referencia Inerciales. En la mecánica de Newton y en RE los SRI's juegan un papel fundamental. Un SRI es entendido como un sistema de ejes rectos ortogonales respecto a los cuales un cuerpo libre de fuerzas externas se mueve en línea recta con velocidad constante. Esta abstracción resulta entonces ser consistente y 'util sólo en ausencia de interacción gravitacional. En presencia de gravitación, por otro lado, \textit{no existen SRI's con extensión infinita}. Esto es debido a que, por un lado, de acuerdo al PED, \textit{todas} las trayectorias de cuerpos son afectadas por la gravedad, es decir, no existen cuerpos libres de esta interacción. Además, en presencia de campos gravitacionales (no homog'eneos) no existe ning'un SR (``rígido'' y de extensión infinita) respecto al cual los cuerpos se muevan en forma rectilínea y uniforme.
Sin embargo, en presencia de gravitación los SRLI's sí tienen realidad física, pero necesariamente tienen una extensión finita en el espaciotiempo.
Como los fenómenos físicos no-gravitacionales conocidos son descritos exitosamente en el marco de la Teoría Especial de la Relatividad, Einstein supuso en su teoría de Relatividad General, que incorpora la gravitación, que \textit{es en los SRLI's en caída libre donde son válidas las leyes conocidas en la teoría de RE}.

En resumen, en presencia de un campo gravitacional general no existen SRI's \emph{globales} (de extensión infinita). No obstante, de acuerdo al PEF sí es posible encontrar SR's en regiones peque\~nas (los SRLI's, es decir, SR's en ``caida libre'') donde las leyes de RE son válidas. Un campo gravitacional no nulo está entonces caracterizado por el hecho que los SRLI's no pueden ``unirse'' para formar un SRI global. Además, si bien en un SRLI no es posible detectar efectos de la gravedad, sí es posible hacerlo comparando cantidades físicas en \textit{distintos} SRLI's. Por ejemplo, en presencia de campo gravitacional no nulo los SRLI's \textit{aceleran entre sí}, ver figura \ref{fig:SRLI}.
\begin{figure}[H]
\centering\includegraphics[width=6cm]{fig/fig-SRLI.pdf}
\caption{Sistemas de referencia localmente inerciales cayendo hacia la Tierra.}
\label{fig:SRLI}
\end{figure}

Una consecuencia directa del PEF es que \textit{la luz debiese ser deflectada por campos gravitacionales}. En efecto, la hipótesis planteada por el PEF es que en los SRLI's, es decir, SR en caida libre, son válidas las leyes Físicas conocidas en la teoría de RE. En particular las ecuaciones de Maxwell en su forma usual, y sus conocidas implicancias respecto de la propagación de la radiación electromagn'etica, son válidas en estos SRLI's. Como consecuencia, es en estos SRLI's en los que la luz debe(ría) moverse en línea recta con velocidad constante. 
 Por otro lado, respecto a un SR que acelera respecto a estos SRLI's, por ejemplo, en un SR a una distancia fija de la Tierra, la luz debe(ría) curvarse. Ver figura \ref{fig:PEF-luz}.
\begin{figure}[H]
\centering\includegraphics[width=6cm]{fig/fig-gravedad-y-aceleracion-luz.pdf}
\caption{Deflexión de la luz. Ambos SR's aceleran respecto a SRLI's.}
\label{fig:PEF-luz}
\end{figure}
Tal como en este caso simple, relacionado con la trayectoria de la luz en un campo gravitacional, el PEF permite determinar cómo se comporta cualquier sistema físico en presencia de un campo gravitacional no nulo (en una región suficientemente peque\~na), ya que es (o debe ser, de acuerdo a este principio) exactamente lo mismo que ocurre cuando el sistema se describe desde un SR que acelera respecto a uno (localmente) inercial. De hecho, el PEF \textit{unifica} localmente (los efectos de) la gravitación con (los de) la aceleración respecto a SRLI's, puesto que son físicamente indistinguibles. En este sentido en la teoría de gravitación de Einstein la gravedad ``es'' aceleración respecto a SLRI's: el hecho que en una región del espacio se experimente un campo gravitacional respecto a un SR es simplemente una consecuencia de que ese SR acelera respecto a los SRLI's que cubren esa región. Note que estas consideraciones son puramente ``cinemáticas'' en el sentido que no permiten por si solas determinar la distribución, orientación y dinámica de los SRLI's (exactamente cómo y hacia donde ``caen'' cada uno de estos SRLI's), 'este es precisamente el rol de las \textbf{ecuaciones de campo de la teoría}, que discutiremos en el capítulo \ref{capTEG}. No obstante, el PEF determina la forma en que un sistema físico responde a un campo gravitacional dado, tal como discutiremos a continuación, ahora en el contexto relativista.

\subsection{Desvío de la luz*}
\begin{equation}
\Delta t'=\frac{L}{c}, \qquad \Delta y' = 0
\end{equation}
\begin{equation}
\Delta y = \frac{1}{2} g (\Delta t)^2 = \frac{1}{2} g (\Delta t')^2 = \frac{gL^2}{2c^2}
\end{equation}
Si $L\approx 1{\,\rm km} = 10^3{\,\rm m}$, y usando $g\approx 9.8{\,\rm m/s^2}$ entonces
\begin{equation}
\Delta y \approx 5.4\times 10^{-11}{\,\rm m}
\end{equation}
\begin{equation}
\theta \approx \frac{d(\Delta y)}{dL} \approx \frac{gL}{c^2} \approx \times 10^{-13}{\,\rm rad}.
\end{equation}

ángulo total de desvío
\begin{equation}
\Theta = \int d\theta \approx \int \frac{g}{c^2}dL = \frac{1}{2c^2}\int g r d\varphi
\end{equation}
Si $g\approx GM/r^2$ y $r(\varphi)\approx D/\sin\varphi$, $\varphi\in [0,\pi]$ (línea recta), entonces
\begin{equation}
\Theta \approx \frac{GM}{c^2D}\int_0^\pi\sin\varphi\,d\varphi = \frac{2GM}{c^2D}.
\end{equation}

\subsection{Redshift gravitacional*}
 
 \begin{equation}
 y_{\rm f}(t) = ct, \qquad y_{\rm d}(t)=h+\frac{1}{2}gt^2.
 \end{equation}
 \begin{equation}
t_{\rm d} = \frac{c-\sqrt{c^2-2gh}}{g}
 \end{equation}
 \begin{equation}
 v_{\rm d}=gt_{\rm d} = c-\sqrt{c^2-2gh}
 \end{equation}
 \begin{equation}
 z=\frac{v}{c}=1-\sqrt{1-\frac{2gh}{c^2}} \approx 1-\left(1-\frac{gh}{c^2}\right) = \frac{gh}{c^2} = \frac{\Delta\phi}{c^2}
 \end{equation}
 \begin{equation}
z = \frac{\lambda_{d}-\lambda_{\rm e}}{\lambda_{\rm e}} = \frac{\nu_{\rm e}}{\nu_{\rm d}}-1
 \end{equation}
 \begin{equation}
 \nu_{\rm e}-\nu_{\rm d} \approx \nu_{\rm d}\frac{\Delta\phi}{c^2}
 \end{equation}
  \begin{equation}
 E_{\rm e}-E_{\rm d} \approx E_{\rm d}\frac{\Delta\phi}{c^2}
 \end{equation}
   \begin{equation}
 E_{\rm e} \approx E_{\rm d} + \left(\frac{E_{\rm d}}{c^2}\right)\Delta\phi
 \end{equation}
\section{Observadores acelerados y gravedad: Versión  relativista}

Tal como Einstein \cite{Einstein56}, seguiremos considerando que un sistema coordenado (SC) está asociado a un SR de modo que un \textit{cambio} de SR queda descrito por una cierta \textit{transformación de  coordenadas} (más adelante, sin embargo, veremos que es posible separar estos conceptos: un sistema de coordenadas no necesita siempre estar asociado a un sistema de referencia).

En RE la ecuación que describe el movimiento de una part{\'\i}cula
en ausencia de fuerzas externas, respecto a un SRI $K$, es
\begin{equation}
\frac{d^2x^\mu}{d\tau ^2}=0, \label{tsri}
\end{equation}
donde $x^\mu(\tau )$ es la trayectoria de la part{\'\i}cula,
expresada en coordenadas (pseudo-)cartesianas $x^\mu=(ct,\vec{x})$, y donde $\tau $ es el correspondiente tiempo propio.

De acuerdo a lo discutido en las secciones anteriores, el PEF implica que (\ref{tsri}) será tambi'en la ecuación de movimiento de un cuerpo bajo la acción de la gravedad (pero libre de otras fuerzas) \textit{respecto a SRLI's}.

Deseamos ahora transformar la ecuación (\ref{tsri}) para expresarla en t'erminos de las coordenadas asociadas a un SR que no sea un SRLI, es decir, un SR que \textit{acelera} respecto a los SRLI's. La transformación de coordenadas correspondiente debe necesariamente ser \textit{no-lineal} (y mezclar coordenadas espaciales y temporales) para poder describir un cambio a un SR con \textit{aceleración} relativa (las transformaciones de Lorentz, que describen cambios entre SR's con velocidad relativa constante, son lineales). Consideraremos entonces una \textbf{transformación general de coordenadas} (TGC) $x^\mu\rightarrow \bar{x}^\mu(x)$ que en el caso considerado aquí describe un cambio desde el (SC asociado al) SRLI $K$ hasta (el SC asociado a) un SR $\bar{K}$ que en general no será localmente inercial\footnote{El nuevo SR será tambi'en localmente inercial si la transformación es una transformación de Lorentz. Además, una TGC no sólo puede aplicarse para describir transformaciones a SR's con movimiento relativo, sino que tambi'en al caso en que se usan \textit{coordenadas curvilíneas en un mismo sistema de referencia}. Por ejemplo, la transformación de coordenadas
$x^\mu\to\bar{x}^\mu$ con $x^\mu=(ct,x,y,z)$ y $\bar{x}^\mu=(ct,r,\theta,\varphi)$, donde $(r,\theta,\varphi)$ son las usuales coordenadas esf'ericas se interpreta
como un simple cambio de coordenadas espaciales, \textit{en el mismo SR}. En general, la TC estará ligada a un cambio de SR si mezcla coordenadas temporales y espaciales. Ver por ejemplo (\ref{tgan}).}. Respecto al SC $\bar{x}^\mu$ la ecuación de movimiento (\ref{tsri}) adopta la forma
\begin{equation}
\boxed{\frac{d^2\bar{x}^\mu}{d\tau ^2}+\bar{\Gamma }_{\ \nu
\lambda }^\mu\frac{d\bar{x}^{\nu }}{d\tau }\frac{d\bar{x}%
^\lambda }{d\tau }=0.} \label{tsrni}\marginnote{Ec. de mov. en coord. arbitrarias}
\end{equation}
En efecto,
\begin{eqnarray}
\frac{d^2\bar{x}^\mu}{d\tau ^2} &=&\frac{d}{d\tau }\left(
\frac{d\bar{x}^\mu}{d\tau } \right) \\
&=&\frac{d}{d\tau }\left( \frac{\partial \bar{x}^\mu}{\partial x^{\nu
}}\frac{dx^{\nu }}{d\tau }\right) \\
&=&\frac{\partial \bar{x}^\mu}{\partial x^\nu } \frac{d^2x^{\nu }}{d\tau
^2}+\frac{\partial ^2\bar{x}^\mu }{\partial x^{\nu }\partial x^\lambda
}\frac{dx^\lambda }{d\tau }\frac{dx^{\nu }}{d\tau } \\
&=&\frac{\partial ^2\bar{x}^\mu }{\partial x^{\nu }\partial x^\lambda
}\frac{dx^\lambda }{d\tau }\frac{dx^{\nu }}{d\tau } \\
&=&-\bar{\Gamma }_{\ \nu
\lambda }^\mu\frac{d\bar{x}^{\nu }}{d\tau }\frac{d\bar{x}%
^\lambda }{d\tau } ,
\end{eqnarray}
donde definimos
\begin{equation}
\boxed{\bar{\Gamma }_{\ \nu\lambda }^\mu(\bar{x}):=-\frac{\partial
^2\bar{x}^\mu }{\partial x^\alpha \partial x^\beta }\frac{\partial
x^\alpha }{\partial \bar{x}^{\nu }}\frac{\partial x^\beta }{\partial
\bar{x}^\lambda }=\frac{\partial\bar{x}^\mu}{\partial x^\sigma
}\frac{\partial ^2x^\sigma }{\partial \bar{x}^{\nu }\partial \bar{x}^\lambda }.}
\label{defGammaSR}
\end{equation}
As{\'\i}, la ecuación del movimiento de una part{\'\i}cula libre en un SC
general posee, adicionalmente al usual t'ermino proporcional a la
segunda derivada de la 4-posición, un t'ermino \textit{bilineal} en la 4-velocidad. Este segundo t'ermino describe las llamadas ``fuerzas inerciales''\footnote{'Este es el t'ermino que, en los límites apropiados, reproduce la aceleración de Coriolis en un SR rotante, o el segundo t'ermino del lado derecho de (\ref{acelprima}).}.

Simultáneamente, el elemento de línea (el tiempo propio, para el caso de separaciones tipo tiempo), adopta la forma
\begin{equation}\marginnote{Elemento de línea en coord. arbitrarias}
 \boxed{ds^2=\bar{g}_{\mu\nu }(\bar{x})\,d\bar{x}^\mu d\bar{x}^{\nu },}
\end{equation}
con
\begin{equation}\marginnote{m'etrica en coord. arbitrarias}
\boxed{\bar{g}_{\mu\nu }(\bar{x}):=\eta _{\lambda \rho }\,\frac{\partial x^\lambda }{\partial \bar{x}^\mu}\frac{\partial x^\rho }{\partial \bar{x}^{\nu }}.} \label{tm}
\end{equation}
Recuerde que tanto en \eqref{defGammaSR} como en \eqref{tm} las coordenadas $x^\mu$ están asociadas a un SRLI.

Derivando (\ref{tm}) y usando (\ref{defGammaSR}) es posible expresar las
componentes de $\bar{\Gamma }$ directamente en t'erminos de las (derivadas de las)
componentes de $\bar{g}$:
\begin{equation}
\boxed{\bar{\Gamma }_{\ \nu\lambda }^\mu\equiv \frac{1}2\bar{g}^{\mu\rho}\left(
\bar{\partial}_\nu\bar{g}_{\lambda\rho}+\bar{\partial}_\lambda\bar{g}_{\nu\rho}
-\bar{\partial}_\rho\bar{g}_{\nu\lambda}\right) ,}
\end{equation}
donde $\bar{g}^{\mu\rho}(\bar{x})$ son las componentes de la \textbf{inversa} de
$\bar{g}_{\mu\nu}(\bar{x})$, definida de modo que (en cada evento)
\begin{equation}
\bar{g}^{\mu\rho}(\bar{x})\,\bar{g}_{\rho\nu}(\bar{x})=\delta^\mu_\nu.
\end{equation}


Vemos con esto que, en peque\~nas regiones del espaciotiempo, pero respecto a SR's generales (asociados a coordenadas $\bar{x}$), los efectos del campo gravitacional sobre la trayectoria de cuerpos quedan descritos por las cantidades $\bar{\Gamma }_{\ \nu\lambda }^\mu$ (con 40 componentes linealmente independientes!) que miden ``que tan no-(localmente-)inercial'' es el SR. \textit{Simultáneamente}, los coeficientes $\bar{g}_{\mu\nu}(\bar{x})$ que determinan el elemento de línea, y por consiguiente el tiempo propio, \textit{dejan de ser constantes y diagonales}  (10 componentes linealmente independientes!).

Para estudiar cómo cambian estas cantidades entre SC's arbitrarios (es decir, ninguno de ellos asociados, en general, a SRLI's), efectuamos una segunda TGC $\bar{x}^\mu\rightarrow
\tilde{x}^\mu(\bar{x})$. Como es de esperar, se encuentra que la ecuación de movimiento de la part{\'\i}cula (libre de fuerzas no-gravitacionales), expresada en coordenadas $\tilde{x}^\mu$ es nuevamente de la forma (\ref{tsrni}), es decir,
\begin{equation}
\frac{d^2\tilde{x}^\mu}{d\tau ^2}+\tilde{\Gamma }_{\ \nu
\lambda }^\mu\frac{d\tilde{x}^{\nu }}{d\tau }\frac{d\tilde{x}%
^\lambda }{d\tau }=0, \label{tsrni2}
\end{equation}
donde
\begin{eqnarray}
\tilde{\Gamma }_{\ \nu\lambda }^\mu(\tilde{x})&:=&
\frac{\partial \tilde{x}^\mu}{\partial x^\sigma }\frac{\partial
^2x^\sigma }{%
\partial \tilde{x}^{\nu }\partial \tilde{x}^\lambda } \\
&=&\frac{\partial
\tilde{x}^\mu}{\partial \bar{x}^\sigma }\frac{\partial
\bar{x}^\rho }{\partial \tilde{x}^{\nu }}\frac{\partial
\bar{x}^{\eta }}{\partial \tilde{x}^\lambda }\bar{\Gamma }_{\ \rho \eta
}^\sigma (\bar{x})+\frac{\partial \tilde{x}^\mu}{%
\partial \bar{x}^\rho }\frac{\partial ^2\bar{x}^\rho }{%
\partial \tilde{x}^{\nu }\partial \tilde{x}^\lambda }. \label{tigam} \marginnote{ley de transf. de conexión}
\end{eqnarray}

Vemos que los coeficientes $\Gamma$ transforman \textit{inhomog'eneamente} bajo
una TGC. Esta propiedad es precisamente la que posibilita que $\Gamma$ sea nulo
en coordenadas pseudo-cartesianas asociadas a SRLI's, pero distinto de cero en coordenadas asociadas a SR's que aceleran respecto a los primeros. Más precisamente, los coeficientes $\Gamma$ \emph{transforman como una conexión} bajo una TGC.

Por otro lado, las componentes de la m'etrica en el SR $\tilde{K}$, con coordenadas $\tilde{x}$, están relacionadas con las componentes en $\bar{K}$ por medio de
\begin{equation}
 \tilde{g}_{\mu\nu}(\tilde{x})=\frac{\partial\bar{x}^\lambda}{\partial \tilde{x}^\mu}
\frac{\partial\bar{x}^\rho}{\partial \tilde{x}^\nu}\,\bar{g}_{\lambda\rho}(\bar{x}).  \marginnote{ley de transf. de m'etrica}
\end{equation}

M'etricas no constantes, conexiones no nulas (así como curvaturas y torsiones, etc.) son objetos matemáticos definidos usualmente en el contexto de la \textit{geometría diferencial} y en particular de la \textit{geometría riemanniana}, o geometría de \textit{espacios curvos}. En el capítulo \ref{cap:tensores} resumiremos algunos aspectos básicos de este vasto tema.

%\section{Relatividad especial y gravitación newtoniana: un conflicto}
%
%En la teoría de la Relatividad Especial, los eventos ocurren en el
%espaciotiempo 4-dimen\-sio\-nal. Existen además observadores
%``privilegiados'', los observadores inerciales. Ellos son entendidos como marcos
%infinitamente extendidos en espacio y tiempo en los que cuerpos \textit{libres}
%(que no son influenciados por ning'un otro cuerpo) se mueven con velocidad
%constante en líneas rectas en
%el sentido geom'etrico euclidiano. A estos SRI's asociamos un conjunto de
%coordenadas \textit{inerciales}, tambi'en llamadas (pseudo-)cartesianas,
%$x^\mu=(x^0,x^1,x^2,x^3)$. En el espacio vacío y respecto a un SRI no existe
%preferencia entre los distintos puntos e instantes y además no existe una
%dirección preferente. Decimos que el vacío es invariante bajo translaciones
%(espaciales y temporales) y bajo rotaciones. Además, el principio de
%relatividad establece que todos los SRI's, con velocidades relativas constantes
%entre ellos, son físicamente equivalentes, es decir, indistinguibles.
%Equivalentemente, las leyes físicas son las mismas en todos los SRI's. Esto, a
%fin de cuentas, es tambi'en una \textit{observación}, es decir, una generalización de resultados experimentales. En RE estas equivalencias entre SRI's se expresan por medio de la covariancia de las ecuaciones respecto a transformaciones de Lorentz y translaciones (es decir, bajo transformaciones de Poincar\`e).
%
%Una teoría de la interacción gravitacional que sea compatible con los
%principios de la TRE (por ejemplo, con una velocidad máxima de propagación de
%las interacciones) requiere entonces que sus ecuaciones sean covariantes bajo
%TL's, es decir, que puedan ser escritas en t'erminos de vectores y tensores
%respecto a TL's.
%
%\subsection{Teoría escalar de la gravitación*}
%La ecuación newtoniana que determina el campo gravitacional es la ecuación de
%Poisson (\ref{Poisson}). Una posible generalización covariante bajo TL's
%(análoga a la ecuación que satisface el potencial electromagn'etico) es la
%ecuación de onda
%\begin{equation}
%\square\phi=-4\pi G\rho, \label{casicasi}
%\end{equation}
%con $\square=\eta^{\mu\nu}\partial_\mu\partial_\nu$. En este caso la ecuación
%de Poisson para el potencial $\phi$ se encuentra en el caso límite de campos
%estáticos.
%
%Un primer candidato a una teoría relativista de la gravitación podría
%considerar que el campo gravitacional $\phi$ es determinado por la ecuación
%(\ref{casicasi}) en el caso general en que tanto $\phi$ como $\rho$ sean
%dependientes del tiempo. Pero, ?`cuál es el significado de la fuente $\rho$ en
%este caso?. En el caso de un fluido sin presión (polvo), o un conjunto de
%partículas puntuales donde todas las componentes se mueven con la misma
%velocidad, es posible construir un escalar proporcional a la densidad de masa
%(la densidad de masas en reposo $\rho$ vista anteriormente). Por otro
%lado, en RE la masa no es entendida como una cantidad independiente, sino
%sólo como una componente de la energía de los cuerpos. Así,
%en una teoría relativista de la gravitación esperamos que todo el contenido de
%energía de un cuerpo aporte su a masa gravitacional. Esto significa que podemos
%intentar reemplazar densidad de masa por densidad de energía como fuente del
%campo gravitacional. De este modo, para completar nuestra generalización
%(\ref{casicasi}) necesitamos una cantidad escalar bajo TL's que contenga la
%densidad de energía y que se reduzca a la usual densidad de masa en los casos
%límites apropiados.
%
%A partir del tensor de energía-momentum $T^{\mu\nu}$ de una distribución de
%materia podemos construir el siguiente escalar:
%\begin{equation}
%T:=T^\mu_{\ \mu}=\eta_{\mu\nu}T^{\mu\nu}.
%\end{equation}
%En el caso de un fluido ideal,
%\begin{equation}
%T=\left( \rho +\frac{p}{c^2}\right) u^\mu u_\mu-p\delta^\mu_\mu=\left( \rho
%+\frac{p}{c^2}\right) c^2-4p=\rho c^2-3p.
%\end{equation}
%Para ``materia no-relativista'', $p\ll \frac{1}{3}\rho c^2$ (verificar esta
%condición para un gas ideal) tenemos $T=\rho c^2$, de modo que podemos
%postular
%\begin{equation}
%\square\phi=-\kappa T, \label{casicasi2}
%\end{equation}
%con $\kappa:=\frac{4\pi G}{c^2}$.
%A primera vista, lo anterior define una teoría relativista viable de la
%gravitación. De hecho, la teoría es \textit{matemáticamente consistente}. Sin
%embargo, se encontró que esta \textit{teoría escalar de la gravitación no
%describe adecuadamente las observaciones}. En general, una teoría gravitacional
%escalar no permite describir el fenómeno de deflección de la luz por campos
%gravitacionales, debido a que un campo escalar no puede acoplarse razonablemente
%al campo electromagn'etico (ya que el tensor de enrgía-momentum
%electromagn'etico tiene traza nula). Por otro lado, hoy en día la desviación
%de la luz por campos gravitacionales es un hecho experimental confirmado más
%allá de toda duda. En resumen, debemos considerar otras posibilidades para una
%teoría relativista de la gravitación.
%

