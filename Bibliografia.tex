\begin{thebibliography}{99}

\addcontentsline{toc}{chapter}{Bibliograf'ia}

\bibitem{7}
Guillermo F. Rubilar, {\it Apuntes Curso Electrodin'amica}, {\sl
Universidad de Concepci'on}. 'Ultima versi'on y c'odigo fuente disponible en \url{https://github.com/gfrubi/electrodinamica}.

\bibitem{Will06} C.M. Will, \href{http://www.livingreviews.org/lrr-2006-3}{\em The Confrontation between General Relativity and Experiment}, {\sl Living Rev. Relativity} {\bf 9} (2006) 3. [Online Article]: cited [2012/01/17].

\bibitem{Turyshev09} S.G. Turyshev, {\em Experimental Tests of General Relativity: Recent Progress and Future Directions}, {\sl Phys. Usp.} {\bf 52} (2009) 1-27. \url{http://arxiv.org/abs/0809.3730}.

\bibitem{Turyshev08} S.G. Turyshev, \href{http://www.annualreviews.org/doi/abs/10.1146/annurev.nucl.58.020807.111839}{\em Experimental Tests of General Relativity}, {\sl Annu. Rev. Nucl. Part. Sci.} {\bf 58} (2008) 207-248. 

\bibitem{Koester76} L. Koester, \href{http://dx.doi.org/10.1103/PhysRevD.14.907}{\em Verification of the equivalence of gravitational and inertial mass for the neutron}, {\sl Phys. Rev.} {\bf D 14} (1976) 907-909.

\bibitem{Zhou15} L. Zhou et al., \href{http://dx.doi.org/10.1103/PhysRevLett.115.013004}{\em Test of Equivalence Principle at $10^{-8}$ Level by a Dual-Species Double-Diffraction Raman
Atom Interferometer}, {\sl Phys. Rev. Lett.} {\bf 115} (2015) 013004.

\bibitem{Duan16} X. Duan et al., \href{http://dx.doi.org/10.1103/PhysRevLett.117.023001}{\em Test of the Universality of Free Fall with Atoms in Different Spin Orientations}, {\sl Phys. Rev. Lett.} {\bf 117} (2016) 023001. 

\bibitem{Turm} Zentrum f\"ur Angewandte Raumfahrttechnologie und Mikrogravitation (Center of Applied Space Technology and Microgravity), \url{http://www.zarm.uni-bremen.de}.

\bibitem{STEP} \url{http://einstein.stanford.edu/STEP}.

\bibitem{Einstein07} A. Einstein, {\it \"Uber das Relativit\"atsprinzip und die aus demselben gezogene Folgerungen}, {\sl Jahrbuch der Radioaktivitaet und Elektronik} {\bf 4} (1907) 411-492.

\bibitem{Einstein56} A. Einstein, \emph{The Meaning of Relativity}, (Fifth
Edition), Princeton University Press (1956).

\bibitem{Einstein11} A. Einstein, {\it \"Uber den Einfluss der Schwerkraft auf die Ausbreitung des Lichtes}, {\sl Annalen der Physik} {\bf 35} (1911) 898-908.

\bibitem{PR60} R.V. Pound and G.A. Rebka, {\em Apparent weight of photons}, {\sl Phys. Rev. Lett.} {\bf 4} (1960) 337-341.

\bibitem{PS64} R.V. Pound and J.L. Snider, {\em Effect of Gravity on nuclear resonance}, {\sl Phys. Rev. Lett.} {\bf 13} (1964) 539-540.

\bibitem{PS65} R.V. Pound and J.L. Snider, {\em Effect of Gravity on Gamma radiation}, {\sl Phys. Rev.} {\bf 140} (1965) B778.

\bibitem{Snider72} J.L. Snider, {\em New measurement of the solar gravitational redshift}, {\rm Phys. Rev. Lett.} {\bf 28} (1972) 853.

\bibitem{Chou2010} C. W. Chou, D. B. Hume, T. Rosenband and D. J. Wineland, {\em Optical Clocks and Relativity}, {\sl Science} {\bf 329} (2010) 1630. \url{http://www.sciencemag.org/content/329/5999/1630.full.html}.

\bibitem{Einstein16} A. Einstein, \href{https://docs.google.com/open?id=0B4RSIcYW5V0HMWY1YzM2MjAtMmJhMy00NmQ5LWFhMjEtNjdlODhhNjcxZTFk}{\it Die Grundlage der allgemeinen Relativitästheorie}, {\sl Annalen der Physik}, Band {\bf 49} (1916). Traducci'on al ingl'es \href{http://goo.gl/E2YFn}{aqu\'i}.

\bibitem{Einstein17} A. Einstein, \href{https://goo.gl/A8miy}{\it Kosmologische Betrachtungen zur allgemeinen Relativitätstheorie}, {\sl Sitzung der physikalisch mathematischen Klasse} (1917) 142.

\bibitem{Einstein15} A. Einstein, \href{https://docs.google.com/open?id=0B4RSIcYW5V0Hd0haajhUS0tRems}{\it Erkl\"arung der Perihelbewegung des Merkur aus der allgemeinen Relativit\"atstheorie}, {\sl Sitzungsberichte der koeniglich preussischen Akademie der Wissenschaften} (1915) 831.



\bibitem{Carroll97} Sean M. Carroll. {\it Lecture Notes on General Relativity} (1997). \url{http://arxiv.org/abs/gr-qc/9712019v1}.

\bibitem{1} Ignazio Ciufolini and John A. Wheeler, {\it Gravitation and Inertia},
{\sl Princeton University Press, New Jersey}, 1st Edition (1995) ISBN 0-691-03323-4.

\bibitem{FC} F. de Felice and C.J.S. Clarke, {\em Relativity on curved manifolds}, Cambridge University Press (1990).

\bibitem{Dinverno} R. D'~Inverno, {\em Introducing Einstein's Relativity}, Clarendon Press, Oxford, (1992).




\bibitem{LeVerrier} U.J. Le Verrier, {\em Theorie du mouvement de Mercure}, {\sl Annales de l'\ Observatoire imperial de Paris}  (1859). \url{http://adsabs.harvard.edu/abs/1859AnPar...5....1L}.

\bibitem{Clemence47} G.M. Clemence, \href{http://rmp.aps.org/abstract/RMP/v19/i4/p361_1}{\it The relativity effect in planetary motions}, {\sl Rev. Mod. Phys.} {\bf 19} (1947) 361-364.

\bibitem{NW86} A.M. Nobili and C.M. Will, {\it The real value of Mercury's perihelion advance}, {\sl Nature} {\bf 320} (1986) 39-41. \url{http://www.nature.com/nature/journal/v320/n6057/abs/320039a0.html}.

\bibitem{OR94} H. Ohanian and R. Ruffini, {\em Gravitation and Spacetime}. Second Edition. W.W. Norton \& Company, New York, (1994).

\bibitem{Taylor93} J.H. Taylor, {\em Pulsar timing and relativistic gravity}, {\sl Class. Quantum. Grav.} {\bf 10}  (1993) S167.



\bibitem{MTW73} C.W. Misner, K.S. Thorne and J.A. Wheeler, {\em Gravitation},
W.H. Freeman and Company, San Francisco, (1973).

\bibitem{CODATA00}P.J. Mohr and B.N. Taylor, {\it CODATA recommended values of
the fundamental physical constants: 1998*}, {\sl Rev. Mod. Phys.} {\bf 72}
(2000) 351.

\bibitem{Will88} C.M. Will, {\em Henry Cavendish, Johann von Soldner, and the deflection of light},  {\sl Am. J. Phys.} {\bf  56} (1988) 413--415. \url{http://dx.doi.org/10.1119/1.15622}.

\bibitem{DED19}F. W. Dyson, A. S. Eddington and C. Davidson, {\em A Determination of the Deflection of Light by the Sun's Gravitational Field, from Observations Made at the Total Eclipse of May 29, 1919}, \href{http://www.jstor.org/stable/91137}{{\sl Phil. Trans. R. Soc. Lond.} {\bf A 220} (1920) 291-333}.

\bibitem{SDLG04} S. S. Shapiro, J. L. Davis, D. E. Lebach and J. S. Gregory, {\em Measurement of the Solar Gravitational Deflection of Radio Waves using Geodetic Very-Long-Baseline Interferometry Data, 1979-1999}, \href{http://link.aps.org/doi/10.1103/PhysRevLett.92.121101
}{{\sl Phys. Rev. Lett.} {\bf 92} (2004) 121101}.

\bibitem{Shapiro64} I. I. Shapiro,  {\em Fourth test of General Relativity}, {\sl Phys. Rev. Lett.} {\bf 13} (1964) 789-791.

\bibitem{Shapiro71} I. I. Shapiro, et al. {\em Fourth test of General Relativity: New Radar Result}, {\sl Phys. Rev. Lett.} {\bf 26} (1971) 1132-1135.

\bibitem{Demorest} Demorest, P., Pennucci, T., Ransom, S. et al., \textit{A two-solar-mass neutron star measured using Shapiro delay}. \href{https://doi.org/10.1038/nature09466}{Nature \textbf{467}, 1081–1083 (2010)}.

\bibitem{Sch96}M. Schneider, {\em Himmelsmechanik, Band III:
Gravitationstheorie}, Spektrum Akad. Verlag, Heidelberg (1996).

\bibitem{Wald84} R.~M.~Wald, {\em General Relativity},  Chicago Univ. Press (1984).

\bibitem{Wei72} S. Weinberg, {\em Gravitation and Cosmology}, Wiley, New
York, (1972).

\bibitem{Luminet98} J.P. Luminet, {\em Black Holes : A General Introduction}, \href{http://dx.doi.org/10.1007/978-3-540-49535-2\_1}{DOI 10.1007/978-3-540-49535-2\_1}, \url{https://arxiv.org/abs/astro-ph/9801252}.


%bibliografia segunda parte del curso




\bibitem{2}
James B. Hartle, {\it Gravity: An Introduction to Einstein's General Relativity},
{\sl Addison Wesley}, 1st Edition (2003) ISBN 0-8053-8662-9.

\bibitem{4}
Kip S. Thorne, {\it Gravitomagnetism, Jets in Quasars and the Stanford Gyroscope Experiment; del libro: ``Near Zero: New Frontiers of Physics''},
{\sl W.H. Freeman and Company, New York}, 1st Edition (1988).

\bibitem{6}
L.I. Schiff, {\it Motion of a Gyroscope according to Einstein's Theory of Gravitation},
{\sl Institute of Theorethical Physics, Stanford University}, Communicated in April 19, 1960.



\bibitem{Everitt11}
C. W. F. Everitt et al., {\it Gravity Probe B: Final Results of a Space Experiment to Test General Relativity}, \href{http://dx.doi.org/10.1103/PhysRevLett.106.221101}{{\it Phys. Rev. Lett.} {\bf 106}, 221101 (2011)}.

\bibitem{Maggiore} M. Maggiore, {\em  Gravitational waves. Volume 1: Theory and Experiments}, Oxford, New York (2008).

\bibitem{PM63}
P. Peters and J. Mathews, {\em Gravitational Radiation from Point Masses in a Keplerian Orbit}, {\sl Phys. Rev.} {\bf 131} (1963), 435-440. \url{http://link.aps.org/doi/10.1103/PhysRev.131.435}.

\bibitem{WT05} J.~M.~Weisberg and J.~H.~Taylor, {\em Relativistic binary pulsar B1913+16: Thirty years of observations and analysis}, {\sl ASP Conf.\ Ser.}\  {\bf 328}, 25 (2005). [\href{http://arxiv.org/abs/astro-ph/0407149}{astro-ph/0407149}].

\bibitem{Oppenheimer39enero} J.R. Oppenheimer y G.M. Volkoff {\em On Massive Neutron Cores}, {\sl Physical Review} {\bf 55}, 374 (1939).

\bibitem{Oppenheimer39julio} J.R. Oppenheimer y H.Snyder {\em On Continued Gravitational Contraction}, {\sl Physical Review} {\bf 56}, 455 (1939).

\bibitem{Chandra64} S. Chandrasekhar {\em The Dynamical Instability of Gaseous Masses Approaching the Schwarzschild Limit in General Relativity}, {\sl Astrophysical Journal} {\bf 140}, 417 (1964).

\bibitem{Bardeen66} J.M. Bardeen, K.S. Thorne y D.W. Meltzer {\em A Catalogue of Methods for Studying the Normal Modes of Radial Pulsation of General-Relativistic Stellar Models}, {\sl Astrophysical Journal} {\bf 145}, 505 (1966).

\bibitem{Fliessbach} T. Flie\ss bach {\em Allgemeine Relativit\"atstheorie}, Spektrum Lehrbuch, (2006).

\bibitem{Misner73} C.W. Misner, K.S. Thorne and J.A. Wheeler, {\em Gravitation},
W.H. Freeman and Company, San Francisco, (1973).

\bibitem{Weinberg72} S. Weinberg, {\em Gravitation and Cosmology}, Wiley, New
York, (1972).

\bibitem{Shapiro83} S.L. Shapiro y S.A. Teukolsky {\em Black Holes, White Dwarfs and Neutron Stars: The Physics of Compact Objects}, John Wiley \& Sons, New York, (1983).

\bibitem{Tolman34} R.C. Tolman {\em Relativity, Thermodynamics and Cosmology}, Oxford University Press, Oxford, (1934).

\bibitem{Chandra39} S. Chandrasekhar {\em An Introduction to the Study of Stellar Structure}, Dover Publications, New York, (1939).


\bibitem{Zeldovich71} Y.B. Zel'dovich y I.D. Novikov {\em Stars and Relativity}, Dover Publications, New York, (1971).

\bibitem{Mathews71} J. Mathews y R.L. Walker {\em Mathematical Method of Physics, $2^{\rm nd}$ edition}, Addison-Wesley, (1971).

\bibitem{Morse53} P.M. Morse y H. Feshbach {\em Methods of Theoretical Physics, part I}, McGraw-Hil Book Company,  (1953).

\bibitem{Kerr63} R.P. Kerr, {\em Gravitational Field of a Spinning Mass as an Example of Algebraically Special Metrics}, \href{https://doi.org/10.1103/PhysRevLett.11.237}{{\sl Physical Review Letters } {\bf 11}, 237 (1963)}.

\bibitem{Kerr65} R.P. Kerr and A. Schild, {\em A new class of vacuum solutions of the Einstein field equations}. In: Atti del Convegno sulla Relativita Generale: Problemi dell’Energia e Onde Gravitazionali. G. Barbèra Editore, Firenze 1965, pp. 1–12. Reprinted in \href{http://dx.doi.org/10.1007/s10714-009-0856-0}{{\sl Gen. Rel. Grav.} {\bf 41} (2009) 2485--2499}. 

\bibitem{Straumann04} N. Straumann {\em General Relativity With Applications to Astrophysics}, Springer, (2004)

\bibitem{Penrose71} R. Penrose y R. M. Floyd {\em Extraction of Energy Rotational from a Black Hole}, {\sl Nature Physical Sciences} {\bf 229}, 177-179 (1971)

\bibitem{Bardeen72} James M. Bardeen, William H. Press and Saul A. Teukolsky {\em Rotating Black Holes: Locally Nonrotating Frames, Energy Extraction, and Scalar Synchrotron Radiation}, {\sl The Astrophysical Journal} {\bf 178}, 347-369 (1972)
\end{thebibliography}
