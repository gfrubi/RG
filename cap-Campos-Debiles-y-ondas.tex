\chapter{Campos gravitacionales débiles y ondas gravitacionales}\label{capdebil1}

\section{Expansión en potencias de \texorpdfstring{$G$}{G}}
Las ecuaciones de Einstein son no-lineales en la métrica. Por esto, sus
soluciones dependen de forma nolineal del tensor de energía-momentum de la
materia (que asumimos conocido). Como $T^{\mu\nu}$ aparece al lado derecho de las ecuaciones de Einstein siempre multiplicado por la constante de gravitación $G$, es decir, en la combinación $GT^{\mu\nu}$, entonces las componentes métricas dependerán no-linealmente de $G$. Podemos verificar esta propiedad, por ejemplo, en el caso de la solución de Schwarzschild.

Asumiendo que el \textit{campo gravitacional es débil} (curvatura pequeña), pero \textit{no necesariamente estacionario}, es posible desarrollar un \textit{método perturbativo}
para expresar las soluciones de las ecuaciones de Einstein como una serie de
términos, cada uno proporcional a una potencia dada de la constante de gravitación $G$.

%Aquí consideraremos sólo la aproximación a primer orden.
% Además las velocidades de
% las partículas no necesitan ser pequeñas comparadas con la velocidad de la
% luz.

Dividimos la métrica en
\begin{equation} \label{exp01}
g_{\mu\nu}=\eta_{\mu\nu}+h_{\mu\nu}, \qquad \left|h_{\mu\nu}\right|<< 1,
\end{equation}
donde $\eta$ es la métrica plana, es decir, $Riemann(\eta)=0$ y usaremos
coordenadas ``cuasi-inerciales'' $x^\mu$ tales
que $\eta_{\mu\nu}=diag(+1,-1,-1,-1)$. Además, separaremos la perturbación $h_{\mu\nu}$ en una serie de potencias de $G$, de modo que
\begin{equation}
h_{\mu\nu}=h^{(1)}_{\mu\nu}+h^{(2)}_{\mu\nu}+h^{(3)}_{\mu\nu}+\cdots,
\end{equation}
donde $h^{(n)}_{\mu\nu}$, $n=0,1,\cdots$, denota el término proporcional a $G^n$. En particular $h^{(0)}_{\mu\nu}=\eta_{\mu\nu}$.

%Nos concentraremos en el término de primer orden, de modo que
% consideraremos que $h_{\mu\nu}$ es lineal en $G$, lo que denotamos por
% $h=\mathcal{O}(G)$.

% Recuerde que en el sistema solar
% $\left|h_{\mu\nu}\right|\sim \frac{|\phi|}{c^2} \sim
% \frac{GM_\odot}{c^2R_\odot} \sim 10^{-6}$.

Realizaremos una expansión similar para cada cantidad relevante. Por ejemplo,
\begin{align}
\Gamma^\lambda_{\mu\nu} &= \Gamma^{\lambda}_{(1)\mu\nu}+\Gamma^{\lambda}_{(2)\mu\nu}+\Gamma^{\lambda}_{(3)\mu\nu}+\cdots, \\
R_{\mu\nu} &= R^{(1)}_{\mu\nu}+R^{(2)}_{\mu\nu}+R^{(3)}_{\mu\nu}+\cdots,\\
G_{\mu\nu} &= G^{(1)}_{\mu\nu}+G^{(2)}_{\mu\nu}+G^{(3)}_{\mu\nu}+\cdots.
\end{align}
Note que, ya que el término de orden cero es la métrica plana (y además usamos coordenadas donde esta métrica plana es constante), las expansiones de la conexión, curvatura, y el tensor de Einstein comienzan con el orden 1.

Similarmente, es necesario en general considerar una expansión del tensor de energía-momentum de la materia:
\begin{equation}
T_{\mu\nu}=T^{(0)}_{\mu\nu}+T^{(1)}_{\mu\nu}+T^{(2)}_{\mu\nu}+\cdots.
\end{equation}
Aquí $T_{\mu\nu}^{(0)}$ representa el tensor de energía-momentum de la materia a orden cero en $G$, \textbf{como si la materia estuviese distribuida en un espacio plano}. Recuerde que en general $T_{\mu\nu}$ depende de la métrica\footnote{Por ejemplo, para un fluido perfecto $T^{\mu\nu}=(\rho+p/c^2) u^\mu u^\nu-p\,g^{\mu\nu}$.}, por lo que es necesario realizar la expansión correspondiente para $T_{\mu\nu}$.

Con esto, las ecuaciones de Einstein adoptan la forma
\begin{equation}
G^{(1)}_{\mu\nu}+G^{(2)}_{\mu\nu}+G^{(3)}_{\mu\nu}+\cdots=
\frac{8\pi G}{c^4}\left(T^{(0)}_{\mu\nu}+T^{(1)}_{\mu\nu}+T^{(2)}_{\mu\nu}+\cdots\right),
\end{equation}
que \textit{separaremos}, consistentemente, en
\begin{align}
G^{(1)}_{\mu\nu}&=\frac{8\pi G}{c^4}T^{(0)}_{\mu\nu},\label{orden1}\\
G^{(2)}_{\mu\nu}&=\frac{8\pi G}{c^4}T^{(1)}_{\mu\nu},\\
G^{(3)}_{\mu\nu}&=\frac{8\pi G}{c^4}T^{(2)}_{\mu\nu}\label{orden3},
\end{align}
etc. Note que $G^{(1)}_{\mu\nu}$ naturalmente depende de las componentes de la métrica hasta orden 1. Específicamente, depende (no linealmente) de $\eta$ y \textit{linealmente} de $h_{\mu\nu}^{(1)}$. Análogamente, la contribución $G^{(n)}_{\mu\nu}$ dependerá no linealmente de $h^{(m)}_{\mu\nu}$ con $m<n$, y linealmente de $h^{(n)}_{\mu\nu}$. Por lo tanto, podemos considerar a la ecuación de orden $n$, 
\begin{equation}
G^{(n)}_{\mu\nu} = \frac{8\pi G}{c^4}T^{(n-1)}_{\mu\nu},
\end{equation}
como una ecuación lineal para $h^{(n)}_{\mu\nu}$, determinada por las contribuciones de orden inferior $h^{(m)}_{\mu\nu}$ con $m<n$ y la contribución $T^{(n-1)}_{\mu\nu}$ del tensor de energía-momentum.

\subsection{Expansión a primer orden}
\subsubsection{Métrica}

Primero requerimos calcular la métrica inversa $g^{\mu\nu}$
\begin{equation}
g^{\mu\nu}=\eta^{\mu\nu}+g^{\mu\nu}_{(1)}+g^{\mu\nu}_{(2)}+\mathcal{O}(G^3).
\end{equation}
Un cálculo simple muestra que
\begin{align}
g^{\mu\nu}_{(1)}&=-\eta^{\mu\lambda}\eta^{\nu\rho}h^{(1)}_{\lambda\rho}=:-h^{\mu\nu}_{(1)} .
\end{align}

\textit{Por convención, en el contexto de la expansión realizada subimos y bajamos los índices usando la métrica plana} $\eta$. Así, por ejemplo, $h^{(1)}:=h^\mu_{(1)}{}_\mu=\eta^{\mu\nu}h^{(1)}_{\mu\nu}$ es la \textit{traza} del tensor $h^{(1)}_{\mu\nu}$ y $\square :=\partial_\mu\partial^\mu=\eta^{\mu\nu}\partial_\mu\partial_\nu$ es el operador de onda.
\subsubsection{Conexión}\label{sec:conG1}

La primera contribución a los símbolos de Christoffel resultan ser
\begin{align}
\Gamma^\lambda_{(1)\mu\nu}&=\frac{1}2\eta^{\lambda\rho}\left(\partial_\mu
h^{(1)}_{\nu\rho} +\partial_\nu h^{(1)}_{\mu\rho} - \partial_\rho h^{(1)}_{\mu\nu}\right)\\
&=\frac{1}{2}\left(\partial_\mu h^\lambda_{(1)\nu} + \partial_\nu h^\lambda_{(1)\mu} -\partial^\lambda h^{(1)}_{\mu\nu}\right) . \label{Gamma1}
\end{align}

\subsubsection{Tensor de curvatura de Riemann}
Similarmente, el término de primer orden del tensor de curvatura es:
\begin{align}
R^\rho_{(1)\mu\nu\lambda}&= \partial_\nu
\Gamma^\rho_{(1)\mu\lambda}-\partial_\lambda \Gamma^\rho_{(1)\mu\nu} \\
&= \frac{1}{2}\left(\partial_\mu\partial_\nu h^\rho_{(1)\lambda} -
\partial_\mu\partial_\lambda h^\rho_{(1)\nu} + \partial_\lambda \partial^\rho
h^{(1)}_{\mu\nu} - \partial_\nu\partial^\rho h^{(1)}_{\mu\lambda}\right)  \, .
\end{align}

\subsubsection{Tensor de Ricci}

Como consecuencia, el tensor de Ricci es de la forma:
\begin{align}
R^{(1)}_{\mu\lambda}&=R^\rho_{(1)\mu\rho\lambda} \\
&=\frac{1}{2}\left(\partial_\mu\partial_\nu h^\nu_{(1)\lambda}
 + \partial_\lambda \partial^\nu h^{(1)}_{\mu\nu}- \partial_\mu\partial_\lambda h^{(1)} -\square h^{(1)}_{\mu\lambda}\right).
\end{align}




\subsubsection{Escalar de Curvatura}

El escalar de curvatura es dado, a primer orden por,
\begin{equation}\label{R1}
R^{(1)}=\eta^{\mu\nu}R^{(1)}_{\mu\nu}=\partial^\mu\partial^\nu h^{(1)}_{\mu\nu} - \square h^{(1)} .
\end{equation}


\subsubsection{Tensor de Einstein}

Finalmente, el tensor de Einstein es
\begin{align}
G^{(1)}_{\mu\nu} &= R^{(1)}_{\mu\nu}-\frac{1}{2}\eta_{\mu\nu}R^{(1)}\\
&= \frac{1}{2}\left[\partial_\mu\partial^\lambda h^{(1)}_{\lambda\nu} +\partial_\nu\partial^\lambda h^{(1)}_{\lambda\mu} - \partial_\mu\partial_\nu h^{(1)} -\square h^{(1)}_{\mu\nu} -\eta_{\mu\nu}\left(\partial^\lambda \partial^\rho h^{(1)}_{\lambda\rho} - \square h^{(1)}\right) \right].
\end{align}
Es conveniente definir el tensor $\bar{t}_{\mu\nu}$, asociado a un
tensor simétrico $t_{\mu\nu}$, como
\begin{eqnarray}
\bar{t}_{\mu\nu}&:=&t_{\mu\nu}-\frac{1}2\eta_{\mu\nu}\, t \label{tbarra}\\
&=&t_{\mu\nu}-\frac{1}2\eta_{\mu\nu}\, \eta^{\lambda\rho}\, t_{\lambda\rho} \, .
\end{eqnarray}
Puede verificarse que 
\begin{equation}
\bar{t}:= \eta^{\mu\nu}\,\bar{t}_{\mu\nu}=-t, \qquad 
\bar{\bar{t}}_{\mu\nu}=t_{\mu\nu}.
\end{equation}
Entonces podemos escribir:
\begin{equation}
G^{(1)}_{\mu\nu} =-\frac{1}{2}\left[\square \bar{h}^{(1)}_{\mu\nu}
+\eta_{\mu\nu}\, \partial^\lambda\partial^\rho\bar{h}^{(1)}_{\lambda\rho}-
\partial_\mu\partial^\lambda\bar{h}^{(1)}_{\lambda\nu}
- \partial_\nu\partial^\lambda \bar{h}^{(1)}_{\lambda\mu}\right].
\end{equation}

\section{Ecuaciones de Einstein linealizadas}
A primer orden, la ecuación (\ref{orden1}) para $h^{(1)}_{\mu\nu}$ es
\begin{equation}\label{lee}
\boxed{\square \bar{h}^{(1)}_{\mu\nu}
+\eta_{\mu\nu}\, \partial^\lambda\partial^\rho\bar{h}^{(1)}_{\lambda\rho}-
\partial_\mu\partial^\lambda\bar{h}^{(1)}_{\lambda\nu}
- \partial_\nu\partial^\lambda \bar{h}^{(1)}_{\lambda\mu} =
-\frac{16\pi G}{c^4}T_{\mu\nu}^{(0)} .}
\end{equation}

Note además que operando con $\partial^\mu$ a ambos lados de  \eqref{lee}, obtenemos que la ``cuadridivergencia'' del lado izquierdo se anula idénticamente. Como consecuencia, el tensor de energía-momentum en el lado derecho de (\ref{lee}) debe satisfacer
\begin{equation}
\partial^\mu T_{\mu\nu}^{(0)}=0, \label{cT0}
\end{equation}
es decir, que la energía y el momentum de la materia descrito por $T_{\mu\nu}^{(0)}$ \textit{se conserva}. Esto es consistente con la interpretación que $T_{\mu\nu}^{(0)}$ describe el contenido de energía-momentum de la materia \textit{en ausencia de gravitación}.


\section{Transformaciones de gauge}
En la sección anterior derivamos las ecuaciones de Einstein linealizadas asumiendo como punto de partida la descomposición (\ref{exp01}) de la métrica, en un sistema de coordenadas cuasi-inercial. Sin embargo, este sistema de coordenadas, y por lo tanto la descomposición (\ref{exp01}), \textit{no es único}. De hecho, \textit{existen infinitos sistemas de coordenadas en los que puede descomponerse la métrica como en} (\ref{exp01}), pero en general con perturbaciones $h_{\mu\nu}$ diferentes.

En efecto, como consecuencia del hecho que estamos usando un espaciotiempo ``de fondo'' plano, el formalismo es naturalmente covariante bajo
transformaciones de Lorentz \textit{globales} de las coordenadas, es
decir, bajo la transformación de Lorentz $x^\mu\rightarrow x'^\mu=\Lambda^\mu_{\ \nu}\, x^\nu$ las componentes de la métrica son transformadas de forma tal que en el nuevo SC una descomposición de la forma (\ref{exp01}) es también válida, con $h^{\mu\nu}\to h'^{\mu\nu}=\Lambda^\mu_{\ \lambda}\Lambda^\nu_{\ \rho}h^{\lambda\rho}$.

 Adicionalmente, transformaciones de la forma
\begin{equation}\label{cc}
x^\mu(P) \rightarrow x'^\mu(P)=x^\mu(P) + \xi^\mu(x(P))
\end{equation}
(de las coordenadas usadas para etiquetar el evento $P$) conducen a nuevas descomposiciones del tipo (\ref{exp01}). Verificamos esto calculando las componentes del tensor métrico en las nuevas coordenadas:
\begin{equation}
g'_{\mu\nu}(x' (P)) = \frac{\partial x^\lambda}{\partial
x'^\mu}(P) \frac{\partial x^\rho}{\partial x'^\nu}(P)\, g_{\lambda\rho}(x(P)).
\end{equation}
Tal como lo hicimos antes, podemos considerar que $\xi^\mu$ tiene una dependencia general con $G$, de modo que\footnote{Un término del tipo $\xi^\mu_{(0)}$ no es considerado puesto que transformaría métricas donde la separación (\ref{exp01}) es válida en métricas que no cumplen con esta descomposición. }
\begin{equation}
\xi^\mu=\xi^\mu_{(1)}+\xi^\mu_{(2)}+\xi^\mu_{(3)}+\cdots.
\end{equation}
A primer orden, obtenemos
\begin{eqnarray}
g'_{\mu\nu}(x' (P))&=&\frac{\partial x^\lambda}{\partial
x'^\mu}(P)
\frac{\partial x^\rho}{\partial x'^\nu}(P)\, g_{\lambda\rho}(x(P)) \\
&=&\left(\delta^\lambda_\mu - \partial_\mu\xi^\lambda_{(1)}(P)
\right)\left(\delta^\rho_\nu -
\partial_\nu\xi^\rho_{(1)}(P) \right) \left(\eta_{\lambda\rho} + h^{(1)}_{\lambda\rho}\right) +
\mathcal{O}(G^2) \\
&=& \eta_{\mu\nu} + h^{(1)}_{\mu\nu}(P)- \partial_\mu\xi_\nu^{(1)}(P) - \partial_\nu
\xi_\mu^{(1)}(P)
+ \mathcal{O}(G^2) \\
&=:& \eta_{\mu\nu} + h'^{(1)}_{\mu\nu}(P)+ \mathcal{O}(G^2) .
\end{eqnarray}
Por lo tanto, en las coordenadas $x'$ las perturbaciones métricas de primer orden $h'^{(1)}_{\mu\nu}$ están dadas por
\begin{equation}\label{pc}\marginnote{Transf. de gauge}
\boxed{h'^{(1)}_{\mu\nu}=h^{(1)}_{\mu\nu}- \partial_\mu\xi_\nu^{(1)} - \partial_\nu\xi_\mu^{(1)} , \qquad \xi_\mu^{(1)}:=\eta_{\mu\nu}\, \xi^\nu_{(1)} .}
\end{equation}

Resumiendo, el cambio de coordenadas (\ref{cc}) transforma una métrica de la forma (\ref{exp01}) en una métrica de la misma forma, pero con una perturbación $h'^{(1)}_{\mu\nu}$ relacionada con la original por medio de (\ref{pc}).


\subsection{Invariancia de gauge}
La propiedad fundamental de las transformaciones de gauge (\ref{cc}) y
(\ref{pc}) es que ellas dejan, a primer orden, el tensor de curvatura, y por consiguiente las ecuaciones linealizadas de Einstein, \textit{invariantes}\footnote{\dots y no sólo covariantes. Ellas \textit{siempre} son covariantes, bajo cualquier transformación coordenada.}. Esto es verificado fácilmente calculando el correspondiente cambio del tensor de Riemann (totalmente covariante):
Verificamos que bajo (\ref{pc}) se cumple que
\begin{equation}
R'^{(1)}_{\mu\nu\lambda\rho}=R^{(1)}_{\mu\nu\lambda\rho}
\end{equation}
y, por lo tanto, $R'^{(1)}_{\mu\nu}=R^{(1)}_{\mu\nu}$, $R'^{(1)}=R^{(1)}$ y
$G'^{(1)}_{\mu\nu}=G^{(1)}_{\mu\nu}$.

\subsection{Gauge de Lorenz}\label{sec:GL}
Ya que el lado izquierdo de las ecuaciones de Einstein es invariante bajo la
transformación (\ref{cc}) en el orden lineal, podemos usar esta libertad de
gauge para seleccionar sistemas de coordenadas en los que las
perturbaciones $\bar{h}^{(1)}_{\mu\nu}$ sean particularmente simples.

Impondremos el \textit{gauge de Lorenz}\footnote{Denominado así por su analogía con el gauge de Lorenz usado en electrodinámica.}, definido por
\begin{equation} \label{lgauge}\marginnote{Gauge de Lorenz}
\boxed{\partial^\nu\bar{h}^{(1)}_{\mu\nu}\stackrel{!}{=} 0 .}
\end{equation}
Este gauge siempre puede ser impuesto. Suponga que se comienza con un campo
$h^{(1)}_{\mu\nu}$ que no satisface (\ref{lgauge}). Entonces podemos realizar una transformación de gauge (\ref{pc}) tal que
\begin{eqnarray}
\bar{h}'^{(1)}_{\mu\nu}&=&h'^{(1)}_{\mu\nu}-\frac{1}2\eta_{\mu\nu}
h'^{(1)} \\
&=& h^{(1)}_{\mu\nu}- \partial_\mu\xi^{(1)}_\nu - \partial_\nu\xi^{(1)}_\mu -\frac{1}{2}\eta_{\mu\nu}
\left(h^{(1)} - 2\partial_\lambda \xi_{(1)}^\lambda \right)  \\
&=& \bar{h}^{(1)}_{\mu\nu}- \partial_\mu\xi^{(1)}_\nu - \partial_\nu\xi^{(1)}_\mu
 +\eta_{\mu\nu}\partial_\lambda \xi_{(1)}^\lambda \, . \label{hp3}
\end{eqnarray}
De este modo, podemos imponer $\partial'^\nu
\bar{h}'^{(1)}_{\mu\nu} = \partial^\nu
\bar{h}'^{(1)}_{\mu\nu}= \partial^\nu\bar{h}^{(1)}_{\mu\nu} -
\square \xi^{(1)}_\mu \stackrel{!}{=} 0 $, que requiere elegir un campo $\xi^{(1)}_\mu$ tal que
\begin{equation}\label{gcond}
\square \xi^{(1)}_\mu= \partial^\nu\bar{h}^{(1)}_{\mu\nu} \, .
\end{equation}
Esta condición siempre puede ser satisfecha, ya que la ecuación de onda
siempre tiene soluciones, dadas las condiciones de borde adecuadas.

Suponga ahora que ya disponemos de un campo $h^{(1)}_{\mu\nu}$ que
satisface el gauge de Lorenz. Entonces existe aún una \textit{libertad residual} (tal
como en electrodinámca), definida por aquellas transformaciones de gauge
generadas por un vector $\xi^{(1)}_\mu$ que sea \textit{armónico}, es decir, que satisfaga
\begin{equation}\label{cxi0}
\square \xi^{(1)}_\mu=0,
\end{equation}
ver (\ref{gcond}).

En el gauge de Lorenz, las ecuaciones de Einstein linealizadas asumen la forma
de una \textit{ecuación de onda inhomogénea}. Usando (\ref{lee}) y (\ref{lgauge}),
encontramos
\begin{equation}\label{leelg}\marginnote{Ecs. lineales gauge Lorenz}
\boxed{\square \bar{h}^{(1)}_{\mu\nu} = -\frac{16\pi G}{c^4} T_{\mu\nu}^{(0)} ,
\qquad \partial^\nu\bar{h}^{(1)}_{\mu\nu}=0 \, .}
\end{equation}
Es decir, en el gauge de Lorenz la perturbación de primer orden $\bar{h}_{\mu\nu}$ satisface la ecuación de onda inhomogénea. En una región sin materia $\bar{h}_{\mu\nu}$ satisface la ecuación de onda (homogénea), lo que implica que pueden existir \textit{soluciones propagantes}, \textbf{cuya velocidad de propagación es la velocidad de la luz}.

Como puede verse, la situación es análoga al caso de las ecuaciones de Maxwell en Electrodinámica Clásica donde las ecuaciones inhomogéneas de Maxwell adoptan, en términos del 4-potencial electromagnético $A_{\mu}^{\rm em}=(\phi_{\rm em},-\vec{A}_{\rm em})$, la forma $\Box A^\mu_{\rm em}-\partial^\mu(\partial_\nu A^\nu_{\rm em})=4\pi J^\mu /c$ (donde $J_{\mu}:=(c\rho,-\vec{j})$ es la $4$-densidad de corriente), pero pueden ser reducidas a ecuaciones de onda inhomogéneas,  $\Box A_{\mu}^{\rm em}=4\pi J_{\mu}/c$ si se usan potenciales que satisfagan el gauge de Lorenz, $\partial^{\mu}A_{\mu}^{\rm em}=0$.

Recuerde que, al orden de aproximación considerado, es suficiente calcular el
tensor de energía-momentum a orden cero, es decir, en ausencia de campo gravitacional y usando la métrica plana para subir y bajar los índices.

Las soluciones particulares correspondientes a \textit{campos retardados asintóticamente nulos} son entonces de la forma
\begin{equation}
\bar{h}^{(1)}_{\mu\nu}(\vec{x},t)=-\frac{4G}{c^4} \int\frac{T_{\mu\nu}^{(0)}
(\vec{x}',t-{\left|\vec{x}-\vec{x}'\right|}/{c})}{\left|\vec{x}-\vec{x}
'\right|}\, d^3x' ,\label{solh0}
\end{equation}
o simplemente,
\begin{equation}
\boxed{\bar{h}^{(1)}_{\mu\nu}(\vec{x},t)=-\frac{4G}{c^4} \int\frac{T_{\mu\nu}^{(0)}(\vec{x}',t_{\rm ret})}{\left|\vec{x}-\vec{x}'\right|}\, d^3x'.} \label{solh}
\end{equation}
La métrica, incluyendo contribuciones hasta primer orden, puede ser obtenida entonces como
\begin{equation}
\boxed{g_{\mu\nu}=\eta_{\mu\nu} + \bar{h}^{(1)}_{\mu\nu}-\frac{1}2\eta_{\mu\nu}\,
\bar{h}^{(1)} \, .} \label{solg}
\end{equation}

\subsubsection{Gauges adicionales en el vacío}

En regiones libres de fuentes, es decir, donde $T^{(0)}_{\mu\nu}=0$, es posible elegir coordenadas tales que, adicionalmente a la condición de Lorenz (\ref{lgauge}), se satisfaga
\begin{equation}\label{ttg}
\boxed{h^{(1)}=0, \qquad h_{0i}^{(1)}=0.}
\end{equation}
En efecto, de (\ref{pc}) se sigue que la transformación de la traza $h^{(1)}$ es de la forma siguiente:
\begin{equation}
h'_{(1)}= h_{(1)}- 2\partial_\mu\xi_{(1)}^\mu .
\end{equation}
Por lo tanto, $h'_{(1)}=0$ si
\begin{equation}
2\partial_\mu\xi_{(1)}^\mu = h_{(1)}. \label{egv1}
\end{equation}
Similarmente, (\ref{pc}) implica que
\begin{equation}
h'^{(1)}_{0i}= h^{(1)}_{0i} -\partial_0\xi^{(1)}_i-\partial_i\xi^{(1)}_0,
\end{equation}
de modo que $h'^{(1)}_{0i}=0$ si
\begin{equation}
\partial_0\xi^{(1)}_i+\partial_i\xi^{(1)}_0= h^{(1)}_{0i} . \label{egv2}
\end{equation}
Las ecuaciones (\ref{egv1}) y (\ref{egv2}) forman entonces un conjunto de cuatro ecuaciones diferenciales para los cuatro campos $\xi_{(1)}^\mu$. Podemos verificar, aplicando el operador de onda sobre (\ref{egv1}) y (\ref{egv2}), que la condición que estas transformaciones adicionales preserven el gauge de Lorenz, es decir, que satisfagan (\ref{cxi0}), implica necesariamente que $\square h^{(1)}=0$ y $\square h^{(1)}_{0i}=0$. Estas condición necesarias son satisfechas, de acuerdo a la ecuación de campo (\ref{leelg}), en regiones libres de fuentes. Para una discusión que muestra que estas condiciones son \textit{suficientes} para asegurar la solución de (\ref{cxi0}), (\ref{egv1}) y (\ref{egv2}), vea \cite{Wald84}, sección 4.4b.

El ``gauge" (es decir, la elección de coordenadas cuasi-inerciales) en el que el campo $h^{(1)}_{\mu\nu}$ satisface tanto el gauge de Lorenz (\ref{lgauge}) como las condiciones (\ref{ttg}) es llamado \textit{gauge transversal sin traza} (o TT-gauge, por ``Transverse Traceless"). En la sección \ref{sec:OGP} consideraremos este gauge en la descripción de ondas gravitacionales planas.

En el gauge, TT no es necesario hacer la distinción entre $h^{(1)}_{\mu\nu}$ y $\bar{h}^{(1)}_{\mu\nu}$, ya que $h^{(1)}=\bar{h}^{(1)}=0$ y por lo tanto
\begin{equation}
h^{(1)}_{\mu\nu}=\bar{h}^{(1)}_{\mu\nu}.
\end{equation}
Además, como $h^{(1)}=h^{(1)\mu}_{\ \mu}=h^{(1)0}_{\ 0}+h^{(1)i}_{\ i}=0$ podemos escribir la componente $h^{(1)0}_{\ 0}$ en función de las componentes puramente espaciales: $h^{(1)}_{00}=-h^{(1)i}_{\ i}=h^{(1)}_{ii}$. Esto permite escribir cualquier expresión que involucre $h^{(1)}_{\mu\nu}$ como una función sólo de las componentes puramente espaciales $h^{(1)}_{ij}$. Finalmente, note además que en el gauge TT la condición de gauge de Lorenz (\ref{lgauge}) se reduce a
\begin{equation}
\partial_0h^{(1)}_{00}=0, \qquad \partial_ih_{(1)}^{ij}=0. \label{dh00}
\end{equation}

\section{Ondas gravitacionales planas: dos polarizaciones}\label{sec:OGP}

Consideramos una región sin materia por la que se propaga una onda gravitacional plana de la forma
\begin{equation}\label{op}
 \bar{h}_{\mu\nu}=\Re\left[A_{\mu\nu}\exp{(ik_\lambda x^\lambda)}\right],
\end{equation}
donde $A_{\mu\nu}$ es el \textit{tensor} (bajo TL) \textit{amplitud de la onda} y $k_\lambda$ es el \textit{4-vector} (bajo TL) \textit{de onda}. Introduciendo (\ref{op}) en la ecuación de onda homogénea, $\square\bar{h}_{\mu\nu}=0$, encontramos que $k_\lambda k^\lambda=0$, mientras que la condición de gauge de Lorenz implica que
\begin{equation}
 A_{\mu\nu}k^\nu=0. \label{glA}
\end{equation}
Estas últimas cuatro condiciones reducen el número de componentes linealmente independientes de la amplitud $A_{\mu\nu}$ de 10 a 6. Adicionalmente, podemos usar la libertad de gauge residual para reducir el número de componentes independientes a sólo 2.

En efecto, dado un vector tipo luz $k_\mu$ arbitrario, podemos elegir los ejes coordenados (por medio de una TL) de modo que la onda se propague a lo largo del eje $z$ positivo, es decir, tal que
\begin{equation}\label{kmu}
 k^\mu=(k,0,0,k), \qquad k_\mu=(k,0,0,-k),
\end{equation}
con $k>0$. Con esto, la condición (\ref{glA}) implica que $A_{\mu 3}=-A_{\mu 0}$, de modo que
\begin{equation}\label{AA}
 A^{\mu\nu}=\left(\begin{array}{cccc}
A^{00} & A^{01} & A^{02} & A^{00} \\
A^{01} & A^{11} & A^{12} & A^{01} \\
A^{02} & A^{12} & A^{22} & A^{02} \\
A^{00} & A^{01} & A^{02} & A^{00} \end{array}
\right)=\left(\begin{array}{cccc}
A^{33} & A^{13} & A^{23} & A^{33} \\
A^{13} & A^{11} & A^{12} & A^{13} \\
A^{23} & A^{12} & A^{22} & A^{23} \\
A^{33} & A^{13} & A^{23} & A^{33} \end{array}
\right).
\end{equation}
Consideramos ahora la transformación de gauge definida por
\begin{equation}\label{xiOP}
 \xi^\mu=-\Re\left[i\epsilon^\mu\exp{(ik_\lambda x^\lambda)}\right],
\end{equation}
donde $\epsilon^\mu$ son constantes, que ajustaremos a continuación. Esta elección de $\xi^\mu$ satisface $\square\xi^\mu=0$, de modo que la transformación preserva la condición de Lorenz. Además, (\ref{pc}) implica que la nueva perturbación $\bar{h}'_{\mu\nu}$ tiene también la forma (\ref{op}) de una onda plana monocromática, pero con la nueva amplitud
\begin{equation}\label{AAp}
 A'_{\mu\nu}=A_{\mu\nu}-\epsilon_\mu k_\nu-\epsilon_\nu k_\mu+\eta_{\mu\nu}(\epsilon^\lambda k_\lambda).
\end{equation}
Usando (\ref{kmu}) y (\ref{AA}) podemos escribir la transformación (\ref{AAp}) explícitamente como
\begin{eqnarray}
A'^{00}&=&A^{00}-k(\epsilon^0+\epsilon^3),\\
A'^{01}&=&A^{01}-k\epsilon^1,\\
A'^{02}&=&A^{02}-k\epsilon^2,\\
A'^{11}&=&A^{11}-k(\epsilon^0-\epsilon^3),\\
A'^{12}&=&A^{12},\\
A'^{22}&=&A^{22}-k(\epsilon^0-\epsilon^3).
\end{eqnarray}
Podemos entonces elegir las constantes $\epsilon^\mu$ de modo que la amplitud $A'^{\mu\nu}$ adopte una forma simple. En particular, resulta conveniente imponer que $A'^{00}\stackrel{!}{=}A'^{01}\stackrel{!}{=}A'^{02}\stackrel{!}{=}0$ y $A'^{11}\stackrel{!}{=}-A'^{22}$. Esta elección requiere elegir
\begin{eqnarray}
\epsilon^0&=&\frac{1}{4k}\left(2A^{00}+A^{11}+A^{22}\right),\\
\epsilon^1&=&\frac{1}{k}A^{01},\\
\epsilon^2&=&\frac{1}{k}A^{02},\\
\epsilon^3&=&\frac{1}{4k}\left(2A^{00}-A^{11}-A^{22}\right),
\end{eqnarray}
y con esto obtenemos finalmente que
\begin{equation}\label{AApf}
 A'^{\mu\nu}=\left(\begin{array}{cccc}
0 & 0 & 0 & 0 \\
0 & A'^{11} & A'^{12} & 0 \\
0 & A'^{12} & -A'^{11} & 0 \\
0 & 0 & 0 & 0 \end{array}
\right)=A'^{11}\left(\begin{array}{cccc}
0 & 0 & 0 & 0 \\
0 & 1 & 0 & 0 \\
0 & 0 & -1 & 0 \\
0 & 0 & 0 & 0 \end{array}
\right)+A'^{12}\left(\begin{array}{cccc}
0 & 0 & 0 & 0 \\
0 & 0 & 1 & 0 \\
0 & 1 & 0 & 0 \\
0 & 0 & 0 & 0 \end{array}
\right).
\end{equation}
Debido a que esta elección de gauge satisface $A'^\mu{}_\mu=0$, entonces $\bar{h}'=h'=0$, es decir, la perturbación es de traza nula. 
%Como consecuencia,  $\bar{h}'_{\mu\nu}=h'_{\mu\nu}$. 
Además, como $h'_{\mu 0}=0$ y $h'_{\mu 3}=0$ 
%se dice que este gauge es \textit{transversal}. Por estas razones, el gauge aquí 
%presentado se conoce como el gauge \textit{transversal sin traza} o ``gauge TT'' 
%(de ``Transverse-Traceless-gauge''). 
verificamos que \textbf{$A'_{\mu\nu}$ describe una solución en el gauge transversal sin traza} (TT). En este sentido, una onda gravitacional plana es \textit{transversal} y tiene sólo \textit{2 estados de polarización independientes}.

Si denotamos las amplitudes (complejas) $A'^{11}$ y $A'^{12}$ en términos de constantes reales, de modo que
\begin{equation}
A'^{11}=:h_+e^{-i\varphi_+}, \qquad A'^{12}=:h_\times e^{-i\varphi_\times},
\end{equation}
entonces \textit{el elemento de línea correspondiente a una onda gravitacional propagándose a lo largo del eje $z$ positivo, con vector de onda $k$ y frecuencia $\omega=ck$, con amplitudes $h_+$ y $h_\times$, y fases $\varphi_+$ y $\varphi_\times$ respectivamente}, es dado por
\begin{equation}\marginnote{El. línea onda grav. plana, gauge TT}
 \boxed{ds^2=c^2dt^2-d\vec{x}^2+h_+\cos(\omega t-kz-\varphi_+)(dx^2-dy^2)+2h_\times\cos(\omega t-kz-\varphi_\times)dxdy .}
\end{equation}

Consideremos por separado las dos posibles polarizaciones linealmente independientes de la onda gravitacional plana. Para el primer estado de polarización posible, con $h_+\neq 0$ y $h_\times=0$, tendremos que el elemento de línea se reduce a
\begin{equation}\label{pol1}
 ds^2=c^2dt^2-d\vec{x}^2+h_+\cos(\omega t-kz-\varphi_+)(dx^2-dy^2).
\end{equation}
En cambio, para el segundo estado de polarización independiente, con $h_+=0$ y $h_\times\neq 0$,
\begin{equation}\label{pol2}
 ds^2=c^2dt^2-d\vec{x}^2+2h_\times\cos(\omega t-kz-\varphi_\times)dxdy.
\end{equation}
Ahora, tomemos el elemento de línea (\ref{pol1}) correspondiente al primer estado de polarización y efectuemos una \textit{rotación de los ejes coordenadas en un ángulo de 45 grados en torno al eje de propagación de la onda}. Es decir, introduzcamos el nuevo sistema coordenado $x'^\mu=(ct,x',y',z)$, con
\begin{equation}
 x':=\frac{1}{\sqrt{2}}(x+y), \qquad y':=-\frac{1}{\sqrt{2}}(x-y).
\end{equation}
Es directo verificar que con este cambio de coordenadas el elemento de línea (\ref{pol1}) adopta la forma siguiente:
\begin{equation}
 ds^2=c^2dt^2-d\vec{x}'^2-2h_+\cos(\omega t-kz-\varphi_+)dx'dy'.
\end{equation}
Concluimos de este modo que \textit{los elementos de línea de las ondas gravitacionales planas correspondientes a los dos estados de polarización físicos, y por consiguiente sus efectos, difieren (obviando su amplitud y fases eventualmente diferentes) sólo en una rotación de $\pi/4$ en el plano normal a la propagación de la onda}. Debido a esta propiedad, podemos en lo sucesivo restringirnos al estudio de las propiedades de un estado de polarización de una onda gravitacional plana, puesto que los efectos de la otra polarización independiente son similares, luego de efectuar una rotación de $\pi/4$ al sistema.

\section{Efectos de una onda gravitacional: principio de un detector de ondas gravitacionales}
Consideremos una región del espaciotiempo por donde viaja una onda gravitacional plana, con vector de onda $k^\mu=(k,0,0,k)$ ($k>0$) dado. Usando (\ref{Gamma1}) y (\ref{AApf}) encontramos que en el gauge TT, a primer orden, $\Gamma^\mu_{\ 00}=0$.
% \begin{eqnarray}
% \Gamma^\mu_{\ 00}&=&0,\\
% \Gamma^0_{\ 0i}&=&0,\\
% \Gamma^i_{\ 0j}&=&\frac{1}{2}\partial_0h^i_{\ j}=-\frac{1}{2}\Re\left[ik_0A^{ij}\exp{(ik_\lambda x^\lambda)}\right],\\
% \end{eqnarray}
Debido a esto, tenemos que las curvas tipo tiempo dadas por $x^\mu(\tau)=(c\tau,\vec{x}_0)$ con $\vec{x}_0$ constante son geodésicas, ya que entonces
\begin{equation}
 \frac{d^2x^\mu}{d\tau^2}+\Gamma^\mu_{\ \nu\lambda}\frac{dx^\nu}{d\tau} \frac{dx^\lambda}{d\tau}=0+\Gamma^\mu_{\ 00}c^2=0.
\end{equation}
Lo anterior implica que una partícula de prueba ubicada en $\vec{x}=\vec{x}_0$ e inicialmente en reposo respecto al SC cuasi-inercial,  es decir, con  $(d\vec{x}/dt)(0)=\vec{0}$, \textit{permanecerá con su coordenada espacial constante} en el futuro. Esto no implica, sin embargo, que la ``distancia'' (o, más precisamente, el tiempo de vuelo) entre distintos cuerpos de prueba inicialmente en reposo en el SC usado sea constante, ya que la métrica del espaciotiempo varía tanto en el espacio como en el tiempo.

Consideremos un conjunto de partículas de prueba ubicadas inicialmente en un círculo en el plano $xy$, de radio $R$ en coordenadas cuasi-inerciales, y el movimiento de ida y regreso de fotones desde el centro, $\vec{x}=\vec{0}$, hasta un punto sobre la circunsferencia, ubicado en un ángulo $\varphi$ respecto al eje $x$.

Parametrizamos la curva desde el origen hasta el punto de retorno sobre la circunsferencia por $\vec{x}(t)=(r(t)\cos\varphi,r(t)\sin\varphi,0)$. El movimiento de un fotón satisface $ds^2=0$ de modo que, en el caso de una onda polarizada tal que el elemento de línea sea (\ref{pol1}), podemos escribir
 \begin{eqnarray}
 cdt&=&\sqrt{dx^2+dy^2-h_+\cos(\omega t-kz-\varphi_+)(dx^2-dy^2)}\\
 &=&\sqrt{dr^2-h_+\cos(\omega t-kz-\varphi_+)(\cos^2\varphi-\sin^2\varphi)dr^2}\\
 &=&dr\sqrt{1-h_+\cos(\omega t-kz-\varphi_+)\cos(2\varphi)}\\
 &=&dr\left[1-\frac{1}{2}h_+\cos(\omega t-kz-\varphi_+)\cos(2\varphi)\right]+\mathcal{O}(G^2). \label{dr}
 \end{eqnarray}
Luego
\begin{eqnarray}
dr&=&\frac{cdt}{1-\frac{1}{2}h_+\cos(\omega t-kz-\varphi_+)\cos(2\varphi)}+\mathcal{O}(G^2)\\
&=&\left[1+\frac{1}{2}h_+\cos(\omega t-kz-\varphi_+)\cos(2\varphi) \right]cdt+\mathcal{O}(G^2).
\end{eqnarray}
Si $t_0$ es (la coordenada temporal correspondiente a) el instante en que el fotón sale del origen y $t_{\rm i}$ al de llegada del fotón a la posición de la partícula ubicada en la circunsferencia de radio $R$, entonces (a primer orden en $G$):
\begin{eqnarray}
\int_{0}^{R}dr&=&\int_{t_0}^{t_{\rm i}}\left[1+\frac{1}{2}h_+\cos(\omega t-kz-\varphi_+)\cos(2\varphi) \right]cdt\\
R&=&c\left[t_{\rm i}-t_{0}+\frac{h_+}{2\omega}\left[\sin(\omega t_{\rm i}-kz-\varphi_+)-\sin(\omega t_0-kz-\varphi_+)\right]\cos(2\varphi) \right].
\end{eqnarray}
A partir de esta expresión, obtenemos
\begin{eqnarray}
t_{\rm i}&=&t_0+\frac{R}{c}-\frac{h_+}{2\omega}\left[\sin(\omega t_{\rm i}-kz-\varphi_+)-\sin(\omega t_0-kz-\varphi_+)\right]\cos(2\varphi) \label{ti1}\\
&=&t_0+\frac{R}{c}+\mathcal{O}(G). \label{ti2}
\end{eqnarray}
Note que \eqref{ti1} define una ecuación trascendental para $t_{\rm i}$. Afortunadamente, debido a que estamos usando un esquema perturbativo, es posible encontrar una solución analítica, a primer orden en $G$. Para esto usamos \eqref{ti2} e ``iteramos'', es decir, reemplazamos esta expresión en el lado derecho de \eqref{ti1}. Entonces, podemos escribir
\begin{eqnarray}
\sin(\omega t_{\rm i}-kz-\varphi_+)&=&\sin(\omega t_0+\frac{R\omega}{c}-\varphi_++\mathcal{O}(G))\\
&= &\sin(\omega t_0+\frac{R\omega}{c}-\varphi_+)+\mathcal{O}(G) .
\end{eqnarray}
Por lo tanto, una expresión para $t_{\rm i}$ (instante de llegada del fotón a la circunsferencia en función el tiempo de partida $t_0$ y $\varphi$), correcta a primer orden en $G$ es dada por
\begin{equation}
t_{\rm i}=t_0+\frac{R}{c}-\frac{h_+}{2\omega}\left[\sin(\omega t_0+\frac{R\omega}{c}-kz-\varphi_+)-\sin(\omega t_0-kz-\varphi_+)\right]\cos(2\varphi)+\mathcal{O}(G^2). \label{ti3}
\end{equation}
Si el tamaño de la circunsferencia es suficientemente pequeño, específicamente si $R\ll 2\pi c/\omega=\lambda$, entonces podemos aproximar \eqref{ti3} realizando una expansión de la siguiente forma:
\begin{equation}
\sin\left(\omega t_0+\frac{R\omega}{c}-kz-\varphi_+\right)\approx \sin(\omega t_0-kz-\varphi_+)+\frac{R\omega}{c}\cos\left(\omega t_0-kz-\varphi_+ \right)+\mathcal{O}(\frac{R^2 \omega^2}{c^2}) .\label{sin}
\end{equation}

Reemplazando (\ref{sin}) en (\ref{ti3}) se obtiene la siguiente relación

\begin{equation}
c(\Delta t)_{\rm i}\approx R\left[1-\frac{h_+}{2}\cos(\omega t_0-kz-\varphi_+)\cos(2\varphi) \right].
\end{equation}

De lo anterior, es posible obtener una expresión para el intervalo $(\Delta t)_{\rm ir}$ de ida y vuelta del fotón desde el centro, dada por

\begin{equation}
c(\Delta t)_{\rm ir}\approx 2R\left[1-\frac{h_+}{2}\cos(\omega t_0-kz-\varphi_+)\cos(2\varphi) \right]. \label{tir}
\end{equation}

Note que es posible obtener este resultado directamente, asumiendo la misma aproximación $(R\ll \lambda)$, a partir de \eqref{dr} ya que en este caso podemos considerar el integrando de \eqref{dr}, $\cos(\omega t-kz-\varphi_+)$ (y por tanto la métrica) como \textit{aproximadamente constante durante el viaje de ida y regreso del fotón}, es decir, $\cos(\omega t-kz-\varphi_+)\approx \cos(\omega t_0-kz-\varphi_+)$. En otras palabras, bajo estas condiciones el tiempo de vuelo del fotón es mucho menor que el tiempo que requiere la onda para cambiar su fase apreciablemente.

Si, de acuerdo a la costumbre, llamamos ``distancia efectiva'' entre el centro y un punto sobre la circunsferencia, correspondiente al ángulo $\varphi$, a la combinación $L(\varphi):=c (\Delta t)_{\rm ir}/2$, entonces
\begin{equation}
\boxed{L(\varphi,t_0)\approx R\left[1-\frac{1}{2}h_+\cos(\omega t_0-kz-\varphi_+)\cos(2\varphi)\right].} \label{elip}
\end{equation}
Esta expresión muestra, como era de esperar, que en ausencia de una onda gravitacional ($h_+=0$) se recobra el resultado usual $L_0(\varphi)=R$, mientras que una onda gravitacional de amplitud $h_+\neq 0$ en forma efectiva cambia las distancias (tiempos de vuelo!) de las partículas de prueba respecto al centro. Más detalladamente, para tiempos $t_0$ en los que $h_+\cos(\omega t_0-kz-\varphi_+)>0$ la expresión (\ref{elip}) implica que las partículas de prueba son ``estiradas'' y ``apretadas'', en forma alternada, a lo largo de los ejes $x$ e $y$ respectivamente, formando elipses.

\begin{center}
\begin{figure}[H]
\centerline{\includegraphics[height=2.1cm]{fig/fig-onda-grav-mas.pdf}}
\centerline{\includegraphics[height=2.1cm]{fig/fig-onda-grav-cruz.pdf}}
\centerline{\includegraphics[height=2.1cm]{fig/fig-onda-grav-circular.pdf}}
\caption{Oscilaciones inducidas por una onda gravitacional con polarización $+$, $\times$ y circular. Figuras adaptadas a partir de originales en \cite{Carroll97}.}
\label{fig:og}
\end{figure}
\end{center}
%\begin{center}
%\begin{figure}[H]
%\centerline{\psfig{file=fig-onda-grav-cruz.pdf,height=2.3cm,angle=0}}
%\caption{Oscilaciones inducidas por una onda gravitacional con polarización $\times$. Figura adaptada a partir de la original contenida en \cite{Carroll97}.}
%\label{fig:ogcruz}
%\end{figure}
%\end{center}
%\begin{center}
%\begin{figure}[H]
%\centerline{\psfig{file=fig-onda-grav-circular.pdf,height=2.3cm,angle=0}}
%\caption{Oscilaciones inducidas por una onda gravitacional con polarización circular. Figura adaptada a partir de la original contenida en \cite{Carroll97}.}
%\label{fig:ogccirc}
%\end{figure}
%\end{center}
Note que la amplitud del cambio de la distancia $L(\varphi)$ es $\delta L=Rh_+/2$, es decir,
\begin{equation}
\boxed{ \frac{\delta L}{L_0}\simeq h .}
\end{equation}
Los detectores de ondas gravitacionales interferométricos actuales alcanzan una sensibilidad que permite detectar amplitudes hasta de $h\sim 10^{-20}$, que es también el orden de magnitud de la amplitud \textit{máxima} de la radiación gravitacional esperada en la Tierra debido a diversas fuentes astronómicas. Por ejemplo, el proyecto \href{http://www.ligo.caltech.edu/}{LIGO}. consta de un interferómetro de 4\,km, de modo que puede detectar fluctuaciones de distancia ($\delta L$) del orden de $10^{-17}\text{m}$, es decir, más pequeñas que un núcleo atómico!.

\begin{center}
\begin{figure}[H]
\centerline{\includegraphics[height=5cm]{fig/fig-LIGO.pdf}}
\caption{Esquema de un detector interferométrico de ondas gravitacionales. Figura original \href{http://commons.wikimedia.org/wiki/File:Ligo.svg}{aquí}.}
\label{fig:LIGO}
\end{figure}
\end{center}
\section{Generación de ondas gravitacionales}\label{sec:GOG}
Tal como en el caso de ondas electromagnéticas, a distancias muy grandes (en la zona lejana o zona de radiación, $r\gg\lambda$) y para fuentes pequeñas (de tamaño $L\ll\lambda$), encontramos que el término dominante de (\ref{solh}) es
\begin{equation}\label{hradT}
\bar{h}_{\rm rad}^{\mu\nu}(\vec{x},t)=-\frac{4G}{c^4} \frac{1}{r}\int T^{\mu\nu}_{(0)}(\vec{x}',t-\frac{r}{c})\, d^3x'.
\end{equation}
Además, en una región limitada del espacio, la onda puede aproximarse por una onda plana. En el gauge de Lorenz, toda la información de la onda está contenida en las componentes puramente espaciales $\bar{h}^{ij}$, ver (\ref{AA}). Además, podemos expresar la integral (retardada) $\int T^{ij}_{(0)}\,d^3x$ en términos de derivadas del momento cuadrupolar de la fuente. En efecto, usando la ley de conservación para el tensor de energía-momentum $T^{\mu\nu}_{(0)}$, podemos escribir
\begin{eqnarray}
 \int \partial_k(T^{ik}_{(0)}x^j)\,d^3x&=&\int (\partial_kT^{ik}_{(0)})x^j\,d^3x +\int T^{ij}_{(0)}\,d^3x \\
&=&-\int (\partial_0T^{i0}_{(0)})x^j\,d^3x +\int T^{ij}_{(0)}\,d^3x \\
&=&-\frac{1}{c}\frac{d\ }{dt}\int T^{i0}_{(0)}x^j\,d^3x +\int T^{ij}_{(0)}\,d^3x .
\end{eqnarray}
Pero la expresión del lado izquierdo puede transformarse en una integral de superficie en la frontera del volumen de integración, que encierra la distribución de energía-momentum que genera el campo gravitacional. Asumiendo que esta distribución está confinada a una región acotada del espacio, tendremos que la integral de superficie se anula y por lo tanto
\begin{equation}
 \int T^{ij}_{(0)}\,d^3x =\frac{1}{c}\frac{d\ }{dt}\int T^{i0}_{(0)}x^j\,d^3x.
\end{equation}
Ya que $T^{ij}$ es simétrico podemos equivalentemente escribir
\begin{equation}\label{intT1}
 \int T^{ij}_{(0)}\,d^3x =\frac{1}{2c}\frac{d\ }{dt}\int \left(T^{i0}_{(0)}x^j+T^{j0}_{(0)}x^i\right)\,d^3x.
\end{equation}
Efectuamos ahora un análisis similar con la expresión $\int\partial_k(T^{0k}_{(0)}x^ix^j)\,d^3x$, que también es nula debido a que puede transformarse a una integral de superficie fuera de la región con fuentes:
\begin{eqnarray}
 0&=&\int\partial_k(T^{0k}_{(0)}x^ix^j)\,d^3x\\
&=&\int(\partial_kT^{0k}_{(0)})x^ix^j\,d^3x+\int \left(T^{0i}_{(0)}x^j+T^{0j}_{(0)}x^i\right)\,d^3x\\
&=&-\int(\partial_0T^{00}_{(0)})x^ix^j\,d^3x+\int \left(T^{0i}_{(0)}x^j+T^{0j}_{(0)}x^i\right)\,d^3x\\
&=&-\frac{1}{c}\frac{d\ }{dt}\int T^{00}_{(0)}x^ix^j\,d^3x+\int \left(T^{0i}_{(0)}x^j+T^{0j}_{(0)}x^i\right)\,d^3x.
\end{eqnarray}
Por lo tanto,
\begin{equation}\label{intT2}
 \int \left(T^{0i}_{(0)}x^j+T^{0j}_{(0)}x^i\right)\,d^3x=\frac{1}{c}\frac{d\ }{dt}\int T^{00}_{(0)}x^ix^j\,d^3x.
\end{equation}
De esta forma, usando (\ref{intT1}) y (\ref{intT2}) encontramos que
\begin{equation}
 \int T^{ij}_{(0)}\,d^3x=\frac{1}{2c^2}\frac{d^2\ }{dt^2}\int T^{00}_{(0)}x^ix^j\,d^3x,
\end{equation}
y entonces
\begin{equation}
\bar{h}_{\rm rad}^{ij}(\vec{x},t)=-\frac{2G}{c^4} \frac{1}{r}\frac{d^2\ }{dt^2}\left[\frac{1}{c^2}\int T^{00}_{(0)}(x')\,x'^ix'^j\,d^3x'\right]_{\rm ret}.
\end{equation}
Como $T^{00}_{(0)}/c^2=\rho(\vec{x},t)$ es la densidad de masa de la fuente (a primer orden), se acostumbra escribir
\begin{equation}\label{hrad}
\boxed{\bar{h}_{\rm rad}^{ij}(\vec{x},t)=-\frac{2G}{c^4} \frac{1}{r}\left[\ddot{M}^{ij}\right]_{\rm ret},}
\end{equation}
donde
\begin{equation}
M^{ij}(t):=\int \rho(\vec{x},t)\,x^ix^j\,d^3x,
\end{equation}
es el \textit{tensor momento de inercia} (con traza) de la fuente.

Note que $\bar{h}_{\rm rad}^{ij}$ (básicamente, el potencial gravitacional de la onda generada) decae con $1/r$ y que la primera contribución no nula corresponde a \textit{radiación cuadrupolar}. Esta diferencia con el caso de ondas electromagnéticas se debe a que la derivada temporal del \textit{momento dipolar gravitacional} $\int \rho x^i\,d^3x$ es proporcional al momentum lineal del sistema, que es conservado (constante) a primer orden, por lo que no aporta a la energía radiada (que depende de la derivada de $\bar{h}_{\rm rad}^{ij}$).

\subsection{Ejemplo}
Consideremos un ejemplo sencillo, en el que dos cuerpos, cada uno de masa $M$, con una separación $2R$, rotan con velocidad angular $\omega$ en un movimiento no-relativista en torno al centro de masa del sistema. Si modelamos las posiciones de ambas masas por $\vec{x}=\pm(R\cos(\omega t),R\sin(\omega t),0)$, es decir en el plano $xy$ y con posiciones iniciales en el eje $x$, tendremos que (la segunda derivada d)el tensor momento de inercia será
\begin{equation}
 \ddot{M}^{ij}=-4MR^2\omega^2\left(
\begin{array}{ccc}
 \cos(2\omega t) & \sin(2\omega t) &0 \\
\sin(2\omega t) & -\cos(2\omega t) &0 \\
0&0&0
\end{array}\right).
\end{equation}
Con esto, (\ref{hrad}) implica que
\begin{eqnarray}
\bar{h}_{\rm rad}^{ij}(\vec{x},t)&=&\frac{8GMR^2\omega^2}{c^4r}\left(
\begin{array}{ccc}
 \cos\left[2\omega(t-r/c)\right] & \sin\left[2\omega(t-r/c)\right] &0\\
\sin\left[2\omega(t-r/c)\right] & -\cos\left[2\omega(t-r/c)\right] &0\\
0&0&0
\end{array}\right)\\
&=&\frac{8GMR^2\omega^2}{c^4r}\left[
\left(\begin{array}{ccc} 1 & 0 & 0\\ 0 & -1 &0 \\ 0&0&0\end{array}\right) \cos\left[2\omega(t-r/c)\right]
+\left(\begin{array}{ccc} 0 & 1 &0 \\ 1 & 0&0\\ 0&0&0\end{array}\right) \sin\left[2\omega(t-r/c)\right]
\right] \nonumber\\
&=&\frac{8GMR^2\omega^2}{c^4r}\Re\left[
\left(\begin{array}{ccc} 1 & 0 &0 \\ 0 & -1 &0 \\ 0&0&0\end{array}\right) e^{i\left[2\omega(t-\frac{r}{c})\right]} +\left(\begin{array}{ccc} 0 & 1 &0 \\ 1 & 0 &0 \\ 0&0&0\end{array}\right) e^{i\left[2\omega(t-\frac{r}{c})-\frac{\pi}{2}\right]}
\right] .
\end{eqnarray}
Vemos de aquí que \textit{un observador ubicado en un punto sobre el eje} $z$ detectará una onda gravitacional de frecuencia angular $2\omega$, que \textit{satisface automáticamente el gauge TT}, y que es una combinación lineal de las polarizaciones $+$ y $\times$, con una diferencia de fase de $\pi/2$. Este estado es análogo al de una onda electromagnética con polarización circular.

El orden de magnitud de la amplitud de la onda es dada por
\begin{equation}
 h\simeq\frac{8GMR^2\omega^2}{c^4r}=8\left( \frac{GM}{c^2}\right) \left( \frac{1}{r}\right) \left( \frac{\omega R}{c}\right)^2 =8\left( \frac{m}{r}\right) \left( \frac{v}{c}\right) ^2.
\end{equation}
Por ejemplo, el pulsar binario PSR 1913+16 consta de 2 estrellas de neutrones de masa $M\approx 1.4 M_\odot$, con periodo orbital $T\approx 8\text{\,h}\approx 3\times 10^4\text{\,s}$, velocidades orbitales $v\simeq 10^2\text{\,km/s}$, a una distancia $r\approx 2\times 10^4\text{\, ly}\approx 10^{20}\text{\, m}$. En este caso, $m\approx 2\text{\,km}$, $v/c\simeq 3\times 10^{-4}$ y entonces $h\simeq 10^{-23}$ para la radiación recibida en la Tierra, a una frecuencia del orden de $10^{-4}Hz$ (no detectable con LIGO).




