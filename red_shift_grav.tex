
\documentclass[notitlepage,11pt]{article}
%%%%%%%%%%%%%%%%%%%%%%%%%%%%%%%%%%%%%%%%%%%%%%%%%%%%%%%%%%%%%%%%%%%%%%%%%%%%%%%%%%%%%%%%%%%%%%%%%%%%%%%%%%%%%%%%%%%%%%%%%%%%%%%%%%%%%%%%%%%%%%%%%%%%%%%%%%%%%%%%%%%%%%%%%%%%%%%%%%%%%%%%%%%%%%%%%%%%%%%%%%%%%%%%%%%%%%%%%%%%%%%%%%%%%%%%%%%%%%%%%%%%%%%%%%%%
\usepackage{amsfonts}
\usepackage{amssymb}
\usepackage{amsmath}
\usepackage[margin=0.9in]{geometry}
\usepackage[spanish]{babel}

\setcounter{MaxMatrixCols}{10}
%TCIDATA{OutputFilter=LATEX.DLL}
%TCIDATA{Version=5.50.0.2953}
%TCIDATA{<META NAME="SaveForMode" CONTENT="1">}
%TCIDATA{BibliographyScheme=Manual}
%TCIDATA{Created=Sunday, June 14, 2020 21:55:14}
%TCIDATA{LastRevised=Friday, November 13, 2020 11:31:56}
%TCIDATA{<META NAME="GraphicsSave" CONTENT="32">}
%TCIDATA{<META NAME="DocumentShell" CONTENT="Standard LaTeX\Blank - Standard LaTeX Article">}
%TCIDATA{Language=American English}
%TCIDATA{CSTFile=40 LaTeX article.cst}

\newtheorem{theorem}{Theorem}
\newtheorem{acknowledgement}[theorem]{Acknowledgement}
\newtheorem{algorithm}[theorem]{Algorithm}
\newtheorem{axiom}[theorem]{Axiom}
\newtheorem{case}[theorem]{Case}
\newtheorem{claim}[theorem]{Claim}
\newtheorem{conclusion}[theorem]{Conclusion}
\newtheorem{condition}[theorem]{Condition}
\newtheorem{conjecture}[theorem]{Conjecture}
\newtheorem{corollary}[theorem]{Corollary}
\newtheorem{criterion}[theorem]{Criterion}
\newtheorem{definition}[theorem]{Definition}
\newtheorem{example}[theorem]{Example}
\newtheorem{exercise}[theorem]{Exercise}
\newtheorem{lemma}[theorem]{Lemma}
\newtheorem{notation}[theorem]{Notation}
\newtheorem{problem}[theorem]{Problem}
\newtheorem{proposition}[theorem]{Proposition}
\newtheorem{remark}[theorem]{Remark}
\newtheorem{solution}[theorem]{Solution}
\newtheorem{summary}[theorem]{Summary}
\newenvironment{proof}[1][Proof]{\noindent\textbf{#1.} }{\ \rule{0.5em}{0.5em}}

\begin{document}


\section{Redshift gravitacional}

Consideremos que estamos cerca de la superficie de la tierra donde hay un
campo gravitacional constante $\vec{g}$. Adem\'{a}s consideremos una caja de
altura $h$, en la tapa m\'{a}s cercana al suelo hay un emisor $D^{\prime }$
que env\'{\i}a un rayo de luz con frecuencia $\nu _{e}$. En la tapa superior
hay un detector $D$ que mide el rayo enviado por $D^{\prime }$. La pregunta
es \textquestiondown Qu\'{e} frecuencia $\nu _{d}$ mide $D$? Desde el punto
de vista Newtoniano deber\'{\i}an ser iguales, ya que $D$ y $D^{\prime }$ est%
\'{a}n en reposo uno respecto del otro. Pero a la luz de el principio de
equivalencia fuerte $\nu _{e}\neq \nu _{d}$.

\begin{center}
\begin{tabular}{cc}
\FRAME{itbpF}{2.0937in}{2.0539in}{0pt}{}{}{Figure}{\special{language
"Scientific Word";type "GRAPHIC";maintain-aspect-ratio TRUE;display
"USEDEF";valid_file "T";width 2.0937in;height 2.0539in;depth
0pt;original-width 3.4497in;original-height 3.384in;cropleft "0";croptop
"1";cropright "1";cropbottom "0";tempfilename
'QJQN4P03.wmf';tempfile-properties "XPR";}} & \FRAME{itbpF}{1.6501in}{%
2.0522in}{0in}{}{}{Figure}{\special{language "Scientific Word";type
"GRAPHIC";maintain-aspect-ratio TRUE;display "USEDEF";valid_file "T";width
1.6501in;height 2.0522in;depth 0in;original-width 3.966in;original-height
4.9493in;cropleft "0";croptop "1";cropright "1";cropbottom "0";tempfilename
'QJQN3Z02.wmf';tempfile-properties "XPR";}} \\ 
Sistema $K$ & SRLI $K^{\prime }$%
\end{tabular}
\end{center}

Consideremos ahora un sistema de referencia localmente inercial $K^{\prime }$
que cae a la tierra. En este sistema de referencia ambos detectores aceleran
hacia arriba. En este sistema el fot\'{o}n sube con velocidad $c$. Sin
embargo, cuando el fot\'{o}n sube el detector $D$ se aleja. Digamos que en $%
t^{\prime }=0$ el fot\'{o}n es emitido y en $t_{d}^{\prime }$ el foton es
detectado por $D$.

En el sistema $K^{\prime }$ la ecuaci\'{o}n del fot\'{o}n que sube es 
\begin{equation}
y_{f}^{\prime }\left( t^{\prime }\right) =ct^{\prime }
\end{equation}

y la ecuaci\'{o}n del detector $D$ es 
\begin{equation}
y_{D}^{\prime }\left( t^{\prime }\right) =h+\frac{g}{2}t^{\prime 2}.
\end{equation}

Entonces al imponer $y_{f}^{\prime }\left( t_{d}^{\prime }\right)
=y_{D}^{\prime }\left( t_{d}^{\prime }\right) $ podemos obtener cuando llega
el fot\'{o}n a $D$, cuya soluci\'{o}n es 
\begin{equation*}
t_{D}^{\prime }=\frac{c\pm \sqrt{c^{2}-2gh}}{g}
\end{equation*}

debemos elegir el signo menos ya que en este caso cuando hacemos $h=0$
entonces el $t_{D}^{\prime }=0$. Luego, la soluci\'{o}n para nuestro caso es 
\begin{equation}
t_{D}^{\prime }=\frac{c-\sqrt{c^{2}-2gh}}{g}.
\end{equation}

La pregunta es \textquestiondown Cu\'{a}l es la frecuencia $\nu _{d}$? El
cambio en la frecuencia estar\'{a} relacionado con el hecho que el detector $%
D$ viaja con una velocidad no nula al detectar el fot\'{o}n. Por lo tanto es
necesario calcular la velocidad del dector en $t_{D}^{\prime }$, la cual es
dada por 
\begin{eqnarray*}
V_{d}^{\prime } &=&gt_{D}^{\prime }, \\
&=&c-\sqrt{c^{2}-2gh}.
\end{eqnarray*}

El redshift es por definici\'{o}n 
\begin{equation}
Z=\frac{\Delta \lambda }{\lambda _{e}}=\frac{\Delta \nu }{\nu _{e}},
\label{miau}
\end{equation}%
Si tenemos dos observadores $B$ y $B^{\prime }$ que se alejan con velocidad $%
\vec{V}$ con $\vec{V}/c\ll 1$ el efecto Doppler debido a una onda que env%
\'{\i}a $B$ a $B^{\prime }$ es dado por 
\begin{equation*}
\nu _{B^{\prime }}\approx \left( 1+\frac{\left\vert \vec{V}\right\vert }{c}%
\right) \nu _{B}
\end{equation*}%
reemplazando en (\ref{miau}) tenemos%
\begin{equation*}
Z\approx \frac{v}{c}
\end{equation*}

por lo que el redshift gravitacional es dado por, 
\begin{eqnarray*}
Z &\approx &1-\sqrt{1-\frac{2gh}{c^{2}}}, \\
&=&\frac{gh}{c^{2}}+\mathcal{O}\left( \left( \frac{gh}{c^{2}}\right)
^{2}\right) \approx \frac{\Delta \phi }{c^{2}}
\end{eqnarray*}%
Ejemplo: $h\approx 100\left[ m\right] $, $g=9.8\left[ m/s^{2}\right] \approx
10\left[ m/s^{2}\right] $ entonces 
\begin{equation*}
Z\approx 10^{-14}
\end{equation*}%
muy chico! pero se midi\'{o} :)

\end{document}
