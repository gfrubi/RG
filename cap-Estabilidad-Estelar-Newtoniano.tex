\chapter{Equilibrio Estelar Newtoniano}\label{chap:eq_newton}

\section{Ecuaciones de equilibrio}
Consideremos modelar una estrella como una distribución \textit{esférica y estática de un fluido ideal no relativista en equilibrio}, esto es, como aquella situación en la que la presión $P_{mat}$ que ejerce la materia hacia el exterior (debida, por ejemplo, a reacciones termonucleares)  es capaz de mantener el equilibrio de la estrella, contrarrestando su propia
atracción gravitacional, que tiende a comprimirla.

Como estamos considerando un fluido ideal, sin viscosidad, la ecuación de Euler adopta la siguiente forma para el campo de velocidades $\vec{v}$:

\begin{equation}\label{euler}
\rho(\vec{r})\frac{d\vec{v}}{dt}=\rho(\vec{r})\left( \dfrac{\partial\vec{v}}{\partial t}+(\vec{v}\cdot\vec{\nabla})\vec{v}\right)=-\vec{\nabla}P(\vec{r})+\vec{f} (\vec{r}).
\end{equation}
En el caso estático considerado el campo velocidad será nulo $\vec{v}(\vec{x},t)=\vec{0}$. Aquí $\vec{f}$ es la \textit{densidad de fuerza externa} que actúa en cada elemento de volumen. Así, la ec. (\ref{euler}) se reducirá a $\vec{\nabla}P=\vec{f}$. Pero en este caso la única fuerza externa considerada es la gravitacional, que actúa sobre un elemento de masa $dm=\rho(\vec{r})dV$ con densidad $\rho(\vec{r})$, y es por lo tanto dado por $d\vec{F}_g=-dm\vec{\nabla}\phi$, donde $\phi$ es el potencial gravitacional newtoniano. De este modo, la densidad de fuerza será $\vec{f}=-\rho\vec{\nabla}\phi$, y por lo tanto, la condición de equilibrio queda expresada por:
\begin{equation}\label{equi}
\vec{\nabla}P(\vec{r})=-\rho\vec{\nabla}\phi(\vec{r}).
\end{equation}

Al asumir \textit{simetría esférica}, tenemos que $P=P(r)$ y $\phi=\phi(r)$, entonces
\begin{equation}\label{equir}
\frac{dP(\vec{r})}{dr}=-\rho(\vec{r})\frac{d\phi}{dr}.
\end{equation}

Por otro lado, el potencial gravitacional $\phi$ satisface la ecuación de Poisson:
\begin{equation}\label{n2p}
\nabla^2\phi(r)=4\pi G\rho(r).
\end{equation}

Integrando la ecuación anterior desde el centro de la estrella, $r=0$, hasta un radio arbitrario $r$,  y denotando la \textit{masa dentro del radio} $r$ de la estrella por
\begin{equation}\label{masa1}\marginnote{Ecuación de masa}
\mathcal{M}(r)=4\pi\int\limits^r_{0}dr'r'^2\rho(r'),
\end{equation}
o, equivalentemente
\begin{equation}\label{masa2}\marginnote{Ecuación de masa como derivada}
\boxed{\frac{d\mathcal{M}(r)}{dr}=4\pi r^2\rho(r),}
\end{equation}
tenemos luego que \eqref{n2p} implica (asumiendo que \textit{no existe} una masa puntual en el centro):
\begin{equation}
\frac{d\phi}{dr}=\frac{4\pi G}{r^2}\int\limits^r_{0}dr'r'^2\rho(r')=\frac{G\mathcal{M}(r)}{r^2}.
\end{equation}

Reemplazando la expresión anterior en (\ref{equir}), tenemos la condición de equilibrio hidrostático newtoniano:
\begin{equation}\label{eqnewton}\marginnote{Equilibrio hidrostático newtoniano}
\boxed{\frac{dP(r)}{dr}=-\frac{G\mathcal{M}(r)}{r^2}\rho(r).}
\end{equation}

Notemos que, debido a que $\rho>0$, \textbf{la presión es una función monótonamente decreciente de la coordenada radial}.


\section{Resolviendo las ecuaciones de estructura}\label{resolviendo}

La ecuación de equilibrio hidrostático (\ref{eqnewton}) y la ecuación de masa, (\ref{masa1}) ó (\ref{masa2}), conforman un sistema de \textit{dos ecuaciones} en las que hay \textit{tres campos escalares}, dependientes de la coordenada radial, por determinar: la presión $P(r)$, la densidad $\rho\,(r)$ y la masa $\mathcal{M}(r)$. Por lo tanto, necesitamos otra ecuación que ligue a las variables anteriores para que exista una solución única a una determinada configuración. Usualmente, esta ecuación faltante liga a la presión con la densidad (cuando la entropía $s$ es constante) y se conoce como \emph{ecuación de estado}\footnote{Para mayores detalles, ver apéndice  \ref{cap:termo}.}:
\begin{equation}\label{estado}\marginnote{Ecuación de estado}
\boxed{P=P(\rho).}
\end{equation}

Así, tenemos un sistema de dos ecuaciones diferenciales y una ecuación algebraica para la estructura estelar, que adoptan la forma
\begin{align}
 P'(r)&=P'(P(r),\mathcal{M}(r)),\\
\mathcal{M}'(r)&=\mathcal{M}(\rho(r)),\\
P(r)&=P(\rho(r)).
\end{align}

Luego, el sistema físico modelado por el sistema de ecuaciones (\ref{eqnewton}), (\ref{masa2}) y (\ref{estado}) quedará completamente determinado imponiendo dos condiciones iniciales apropiadas (tenemos dos ecuaciones diferenciales de primer orden):
\begin{enumerate}
 \item $\mathcal{M}(r=0)=0$. Esto sucederá siempre que $\rho(r=0)$ sea \textit{finito}, como es razonable suponer.
 \item $P(r=0)=P_0$. Se asigna un valor dado a la presión en el centro de la estrella.
\end{enumerate}

Como la presión \marginnote{Procedimiento de \\resolución} es una función monótonamente decreciente (puesto que $\rho(r)\geq0$), la determinación de un modelo estelar $[P(r), \rho(r),\mathcal{M}(r)]$, dada una ecuación de estado y una cierta presión central, se obtendrá integrando el sistema de ecuaciones antes mencionado desde el centro hacia afuera, hasta llegar a un punto $r=R$ tal que $P(r=R)=0$. A dicha coordenada $R$ se le denominará \textit{radio de la estrella}. Aquí se debe considerar la imposición física que $(\forall\; r\geq R) \quad P=0 \quad\mbox{y}\quad \rho=0$. Evaluando la ecuación (\ref{masa1}) en el radio $R$, definiremos \textit{la masa total $M$ de la estrella}  como
\begin{equation}\marginnote{Masa total}
M:=\mathcal{M}(r=R).
\end{equation}



\section{Solución: Densidad constante}
\subsection{Obteniendo la presión  \texorpdfstring{$P(r)$}{P(r)}}
El caso más simple de resolver el sistema de ecuaciones de equilibrio estelar es asumir la ecuación de estado para densidad constante,
\begin{equation}\label{cte0}
\rho=\rho_{\rm c}=cte,\quad r\leq R,
\end{equation}
i.e., considerar materia incompresible. ésta es una suposición poco realista, pero el modelo estelar así construido ya muestra muchas de las características de otros más complejos.

Con dicha ecuación de estado, es posible integrar directamente la ecuación de masa (\ref{masa1}):

\begin{equation}\label{cte1}
\mathcal{M}(r)=4\pi\int^r_{0}dr'r'^2\rho_{\rm c}=4\pi\rho_{\rm c}\int^r_{0}dr'r'^2=\frac{4}{3}\pi\rho_{\rm c} r^3.
\end{equation}

Luego, reemplazando lo anterior en la ecuación de equilibrio hidrostático (\ref{eqnewton}) e integrando de $r=0$ a $r=r$ (recordar que $P(r=0)=P_0$ es la presión central), obtenemos
\begin{align*}
 \frac{dP(r)}{dr}&=-\frac{G\mathcal{M}(r)}{r^2}\rho_{\rm c}=-\frac{4}{3}\pi G\rho_{\rm c}^2 r,
\end{align*}
y por lo tanto,

\begin{equation}
\label{cte2}\boxed{P(r)=P_0-\frac{2}{3}\pi G\rho_{\rm c}^2 r^2.}
\end{equation}

\marginnote{Solución \text{con $\rho=cte$}}
Así, hemos resuelto el problema al obtener la forma explícita de los tres campos escalares involucrados: ver (\ref{cte0}), (\ref{cte1}) y (\ref{cte2}). Ahora podemos encontrar el radio total de la estrella en función de la presión central, de acuerdo a la definición anterior ($P(R)=0$), evaluando (\ref{cte2}) en $r=R$:
\begin{equation}\label{ctep}
 P_0=\frac{2}{3}\pi G\rho_{\rm c}^2 R^2.
\end{equation}
Por otra parte, la masa total de la estrella se encuentra, de acuerdo a su definición, evaluando (\ref{cte1}) en $r=R$ dado por (\ref{ctep}):
\begin{equation}\label{ctem}
 M=\mathcal{M}(R)=\frac{4}{3}\pi\rho_{\rm c} R^3.
\end{equation}
Si se reemplaza (\ref{ctem}) en (\ref{ctep}) y se recuerda la definición según Relatividad General del radio de Schwarzschild,
\begin{equation}\label{radiosch}
 r_{\rm s}=\frac{2GM}{c^2},
\end{equation}
tendremos que
\begin{align*}
P_0 &=\rho_{\rm c} c^2\frac{r_{\rm s}}{4R},
\end{align*}
y por lo tanto,\marginpar{\footnotesize \vspace{0.5cm}Solución para $P(r)$ en términos de $r_S$}
\begin{equation}\label{presionnewton}
 \boxed{P(r)=\rho_{\rm c} c^2\left(\frac{r_S}{4 R^3}\right)\left(R^2-r^2\right)}.
\end{equation}

\subsection{Validez de la descripción newtoniana}

De las relaciones anteriores es posible establecer la condición:
\begin{align}
\frac{r_{\rm s}}{4R}&=\frac{P_0}{\rho_{\rm c} c^2}. \label{rango}
\end{align}
En general, el cuociente anterior será muy peque\~no para materia en estrellas corrientes, indicando que los efectos relativistas son despreciables y por lo tanto imponiendo una condición para la validez de la solución de equilibrio estelar newtoniana presentada. En forma equivalente, podemos decir que la descripción newtoniana desarrollada antes es válida si el radio de la estrella es mucho mayor que el radio de Schwarzschild asociado a su masa. Una estimación de órdenes de magnitud extendido a casos en que $\rho$ no sea constante sería por lo tanto
\marginnote{Validez de la descripción no relativista}
\begin{equation}\label{cuo}
\frac{r_s}{4R}\sim\frac{P_0}{\rho c^2}\ll 1.
\end{equation}
En caso que no se cumpla la desigualdad anterior, es posible anticipar que los efectos relativistas no serán despreciables. Para comprobar esta afirmación, podemos evaluar aproximadamente (\ref{rango}) \marginnote{Efectos relativistas \\para el Sol} en ciertos casos particulares. Por ejemplo, para una estrella ordinaria de secuencia principal como el Sol, se puede asumir que su materia satisface la ecuación de estado de los gases ideales,

\begin{equation}
P=\frac{\rho k_{\rm B} T}{m},
\end{equation}
en donde $T$ es la temperatura, $k_B$ la constante de Boltzmann y $m$ es la masa de cada una de las partículas del gas ideal. Entonces, de (\ref{cuo}) tenemos que
\begin{equation}
\frac{r_{\rm s}}{4R}\sim\frac{P}{\rho c^2}=\frac{k_{\rm B}T}{mc^2}.
\end{equation}

El constituyente principal de estrellas como el Sol son átomos de hidrógeno, que con una masa atómica $m_p$ poseen una energía en reposo
\begin{equation}
mc^2\approx1\;GeV.
\end{equation}
Además, se sabe que en el centro del Sol las reacciones termonucleares producen una temperatura del orden de $T\sim 10^7\; K$, cuya energía asociada es
\begin{equation}\label{17}
k_{\rm B}T\sim 1\; {\rm keV}.
\end{equation}
Luego, evaluando (\ref{cuo}), encontramos que
\marginnote{Efectos relativistas para una estrella ``normal''}\begin{equation}\label{18}
\frac{r_S}{4R}\sim\frac{k_{B}T}{mc^2}\sim10^{-6},
\end{equation}
lo que indica que los efectos relativistas para una estrella ``normal'' como el Sol son despreciables, y por lo tanto es válido usar la aproximación newtoniana.

\section{Solución: Estrellas politrópicas}
\subsection{Ecuación de Lane-Emden}
En esta sección resolveremos el sistema de ecuaciones de estructura estelar asumiendo una relación $P=P(\rho)$ un poco más realista que la anterior, en la forma de una ecuación de estado politrópica \eqref{estadopolitropica}. Para determinar las incógnitas $P$, $\rho$ y $\mathcal{M}$, se parte de la ecuación de equilibrio hidrostático (\ref{eqnewton}), derivándola con respecto a $r$:
\begin{align}
\quad\frac{r^2}{\rho}\frac{dP}{dr}&=-G\mathcal{M},\quad\\
\Rightarrow\qquad\frac{d}{dr}\left(\frac{r^2}{\rho}\frac{dP}{dr}\right)&=-G\frac{d\mathcal{M}}{dr}\\
&=-4\pi G\rho r^2,\label{eq-hidrostatico-sinm}
\end{align}
en donde en la última igualdad se ha usado la ecuación de masa en forma de derivada (\ref{masa2}), eliminando así $\mathcal{M}$ y reduciendo el número de incógnitas a 2. Para que la ecuación anterior quede expresada únicamente en la variable $\rho$, se requiere usar ahora la ecuación de estado politrópica (\ref{estadopolitropica}), notando que ${dP}/{dr}=K\gamma\rho^{\gamma-1}d\rho/dr$. Así,
\begin{equation}\label{laneprevia}
 \frac{d}{dr}\left(\frac{r^2}{\rho}\frac{dP}{dr}\right)=K\gamma\frac{d}{dr}\left(r^2\rho^{\gamma-2}\frac{d\rho}{dr}\right)=-4\pi G\rho r^2.
\end{equation}

Luego, hemos reducido el sistema a una ecuación diferencial de segundo orden para $\rho(r)$. Como tal, se requieren dos condiciones de borde para encontrar una solución particular:
\begin{enumerate}
\item $\rho(0)=\rho_{\rm c}<\infty$: La densidad debe ser \textit{finita} en el centro de la estrella.
\item $\rho'(0)=0$: El gradiente de densidad debe ser nulo en el centro de la estrella. Esto implica que la densidad alcanza su máximo en el centro de la estrella (ya que la presión es monótonamente decreciente)\footnote{La justificación de este hecho proviene de derivar explícitamente el lado izquierdo de la ecuación de Lane-Emden \eqref{laneprevia} (denotando $()':=d/dr$)
\begin{align}
\left(r^2\rho^{\gamma-2}\rho'\right)'&=-\frac{4\pi G}{K \gamma}\rho r^2,\\
 2r\rho^{\gamma-2}\rho'+r^2(\gamma-2)\rho^{\gamma-3}\rho'+r^2\rho^{\gamma-2}\rho''&=-\frac{4\pi G}{K \gamma}\rho r^2,
\end{align}
y para $r\to 0$:
\begin{equation}
2\left[\rho(0)\right]^{\gamma-3}\left(\frac{\left[\rho(0)\right]'}{r}\right)+(\gamma-2)\left[\rho(0)\right]^{\gamma-4}\left[\rho(0)\right]'
+\left[\rho(0)\right]^{\gamma-3}\left[\rho(0)\right]''=-\frac{4\pi G}{K \gamma}=cte.
\end{equation}
Así, el primer término divergerá a menos que
\begin{equation}\label{divergencia2}
 \lim_{r\to 0}\left(\frac{\rho'}{r}\right)<\infty\quad\Rightarrow\quad\rho'(0)=0.
\end{equation}}.
\end{enumerate}

Ahora bien, para simplificar (\ref{laneprevia}), se introducen las variables adimensionales para la coordenada radial $x$ y la densidad $\Theta(x)$, relacionadas con sus respectivas cantidades físicas por:
\begin{align}
 r&=ax&\Leftrightarrow& &x&:=\frac{r}{a},\label{x}\\
\rho&=\rho_{\rm c}\,\Theta(x)^{\frac{1}{\gamma-1}}&\Leftrightarrow& &\Theta(x)&:=\left(\frac{\rho}{\rho_{\rm c}}\right)^{\gamma-1}\label{theta},
\end{align}
en donde $\Theta(x)$ se denomina \textit{función de Lane-Emden}\footnote{\href{http://en.wikipedia.org/wiki/Jonathan_Homer_Lane}{Jonathan Lane} (1819-1880): astrofísico e inventor estadounidense.} \footnote{\href{http://en.wikipedia.org/wiki/Robert_Emden}{Jacob Robert Emden} (1862-1940): astrofísico y meteorólogo sueco.}. y  $a$ es una escala de longitud. Para determinarla explícitamente, se sustituyen las expresiones anteriores en (\ref{laneprevia}), obteniendo:
\begin{align}
 K\gamma\frac{d}{d(ax)}\left((ax)^2\left(\rho_{\rm c}\Theta^{\frac{1}{\gamma-1}}\right)^{\gamma-2}\frac{d}{d(ax)}\left( \rho_{\rm c}\Theta^{\frac{1}{\gamma-1}}\right)\right)&=-4\pi G \left(\rho_{\rm c} \Theta^{\frac{1}{\gamma-1}}\right)\, \left(ax\right)^2,\\
\frac{1}{x^2}\frac{d}{dx}\left(x^2\frac{d\Theta}{dx}\right)&=-a^2\frac{4\pi G(\gamma-1)}{K\gamma}\frac{1}{\rho_{\rm c}^{\gamma-2}}\Theta^{\frac{1}{\gamma-1}}.
\end{align}
Así, defininiendo:
\begin{equation}\label{lanemden-a}
 a=\left(\frac{K\gamma}{4\pi G(\gamma-1)}\right)^{1/2}\rho_{\rm c}^{\frac{\gamma}{2}-1},
\end{equation}
obtenemos la llamada \emph{ecuación de Lane-Emden} de índice $1/(\gamma-1)$,
\begin{equation}\label{laneemden}\marginnote{Ecuación de Lane-Emden}
 \boxed{\frac{1}{x^2}\frac{d}{dx}\left(x^2\frac{d\Theta}{dx}\right)+\Theta^{\frac{1}{\gamma-1}}=0.}
\end{equation}
Por conveniencia, la relación anterior también puede ser expresada en términos del siguiente parámetro auxiliar:
 \begin{equation}\label{ngamma}
 n=\frac{1}{\gamma-1}
\end{equation}
Además, las condiciones de borde antes mencionadas serán:
\begin{enumerate}
 \item \begin{equation}\label{bcs_laneemden1}
\Theta(0)=\left(\frac{\rho(r=0)}{\rho_{\rm c}}\right)^{\gamma-1}=\left(\frac{\rho_{\rm c}}{\rho_{\rm c}}\right)^{\gamma-1}=1,
\end{equation}

\item
\begin{equation}\label{bcs_laneemden2}
\Theta'(0)=\left.\frac{d}{dr}\left(\frac{\rho}{\rho_{\rm c}}\right)^{\gamma-1}\right|_{0}=(\gamma-1)\left.\frac{\rho^{\gamma-2}}{\rho_{\rm c}^{\gamma-1}}\rho'\right|_0=(\gamma-1)\frac{1}{\rho_{\rm c}^{\gamma}}\underbrace{\cancelto{0}{\rho'(0)}}_{ \eqref{divergencia2}}=0.
\end{equation}
\end{enumerate}
Esta es la ecuación fundamental que determina la estructura estelar para una estrella newtoniana con ecuación de estado politrópica.



\subsection{Propiedades físicas de las funciones de Lane-Emden}

La propiedad básica que satisfacen las soluciones $\Theta(x)$ de la ecuación de Lane-Emden es $d\Theta(x)/dx<0$, es decir, son monótonamente decrecientes desde su máximo en el origen $\Theta(x=0)=1$. Esto equivalente físicamente a que la densidad sea máxima en el centro de la estrella: $\rho(r=0)=\rho_{\rm c}$, y que decrezca conforme la coordenada radial aumenta. Para probar esta propiedad, nos remitimos a la ecuación de equilibrio hidrostático \eqref{eqnewton}, la cual simplificamos mediante el uso de la ecuación de estado politrópica \eqref{estadopolitropica} y la reexpresamos en términos de las variables $x$ \eqref{x} y $\Theta$ \eqref{theta}:
\begin{align}
\gamma K\rho^{\gamma-1}\frac{d\rho}{dr}&=-\frac{G\mathcal{M}}{r^2}\rho\\
\Rightarrow\quad \frac{d\Theta}{dx}&=-\frac{\frac{G\mathcal{M}}{r^2}\rho}{\gamma K \rho^{\gamma-1}\rho_{\rm c}\frac{1}{a(\gamma-1)}\Theta^{\frac{2-\gamma}{\gamma-1}}}\label{theta-decreciente}
\end{align}
De donde se observa fácilmente que la propiedad mencionada se satisface, pues todos los factores del lado derecho son definidos positivos por requerimientos físicos. Ahora bien, en base a esta consideración es que podemos hallar las siguientes variables características de una estrella modelada por la ecuación de estructura de Lane-Emden:

\subsubsection{Radio estelar politrópico}

Por consideraciones físicas, es de esperar que debido al comportamiento decreciente de la densidad mencionado, exista un punto donde se llegue al borde de la estrella y ésta se anule. De hecho, para $\gamma>6/5$\footnote{Ver subsección \ref{sec:exactas-n5} para una justificación}, existe una coordenada radial adimensional $x=x_1$ tal que la función de Lane-Emden posee una raíz allí: $\Theta(x_1)=0$. Esto equivale a decir, por su definición (\ref{theta}), que la densidad se anula para el radio correspondiente a dicho punto: $R=ax_1\Rightarrow\rho(R)=0$, y por la ecuación de estado (\ref{estadopolitropica}), tenemos que lo anterior también implica que la presión se anula allí: $P(R)=0$. A la coordenada radial $R$ donde sucede eso, se le define, por las propiedades anteriores, como el radio de la estrella. Por lo tanto, de la definición de $x$ \eqref{x} y de la escala de longitud $a$ en \eqref{lanemden-a}, tenemos que el radio estelar será expresable en función de la raíz de la solución de Lane-Emden $x_1$ para un $\gamma$ ó $n$  dado como:
\begin{equation}\label{radiopolitropico}\marginnote{Radio de la estrella}
\boxed{
\begin{aligned}
R&=\left(\frac{K\gamma}{4\pi G(\gamma-1)}\right)^{1/2}\rho_{\rm c}^{\frac{\gamma}{2}-1}x_1\\
&=\left(\frac{(1+n)K}{4\pi G}\right)^{1/2}\rho_{\rm c}^{\frac{1-n}{2n}}x_1
\end{aligned}}
\end{equation}

Debido a la dificultad de la resolución analítica de la ecuación de Lane-Emden, las raíces $x_1$ de $\Theta(x)$ se hallan, en general, mediante métodos numéricos, los que se describirán en la sección \ref{sec:lane-numerico}. Allí se obtendrá la tabla \ref{tablalaneemden} que proporciona dichos valores de $x_1$ para distintos índices $n$ de la ecuación de Lane-Emden.

\subsubsection{Masa estelar politrópica}

La masa de la estrella al interior del radio $r$ vendrá dada  por la integral \eqref{masa1}, en donde se integra hasta la coordenada normalizada $x$ usando las definiciones \eqref{x} y \eqref{theta}:
\begin{align}
\mathcal{M}(x)&=4\pi\int_0^r dr'\,r'^2\rho(r'),\\
&=4\pi\int_0^{x} \left(d(ax)'\right)\,(ax')^{2}\left(\rho_{\rm c} \Theta^{\frac{1}{\gamma-1}}\right)\label{masapol1},\\
&=4\pi a^3\rho_{\rm c}\int\limits_0^{x}dx'\, x'^2\Theta^{\frac{1}{\gamma-1}}
% &=4\pi\rho_{\rm c}^{\frac{3\gamma-4}{2}}\left(\frac{K\gamma}{4\pi G(\gamma-1)}\right)^{3/2}\int\limits_0^{x}dx'\, x'^2\Theta^{\frac{1}{\gamma-1}}.
\end{align}
Usando la ecuación de Lane-Emden (\ref{laneemden}), es posible expresar la masa en términos de la derivada de la función de Lane-Emden  $\Theta'(x)$, ya que:
\begin{align}
 \frac{d}{dx}\left(x^2\frac{d\Theta}{dx}\right)&=-x^2\Theta^{\frac{1}{\gamma-1}},\\
\Rightarrow\quad x^{2}\Theta'(x)&=-\int\limits_0^{x}dx'\,x'^2\Theta^{\frac{1}{\gamma-1}},\label{masapol2}
\end{align}
y como $\Theta'(x)<0$, tomando su valor absoluto tenemos para la masa de la estrella al interior de $r$, usando la definición de la escala de longitud $a$ dada en \eqref{lanemden-a}:
\begin{equation}
 \mathcal{M}(x)=4\pi a^3\rho_{\rm c}\, x^2\left|\Theta'(x)\right|=4\pi\left(\frac{K\gamma}{4\pi G(\gamma-1)}\right)^{3/2}\rho_{\rm c}^{\frac{3\gamma-4}{2}}x^2\left|\Theta'(x)\right|.\label{masaparcial-laneemden}
\end{equation}
De aquí se puede obtener fácilmente la masa total de la estrella, evaluando la expresión anterior a partir de \eqref{x} en el radio $r=R=ax_1$:
\begin{equation}\label{masalaneemden}\marginnote{Masa de la estrella}
 \boxed{
\begin{aligned}M&=4\pi\left(\frac{K\gamma}{4\pi G(\gamma-1)}\right)^{3/2}\rho_{\rm c}^{\frac{3\gamma-4}{2}}x_1^2\left|\Theta'(x_1)\right|\\
 &=4\pi\left(\frac{(1+n)K}{4\pi G}\right)^{3/2}\rho_{\rm c}^{\frac{3-n}{2n}}x_1^2\left|\Theta'(x_1)\right|.
\end{aligned}}
\end{equation}
Además, notando que de \eqref{x}:
\begin{equation}\label{laneemden-rax}
a=\frac{r}{x}=\frac{R}{x_1}\quad\Rightarrow\quad x=x_1\frac{r}{R},
\end{equation}
podemos expresar la masa parcial (aquella al interior del radio $r$) en términos de la masa total de la estrella, dividiendo \eqref{masaparcial-laneemden} entre \eqref{masalaneemden}:
\begin{equation}\label{masapoli-en-r}
\mathcal{M}(r)=\left\{\frac{\left(x_1\frac{r}{R}\right)^2\left|\Theta'\left(x_1\frac{r}{R}\right)\right|}{x_1^2\left|\Theta'(x_1)\right|}\right\}M
\end{equation}
Debido a su aparición explícita en las relaciones anteriores, la cantidad $x_1^2\left|\Theta'(x_1)\right|$, que es obtenida mediante la solución numérica de la ecuación de Lane-Emden, también se muestra explícitamente en la tabla \ref{tablalaneemden}.

\subsubsection{Relación Masa-Radio}
De este modo, tanto el radio de la  estrella dado por \eqref{radiopolitropico} y su masa dada por \eqref{masalaneemden}, quedan completamente determinados para una ecuación de estado politrópica con un índice adiabático $\gamma$ dado, debido a su dependencia en $x_1$ y $x_1^2\left|\Theta'(x_1)\right|$, respectivamente. Por lo tanto, el radio y la masa son funciones bien definidas de la densidad central $\rho_{\rm c}$:
\begin{align}
M&=cte(\gamma)\cdot\rho_{\rm c}^{\frac{3\gamma-4}{2}}=cte(n)\cdot\rho_{\rm c}^{\frac{3-n}{2n}}\label{lane-masagamma}\\
R&=cte(\gamma)\cdot\rho_{\rm c}^{\frac{\gamma}{2}-1}=cte(n)\cdot\rho_{\rm c}^{\frac{1-n}{2n}}\label{lane-radiogamma}
\end{align}
Además, aún podemos encontrar otra expresión que relacione la masa con el radio de la estrella directamente. Para ello, primero despejamos la densidad central en términos del radio de la estrella mediante \eqref{radiopolitropico}
\begin{equation}
 \rho_{\rm c}=\left(\frac{R}{x_1}\right)^{\frac{2}{\gamma-2}}\left(\frac{4\pi G(\gamma-1)}{K\gamma}\right)^{\frac{1}{\gamma-2}}=\left(\frac{R}{x_1}\right)^{\frac{2n}{1-n}}\left(\frac{4\pi G}{(1+n)K}\right)^{\frac{n}{n-1}}
\end{equation}
Luego, reemplazando en \eqref{masalaneemden},
\begin{equation}
 M=4\pi\left[\left(\frac{R}{x_1}\right)^{\frac{2}{\gamma-2}}\left(\frac{4\pi G(\gamma-1)}{K\gamma}\right)^{\frac{1}{\gamma-2}}\right]^{\frac{3\gamma-4}{2}}\left(\frac{K\gamma}{4\pi G(\gamma-1)}\right)^{3/2}x_1^2\left|\Theta'(x_1)\right|,
\end{equation}
obtenemos finalmente la relación masa-radio de estrellas politrópicas:
\begin{equation}\label{masaradiopolitropico}
\boxed{
\begin{aligned}
 M&=4\pi R^{\frac{3\gamma-4}{\gamma-2}}\left(\frac{K\gamma}{4\pi G(\gamma-1)}\right)^{\frac{1}{2-\gamma}}x_1^{-\frac{3\gamma-4}{\gamma-2}}\,x_1^2\left|\Theta'(x_1)\right|\\
&=4\pi R^{\frac{n-3}{n-1}}\left(\frac{(1+n)K}{4\pi G}\right)^{\frac{n}{n-1}}x_1^{-\frac{n-3}{n-1}}\,x_1^2\left|\Theta'(x_1)\right|
\end{aligned}}
\end{equation}


\subsubsection{Densidad central y media *}
 La densidad media de materia $\overline{\rho}(r)$ al interior de la coordenada $r$ de la estrella se define por:
\begin{equation}\label{defin-densidadmedia}
 \overline{\rho}(x)=\frac{\mathcal{M}(x)}{\frac{4}{3}\pi r^3}
\end{equation}
Usando \eqref{x} y la definición de masa parcial \eqref{masaparcial-laneemden}, expresada en términos de $a$ \eqref{lanemden-a}, tenemos que:
\begin{align}
 \overline{\rho}(x)&=\frac{4\pi\rho_{\rm c}\, a^3 x^2\left|\Theta'(x)\right|}{\frac{4}{3}\pi(ax)^3}\\
 &=3\rho_{\rm c}\left[\frac{x^2\left|\Theta'(x)\right|}{x^3}\right]\label{densidadmedia1}
\end{align}
Si evaluamos la expresión anterior en el radio adimensional de la estrella $x=x_1$, encontramos el interesante resultado:
\begin{equation}\label{laneemden-densidadmedia}
\boxed{ \overline{\rho}:=\overline{\rho}(x_1)=3\rho_{\rm c}\frac{x_1^2\left|\Theta'(x_1)\right|}{x_1^3}},
\end{equation}
es decir, la densidad media total de la estrella es un múltiplo de su densidad central. Pero como $\bar{\rho}$ se puede expresar explícitamente en función de la masa y radio totales de la estrella,
\begin{equation}
 \overline{\rho}=\frac{M}{\frac{4}{3}\pi R^3}=3\rho_{\rm c}\frac{x_1^2\left|\Theta'(x_1)\right|}{x_1^3},
\end{equation}
tenemos que la densidad central se puede determinar explícitamente si la masa y radio totales de la estrella son conocidos, además del índice politrópico $\gamma$, pues está dada por:
\begin{equation}\label{densidad0politropica}
\boxed{ \rho_{\rm c}=\frac{1}{4\pi}\frac{x_1^3}{x_1^2\left|\Theta'(x_1)\right|}\frac{M}{R^3}.}
\end{equation}

% Por último, podemos expresar \eqref{densidadmedia1}, en términos de la coordenada radial $r$ y cantidades conocidas mediante \eqref{defin-densidadmedia} y \eqref{masapoli-en-r}
% \begin{align}
%  \overline{\rho}(r)=\frac{M(r)}{\frac{4}{3}\pi r^3}=\frac{M}{\frac{4}{3}\pi r^3}\left\{\frac{\left(x_1\frac{r}{R}\right)^2\left|\Theta'\left(x_1\frac{r}{R}\right)\right|}{x_1^2\left|\Theta'(x_1)\right|}\right\}
% \end{align}

\subsection{Soluciones exactas de la ecuación de Lane-Emden}\label{sec:exactas}
En general, la ecuación de Lane-Emden es difícil de resolver analíticamente, excepto para ciertos casos particulares, los cuales se obtendrán en esta sección. En efecto, según el texto de Chandrasekhar \cite{Chandra39}, existen tres valores de $n$ para los cuales existe una solución analítica de la ecuación anterior:

\subsubsection{Caso \texorpdfstring{$n=0$}{n0}}

Físicamente, de \eqref{ngamma}, es posible ver que este caso equivale al límite en que el índice adiabático $\gamma\to\infty$. De la ecuación de estado adiabático \eqref{estadopolitropica}, podemos ver que
\begin{equation}
 P=K\rho^{\gamma}\quad\Rightarrow\quad \rho=\rho_0\left(\frac{P}{P_0}\right)^{1/\gamma},
\end{equation}
y para $\gamma\to\infty$
\begin{equation}
 \rho\to\rho_0\left(\frac{P}{P_0}\right)^{0}=\rho_0=\text{cte.},
\end{equation}
es decir, este caso corresponde al de materia incompresible.

Ahora, para resolver \eqref{laneemden} en este caso $n=0$, se debe notar que es posible su integración directa:
\begin{align}
 &\frac{1}{x^2}\frac{d}{dx}\left(x^2\frac{d\Theta}{dx}\right)=-1\\
\Rightarrow \quad &x^2\frac{d\Theta}{dx}=-\int x^2\,dx=-\frac{x^3}{3}-C\\
\Rightarrow\quad &\frac{d\Theta}{dx}=-\frac{x}{3}-\frac{C}{x^2},
\end{align}
con $C$ una constante de integración. Integrando nuevamente, y denotando con $D$ a la nueva constante de integración, notamos que:
\begin{equation}
 \Theta(x)=D+\frac{C}{x}-\frac{x^2}{6}.
\end{equation}
Pero de la condición de borde \eqref{bcs_laneemden2}, que implica en particular que $\Theta$ sea finita en el origen, encontramos $C=0$. Finalmente, aplicando la otra condición de borde \eqref{bcs_laneemden1}
\begin{equation}
 \Theta(0)=1=D-\frac{0}{6}\quad\Rightarrow\quad D=1,
\end{equation}
tenemos que la solución de Lane-Emden para el caso $n=0$ será:
\begin{equation}\label{lane0}
 \boxed{\Theta_0=1-\frac{x^2}{6}.}
\end{equation}
que es monónotonamente decreciente, y cuya primera y única raíz para $x>0$ está en $x_1=\sqrt{6}\approx2.44$. También se puede mostrar que este número es el menor entre todas las raíces soluciones de la ecuación de Lane-Emden para un $n$ arbitrario.

\subsubsection{Caso \texorpdfstring{$n=1$}{n1}}
En este caso, con $\gamma=2$, la ecuación a resolver \eqref{laneemden} adopta la forma:
\begin{align}
&\frac{1}{x^2}\frac{d}{dx}\left(x^2\frac{d\Theta}{dx}\right)=-\Theta,\\
&\frac{d}{dx}\left(x^2\frac{d\Theta}{dx}\right)+\Theta x^2=0,\label{bessel}
\end{align}
que es equivalente a la ecuación esférica de Bessel de orden $n$:
\begin{equation}
 \frac{d}{dr}\left(r^2\frac{dR}{dr}\right)+\left[k^2 r^2-n(n+1)\right]R=0
\end{equation}
en donde, para este caso, $k=1$ y $n=0$. Luego, la solución general de \eqref{bessel} es conocida y dada por:
\begin{equation}
 \Theta=Aj_0(x)+Bn_0(x),
\end{equation}
en donde $j_0(x)=\sin x/x$ es la función esférica de Bessel del primer tipo de orden $n=0$, y $n_0(x)=-\cos x/x$ es la función esférica de Bessel del segundo tipo de orden $n=0$. Pero, para respetar la condición de borde \eqref{bcs_laneemden2}, se requiere que  $B=0$ pues $n_0(x)$ diverge en el origen. Aplicando la otra condición de borde \eqref{bcs_laneemden1}, encontramos directamente que $A=1$, por lo que usando la forma conocida de $j_0(x)$, tenemos que la solución de Lane-Emden para $n=1$ es:
\begin{equation}\label{lane1}
 \boxed{\Theta_1=\frac{\sin x}{x}}
\end{equation}
Esta función también es monótonamente decreciente en el intervalo $[0,\pi]$, y su primera raíz es $x_1=\pi$. Tal cual se había comentado después de \eqref{lane0}, se verifica la desigualdad $x_{1,n=0}=\sqrt{6}<\pi=x_{1,n=1}.$ (mínimo radio estelar para $n=0$)


\subsubsection{Caso \texorpdfstring{$n=5$}{n5}}\label{sec:exactas-n5}
Para abordar este caso, conviente primero definir una nueva variable para el inverso de la longitud:
\begin{equation}\label{lane5cambio}
 \xi=\frac{1}{x}\qquad\Rightarrow\qquad \frac{d}{d x}=-\xi^2\frac{d}{d\xi}.
\end{equation}
Así, Lane-Emden \eqref{laneemden} en términos de la nueva variable $\xi$ es:
\begin{align}
\xi^2(-\xi^2)\frac{d}{d\xi}\left(\frac{1}{\xi^2}(-\xi^2\frac{d\Theta}{d\xi})\right)&=-\Theta^n, \\
\xi^4\frac{d^2\Theta}{d\xi^2}=-\Theta^n.\label{transkelvin}
\end{align}

Escribimos la solución en la forma siguiente:
\begin{equation}\label{lane5solgeneral}
 \Theta(\xi)=a\xi^{\omega}z(\xi),
\end{equation}
donde $a$ y $\omega$ son constantes a elegir convenientemente. Reemplazando \eqref{lane5solgeneral} en \eqref{transkelvin}, obtendremos una ecuación  para $z$, más simple de resolver:
\begin{align}\label{lane5eqgen}
 \xi^4\left[a\xi^{\omega}\frac{d^2 z}{d\xi^2}+2a\omega\xi^{\omega-1}\frac{dz}{d\xi}+a\omega(\omega-1)\xi^{\omega-2}z\right]&=-a^{n}\xi^{n\omega}z^n
 \end{align}
 Para simplicar esta ecuación elegimos las constantes $a$ y $\omega$ de modo que
\begin{align}
\omega(\omega-1)&=-a^{n-1},\label{lane5a}\\
\omega+2&=n\omega,
\end{align}
de donde obtenemos:
\begin{align}
\omega&=\frac{2}{n-1} \label{lane5omega},\\
a&=\left[2\frac{(n-3)}{(n-1)^2}\right]^{1/(n-1)}\label{lane5a2}.
\end{align}
Con esto, \eqref{lane5eqgen} se reduce a 
\begin{align}
\xi^2\frac{d^2 z}{d\xi^2}+2\omega\xi\frac{dz}{d\xi}+\omega(\omega-1)z&=-a^{n-1}z^n,
\end{align}
que es una ecuación tipo Euler-Cauchy, por lo que usando la sustitución estándar:
\begin{equation}\label{euler-sust}
 \xi=e^{t}\quad\Rightarrow\quad \frac{dt}{d\xi}=e^{-t}.
\end{equation}
La ecuación en $z$ toma la forma:
\begin{equation}
 \frac{d^2 z}{dt^2}+(2\omega-1)\frac{dz}{dt}+\omega(\omega-1)z+a^{n-1}z^n=0.
\end{equation}
Usando \eqref{lane5a} y \eqref{lane5omega} podemos escribir esta ecuación en términos de $n$, obteniendo
\begin{equation}\label{translane5}
 \frac{d^2 z}{dt^2}+\frac{5-n}{n-1}\frac{dz}{dt}+2\frac{3-n}{(n-1)^2}z(1-z^{n-1})=0.
\end{equation}
En el caso $n=5$ vemos que el segundo término de la ecuación \eqref{translane5} se anula, por lo que ésta toma la forma:
\begin{equation}\label{translane5b}
 \frac{d^2 z}{dt^2}=\frac{z}{4}(1-z^4).
\end{equation}
Para resolverla, se puede multiplicar a ambos lados por $dz/dt$, ya que,
\begin{align}
\left[\frac{d}{dt}\left(\frac{dz}{dt}\right)\right]\frac{dz}{dt}&=\frac{z}{4}(1-z^4)\frac{dz}{dt},\\
\Rightarrow\quad \frac{1}{2}\frac{d}{dt}\left(\frac{dz}{dt}\right)^2&=\frac{z}{4}(1-z^4)\frac{dz}{dt}.
\end{align}
Integrando con respecto a $t$ y denotando a la constante de integración como $D$, obtenemos:
\begin{equation}
\left(\frac{dz}{dt}\right)^2=\frac{z^2}{4}-\frac{z^6}{12}+2D.
\end{equation}
Notemos que si $z\to\pm\infty$, entonces $(dz/dt)^2\to-\infty$, lo que se contradice con el hecho que $dz/dt\in \mathbb{R}$, y así se deduce que $z$ debe estar acotado. Ahora, la solución a la ecuación anterior se reduce al problema de hallar la integral de:
\begin{equation}
 \frac{dz}{\left(2D+\frac{1}{4}z^2-\frac{1}{12}z^6\right)^{1/2}}=\pm dt
\end{equation}
Si $D\neq0$, la integración es complicada al involucrar integrales elípticas. Pero en el caso estudiado, sólo interesa $D=0$ (?), simplificándose notablemente el problema anterior a:
\begin{equation}\label{lane5integral}
 \int\frac{dz}{z\left(1-\frac{1}{3}z^4\right)^{1/2}}=\pm\int\frac{1}{2}dt,
\end{equation}
integral soluble analíticamente en forma simple. En efecto, mediante la sustitución trigonométrica:
\begin{equation}\label{lane5sustitucion}
\frac{1}{3}z^4=\sin^2 \zeta,
\end{equation}
encontramos que \eqref{lane5integral} equivale a:
\begin{equation}
\int\cosec\zeta d\zeta =\ln\left[\tan\left(\frac{\zeta}{2}\right)\right]=\pm t+C',
\end{equation}
es decir, considerando $C'$ como la constante de integración y $C=e^{C'}$,
\begin{equation}
 \tan\left(\frac{\zeta}{2}\right)=C e^{\pm t}
\end{equation}
Entonces, volviendo a la variable original $z$, dada por \eqref{lane5sustitucion} y teniendo presente que:
\begin{equation}
\frac{z^4}{3}= \sin^2 \zeta=\frac{1}{1+\dfrac{1}
{\tan^2\zeta}}=\frac{1}{1+\left(\dfrac{1-\tan^2(\zeta/2)}
{2\tan(\zeta/2)}\right)^2}=\frac{4\tan^2(\zeta/2)}{\left(1+\tan^2(\zeta/2)\right)^2},
\end{equation}
podemos ver que la solución a \eqref{translane5b} es:
\begin{equation}
 z=\pm\left[\frac{12C^2e^{\mp 2t}}{(1+C^2e^{\mp 2t})^2}\right]^{1/4}
\end{equation}
Expresando la solución anterior en términos de $\xi$ mediante \eqref{euler-sust},
\begin{equation}
 z=\pm\left[\frac{12C^2\xi^{\mp2}}{(1+C^2\xi^{\mp2})^2}\right]^{1/4},
\end{equation}
y recordando que tenemos la restricción física $\Theta>0$, podemos reemplazar la relación hallada para $z$ con signo $+$, en la solución propuesta de la ecuación de Lane-Emden \eqref{lane5solgeneral} en términos de la variable $\xi$, y así obtener
\begin{equation}
 \Theta=a\xi^{\omega}\left[\frac{12C^2\xi^{\mp2}}{(1+C^2\xi^{\mp2})^2}\right]^{1/4}
\end{equation}
Reemplazando $n=5$ en las relaciones para $\omega$ \eqref{lane5omega} y $a$ \eqref{lane5a2}, encontramos que:
\begin{equation}
\Theta=\left(\frac{1}{4}\right)^{1/4}\xi^{2/4} \left[\frac{12C^2\xi^{\mp2}}{(1+C^2\xi^{\mp2})^2}\right]^{1/4}=\left[ \frac{3C^2}{(1+C^2\xi^{-2})^2}\right]^{1/4}
\end{equation}
y retornando a la variable original $x$ dada por \eqref{lane5cambio}:
\begin{equation}
\theta=\left(\frac{3C^2}{(1+C^2x^2)^2}\right)^{1/4}
\end{equation}
Finalmente, aplicando la condición de borde \eqref{bcs_laneemden1}:
\begin{equation}
 \Theta(0)=1=(3C^2)^{1/4}\quad\Rightarrow C^2=\frac{1}{3},
\end{equation}
de modo que la solución a la ecuación de Lane-Emden para $n=5$ ó $\gamma=6/5$ está dada por
\begin{equation}\label{lane5}
\boxed{\Theta_5=\frac{1}{\left(1+\frac{1}{3}x^2\right)^{1/2}}}
\end{equation}
Podemos notar que, si bien esta solución es monótonamente decreciente, tiende a $0$ conforme $x\to\infty$. Esto implica que la primera raíz de esta función se puede considerar como $x_1=\infty$.

\subsection{Soluciones numéricas de la ecuación de Lane-Emden}\label{sec:lane-numerico}

La ecuación de Lane-Emden \eqref{laneemden} para un $n$ dado puede ser resuelta  numéricamente mediante el conocido algoritmo de Runge-Kutta de 4${}^{\circ}$ orden. Para ello, primero se debe reducir la ecuación original a un sistema acoplado de ecuaciones diferenciales de primer orden, lo que se logra definiendo las variables:
\begin{equation}
Y_1:=\Theta(x)\qquad\text{y}\qquad Y_2:=\frac{d\Theta(x)}{dx}.
\end{equation}
Luego, notando que la ecuación de Lane-Emden se puede escribir en la forma
\begin{align}
 \frac{1}{x^2}\left(x^2\frac{d^2\Theta(x)}{dx^2}+2x\frac{d\Theta(x)}{dx}\right)&=-\Theta(x)^n,\\
\Rightarrow\quad \Theta''(x)&=-\left(\frac{2}{x}\Theta'(x)+\Theta(x)^n\right),
\end{align}
tenemos que el sistema de ecuaciones buscado es:
\begin{equation}
\boxed{
\begin{aligned}
 Y_1'&=Y_2,\\
Y_2'&=-\left(\frac{2}{x}Y_2+Y_1^n\right),
\end{aligned}}
\end{equation}
sujeto a las condiciones iniciales, debido a \eqref{bcs_laneemden1} y \eqref{bcs_laneemden2}:
\begin{align}
 Y_1(0)=1\qquad\text{y}\qquad Y_2(0)=0.
\end{align}

Este sistema de ecuaciones se resuelve entonces para $Y_1$ e $Y_2$ mediante el método se\~nalado, escogiendo un tama\~no de paso apropiado, tal como $\Delta x=1\cdot10^{-3}$. Como $Y_1$ es decreciente (ver \eqref{theta-decreciente}), eventualmente llegará hasta un punto (para $\gamma>6/5$) en donde se anule y posteriormente se vuelva negativo. Pero tenemos la restricción física que $Y_1>0$, debido a que está relacionado con la densidad de la estrella mediante \eqref{theta}, por lo que la integración se debe detener en el punto $x=x_1$ donde $Y_1(x_1)=0$. En dicho punto, el método de Runge-Kutta también proporcionará el valor $Y_2$, con el cual se podrá determinar la cantidad característica que aparece en las relaciones de masa estelar, $x_1^2\left|\Theta'(x_1)\right|=x_1^2\left|Y_2(x_1)\right|$. De esta forma, tendremos las dos cantidades que nos permiten determinar el radio, masa y densidad central de la estrella por cada valor del índice $n$ de Lane-Emden.

En el gráfico \ref{graficolane-emden}, se representan los resultados de la integración numérica: las funciones de Lane-Emden $\Theta(x)=Y_1$ para los tres casos analíticamente solubles (ver \ref{sec:exactas}), además de los casos $\gamma=4/3\Leftrightarrow n=3$ y $\gamma=5/3\Leftrightarrow n=3/2$, que son relevantes como casos límite de la ecuación de estructura de Fermi exacta (ver sección \ref{sec:ec-fermi-limites}).
\begin{figure}[H]
\centering
\includegraphics[angle=0,width=0.8\textwidth]{fig/fig-Lane-Emden.pdf}
\caption{Algunas soluciones de la ecuación de Lane-Emden, con $n=1/(\gamma-1)$. Código Python \href{https://github.com/gfrubi/GR/blob/master/figuras-editables/fig-Lane_Emden.py}{aquí} (G. Neumann).}
\label{graficolane-emden}
\end{figure}

En la tabla \ref{tablalaneemden} se resumen los valores\footnote{Estos valores fueron calculados por el algoritmo de integración usado y corresponden con todos los decimales (a excepción del último, que puede atribuirse a la aproximación o truncamiento utilizado) a los valores dados en el texto de Weinberg \cite{Weinberg72}.} de las raíces $x_1$ de la función de Lane-Emden $\Theta(x)$, además de las cantidades características $x_1^2\left|\Theta'(x_1)\right|$, para distintos valores del índice $\gamma$ ó $n$. Estos son los valores que se usarán en todas las secciones posteriores que los requieran para la obtención de propiedades físicas estelares.

\begin{table}[H]
\begin{center}
\caption{Tabla de Raíces de la ecuación de Lane-Emden}\label{tablalaneemden}
\vspace{2mm}
\begin{tabular}{|c|c|c|c|}\hline
$n$&$\gamma$ &$x_1$&$x_1^2\left|\Theta'(x_1)\right|$\\ \hline
5&6/5&$\infty$&1.73205\\
9/2&11/9&31.83646&1.73780\\
4&5/4&14.97155&1.79723\\
7/2&9/7&9.53581&1.89056\\
3&4/3&6.89685&2.01824\\
5/2&7/5&5.35528&2.187(20)\\
2&3/2&4.35287&2.41105\\
3/2&5/3&3.65375&2.71406\\
1&2&$\pi$&$\pi$\\
1/2&3&2.75280&3.78710\\
0&$\infty$&$\sqrt{6}$&2$\sqrt{6}$\\ \hline
\end{tabular}
\end{center}
\end{table}


\subsection{Comportamiento físico de algunas soluciones particulares}\label{sec:casos-lane-emden}

De las ecuaciones anteriores para la masa y radio de las estrellas surgen varias consecuencias importantes para el comportamiento de un modelo estelar politrópico con determinados valores del índice politrópico $\gamma$ ó $n$, los que se detallarán a continuación:

\begin{itemize}

\item Para $\gamma\to\infty$ ó $n=0$, que corresponde a materia incompresible, la solución encontrada analíticamente $\Theta_0(x)$ en \eqref{lane0} coincide con la encontrada explícitamente en \eqref{presionnewton}, luego de retornar a las variables físicas por medio de las transformaciones \eqref{theta} y \eqref{x}. En efecto, considerando además la ecuación de estado \eqref{estadopolitropica}, \eqref{lane0} equivale a (sin consider aún el límite $\gamma\to\infty$):
\begin{align}
 \left(\frac{\rho}{\rho_{\rm c}}\right)^{\gamma-1}=\left(\left(\frac{P}{P_0}\right)^{1/\gamma}\right)^{\gamma-1}&=1-\frac{1}{6}\left(\frac{4\pi G}{K}\left(1-\frac{1}{\gamma}\right)\right)\rho_{\rm c}^{2-\gamma}r^2,\\
\frac{P}{P_0}\left(\frac{P}{P_0}\right)^{1/\gamma}&=1-\frac{1}{6}\left(4\pi G\left(1-\frac{1}{\gamma}\right)\right)\frac{1}{K}\rho_{\rm c}^{2-\gamma}r^2,\\
P\left(\frac{P}{P_0}\right)^{1/\gamma}&=P_0-\frac{2}{3}\pi G\left(1-\frac{1}{\gamma}\right)\cancelto{1}{\frac{P_0}{K\rho_{\rm c}^{\gamma}}}\rho_{\rm c}^{2}r^2,
\end{align}
en donde se ha usado nuevamente \eqref{estadopolitropica} para simplificar el factor del lado derecho. Ahora,  para $\gamma\to\infty$, tendremos que $1/\gamma\to 0 $, por lo que $(P/P_0)^{1/\gamma}\to 1$ y así recuperamos la ecuación \eqref{presionnewton} que corresponde al caso analizado de materia incompresible:
\begin{equation}
 P(r)=P_0-\frac{2}{3}\pi G\rho^2r^2
\end{equation}


Por otra parte, el hecho que la raíz $x_1$ sea la menor de todas las funciones $\Theta_n(x)$ para un $n$ dado, implica físicamente que una estrella compuesta de materia incompresible ($n=0$) es la que posee el menor radio de entre todas las estrellas politrópicas con la misma presión (ó densidad) central. Este radio estará dado por la expresión general \eqref{radiopolitropico} y la ecuación de estado \eqref{estadopolitropica}:
\begin{align}
 R&=\lim_{\gamma\to\infty}\left(\frac{K\gamma}{4\pi G(\gamma-1)}\right)^{1/2}\rho_{\rm c}^{\frac{\gamma}{2}-1}x_1\\
 &=\left(\frac{1}{4\pi G}\right)^{1/2}\frac{x_1}{\rho_{\rm c}}\lim_{\gamma\to\infty}\cancelto{1}{\left(\frac{\gamma}{\gamma-1}\right)^{1/2}}(K\rho_{\rm c}^\gamma)^{1/2}\\
 &=\left(\frac{1}{4\pi G}\right)^{1/2}\frac{P_0^{1/2}}{\rho_{\rm c}}x_1.
\end{align}
Pero, como $x_1=\sqrt{6}$, tendremos que el radio será conocido sólo si se especifica la razón $P_0/\rho_{\rm c}^2$, ya que éste puede expresarse como:
\begin{equation}\label{rlane0}
 R=\sqrt{\frac{3}{2\pi G}\frac{P_0}{\rho_{\rm c}^2}}.
\end{equation}
% Esto puede entenderse de la manera siguiente: Si la ecuación de estado politrópica \eqref{estadopolitropica} es válida con $\gamma$ finito, entonces el radio $R$ será expresable (por \eqref{radiopolitropico}) como función de $R=R(\gamma,K,\rho_{\rm c})$. Pero en este caso, no existe un $K$ definido, por lo que se requiere expresar el radio en función de otra variable, como lo es la presión central en la estrella, de modo tal que $R=R(\gamma,P_0,\rho_{\rm c})=R(\gamma,P_0/\rho_{\rm c}^2)=R(P_0/\rho_{\rm c}^2)$.
Cabe destacar que la relación \eqref{rlane0} también puede obtenerse imponiendo la condición que $P(R)=0$ sobre la ecuación \eqref{presionnewton}, conduciendo al mismo resultado antes hallado \eqref{ctep}.

Finalmente, la masa y la relación masa-radio para este caso de materia incompresible, puede obtenerse fácilmente, en vez de usar las ecuaciones \eqref{masalaneemden} y \eqref{masaradiopolitropico}, mediante la definición de densidad y la relación antes hallada \eqref{rlane0}:
\begin{align}
 M=\rho V&=\frac{4}{3}\pi R^3 \rho_{\rm c}\\
 &=\frac{4}{3}\pi\rho_{\rm c} \left[\frac{3}{2\pi G}\frac{P_0}{\rho_{\rm c}^2}\right]^{3/2}\\
 &=\sqrt{\frac{6}{\pi}}\rho_{\rm c} \left(\frac{P_0}{G\rho_{\rm c}^2}\right)^{3/2}.
\end{align}


\item Para $\gamma=2$ ó $n=1$, se obtuvo la solución analítica $\Theta_1(x)$ en \eqref{lane1}, cuya primera raíz es $x_1=\pi$. Reemplazando en el radio de la estrella \eqref{radiopolitropico}, se observa de inmediato que éste será independiente de la densidad central, pues
 \begin{equation}
  R=\left(\frac{2K}{4\pi G(2-1)}\right)^{1/2}\rho_{\rm c}^{\frac{2}{2}-1}\pi.
 \end{equation}
Así, tendremos el radio para una estrella politrópica con $\gamma=2$:
\begin{equation}
 R=\sqrt{\frac{\pi K}{2G}},
\end{equation}
que depende únicamente de la constante $K$ de la ecuación de estado politrópica.

\item  Para $\gamma=6/5$ ó $n=5$, también se obtuvo una solución analítica $\Theta_5(x)$ en \eqref{lane5}, en donde la primera raíz sólo se alcanza en $x_1\to\infty$. Por \eqref{radiopolitropico}, una ``estrella''\, con este índice adiabático tendría un radio infinito, para cualquier valor finito de la densidad central $\rho_{\rm c}$. Por otra parte, la masa de estas soluciones estaría dada por \eqref{masalaneemden}, tomando el límite cuando $x_1\to\infty$:
\begin{align}
 M=4\pi\rho_{\rm c}^{\frac{3\cdot6/5-4}{2}}\left(\frac{K\cdot6/5}{4\pi G(6/5-1)}\right)^{3/2}\lim_{x_1\to\infty}x_1^2\left|\Theta'_5(x_1)\right|.
\end{align}
El límite anterior se puede calcular en base a \eqref{lane5}:
\begin{align}
 \lim_{x_1\to\infty}x_1^2\left|\Theta'_5(x_1)\right|&=\lim_{x_1\to\infty}x_1^2\left|\frac{d}{dx}\left.\left(\frac{1}{1+\frac{x^2}{3}}\right)^{1/2}\right|_{x=x_1}\right|\\
 &=\lim_{x_1\to\infty}\frac{1}{3}\frac{x_1^3}{\left(1+\frac{x_1^2}{3}\right)^{3/2}}\\
 &=\lim_{x_1\to\infty}3^{1/2}\frac{1}{\left(\frac{3}{x_1^2}+1\right)^{3/2}}\\
 &=\sqrt{3}
\end{align}
De esta forma, podemos ver que la masa de una solución con $\gamma=6/5$ es finita, aunque su radio sea infinito, estando dada por:
\begin{equation}
 M=36\pi\left(\frac{K}{4\pi G}\right)^{3/2}\rho_{\rm c}^{-1/5}.
\end{equation}


 \item De la ecuación de masa \eqref{masalaneemden} ó \eqref{lane-masagamma}, se puede ver que la masa de la estrella $M$ es creciente con respecto su densidad central $\rho_{\rm c}$ para $\gamma>4/3$ ($n<3$), mientras que para $\gamma<4/3$ ($n>3$) es decreciente. De forma análoga, considerando ahora la relación masa-radio \eqref{masaradiopolitropico}, podemos ver que el mismo comportamiento existe con respecto al radio: la masa de la estrella $M$ es creciente con respecto a su radio $R$ para $\gamma>4/3$ ($n<3$), mientras que para $\gamma<4/3$ ($n>3$) es decreciente. Estos hechos tienen importantes consecuencias para la estabilidad estelar.

 \item Para $\gamma=4/3$ ó $n=3$, de acuerdo a la discusión del ítem anterior, la masa de la estrella será independiente de la densidad central y del radio, estando dada por \eqref{masalaneemden}:
\begin{align}\label{masa4/3}
 M=4\pi\left(\frac{K}{\pi G}\right)^{3/2}x_1^2\left|\Theta'(x_1)\right|
 =4\cdot2.01824\cdot\pi\left(\frac{K}{\pi G}\right)^{3/2}=:M_{\rm ch},
\end{align}
en donde se ha usado el valor de $x_1^2\left|\Theta'(x_1)\right|$ obtenido numéricamente de la función de Lane-Emden $\Theta_3(x)$, recopilado en la tabla \ref{tablalaneemden}. Esta cantidad se denomina \emph{Masa de Chandrasekhar} $M_{\rm ch}$, y representa un valor límite que posee una importante interpretación física para ciertos tipos de estrellas, como se mostrará posteriormente.
 Por completitud, también se dará la expresión explícita para el radio asociado a este índice politrópico usando \eqref{radiopolitropico}, aunque más adelante se verá que esta aproximación carece de valor práctico.
\begin{align}\label{radio4/3}
 R=\left(\frac{K}{\pi G}\right)^{1/2}\rho_{\rm c}^{-1/3}x_1
 =6.89685\left(\frac{K}{\pi G}\right)^{1/2}\rho_{\rm c}^{-1/3},
\end{align}
en donde se ha usado el valor de $x_1$ hallado en la tabla \ref{tablalaneemden}.

\item Para $\gamma=5/3$ ó $n=3/2$, que corresponde al índice politrópico de un gas ideal (ver justificación en el apéndice \ref{cap:termo}), podemos ver que su radio se obtiene de \eqref{radiopolitropico} simplemente reemplazando este valor y usando los valores de $x_1$ de la tabla \ref{tablalaneemden}:
\begin{equation}\label{radio5/3}
 R=\left(\frac{5K}{8\pi G}\right)^{1/2}\rho_{\rm c}^{-1/6}x_1=3.65375\left(\frac{5K}{8\pi G}\right)^{1/2}\rho_{\rm c}^{-1/6}.
\end{equation}
Así mismo, podemos encontrar una expresión para la masa a partir de \eqref{masalaneemden} y de la tabla \ref{tablalaneemden}:
\begin{equation}\label{masa5/3}
 M=4\pi\rho_{\rm c}^{1/2}\left(\frac{5K}{8\pi G}\right)^{3/2}x_1^2\left|\Theta'(x_1)\right|=4\cdot2.71406\cdot\pi\rho_{\rm c}^{1/2}\left(\frac{5K}{8\pi G}\right)^{3/2}.
\end{equation}
Finalmente, usando la ecuación masa-radio \eqref{masaradiopolitropico}, podemos ver que:
\begin{align}
M&=4\pi R^{-3}\left(\frac{5K}{8\pi G}\right)^3x_1^3\left\{x_1^2\left|\Theta'(x_1)\right|\right\}\label{masaradio5/3},\\
\Rightarrow\quad M\left[\frac{4\pi}{3}R^3\right]&=\frac{1}{3}\frac{1}{4\pi}\left(\frac{5K}{2G}\right)x_1^3\left\{x_1^2\left|\Theta'(x_1)\right|\right\}.
\end{align}
Identificando el factor en paréntesis cuadrado en el lado izquierdo de la igualdad anterior como el volumen $V$ de la estrella, encontramos que el producto de la masa y volumen para este modelo estelar es constante, estando dado por
\begin{equation}
 MV=\frac{1}{3\cdot 2^5\pi}\left(\frac{5K}{G}\right)^3 x_1^ 3\left\{x_1^2\left|\Theta'(x_1)\right|\right\}=\frac{132.384}{3\cdot 2^5\pi}\left(\frac{5K}{G}\right)^3\label{masavolumen5/3}.
\end{equation}

\end{itemize}


\subsubsection{Determinación de parámetros estelares en función de \texorpdfstring{$M$}{M}, \texorpdfstring{$R$}{R} y \texorpdfstring{$\gamma$}{gamma}}

Para poder constrastar los resultados obtenidos mediante el modelo estelar newtoniano descrito, con respecto a las cantidades físicas directamente provistas por las observaciones, se requiere reescribir las relaciones anteriores y sus soluciones de la forma detallada a continuación. Se supondrá que los parámetros conocidos de una estrella son su \emph{masa} $M$, su \emph{radio} $R$ y el \emph{índice politrópico} $\gamma$ del gas que la compone. Luego, el procedimiento para modelar dicha estrella será:

\begin{enumerate}
 \item Como $\gamma$ es dado, $n$ será conocido mediante \eqref{ngamma}. Así, por cada $n\in[0,5]$, la ecuación de Lane-Emden \eqref{laneemden} tendrá una solución única $\Theta_n(x)$, que en general se encuentra numéricamente. El conocimiento de la función de Lane-Emden anterior también proporciona su primera raíz positiva $x_1$ (en donde $\Theta(x_1)=0$), y además $x_1^2\left|\Theta'(x_1)\right|$.
\item La constante $K$ de la ecuación de estado politrópica \eqref{estadopolitropica} queda completamente determinada. En efecto, de la relación \eqref{masaradiopolitropico}, escrita en términos de $n$, es posible despejar $K$ en términos de las cantidades conocidas $M$, $R$, $x_1$ y $x_1^2\left|\Theta'(x_1)\right|$:
\begin{align}
M&=4\pi R^{\frac{n-3}{n-1}}\left(\frac{K(1+n)}{4\pi G}\right)^{\frac{n}{n-1}}x_1^{-\frac{n-3}{n-1}}\,x_1^2\left|\Theta'(x_1)\right|,\\
% M^{\frac{n-1}{n}}&=(4\pi)^{\frac{n-1}{n}}R^{\frac{n-3}{n}}\frac{K}{4\pi G}(1+n)x_1^{\frac{n+1}{n-1}\frac{n-1}{n}}\left|\Theta'(x_1)\right|^{\frac{n-1}{n}}\\
% GM^{\frac{n-1}{n}}R^{\frac{3-n}{n}}&=K\frac{(n+1)}{(4\pi)^{1/n}}x_1^{\frac{n+1}{n}}\left|\Theta'(x_1)\right|^{\frac{n-1}{n}}\\
% \Rightarrow\quad K&=\frac{1}{n+1}\left(\frac{4\pi}{x_1^{n+1}\left|\Theta'(x_1)\right|^{n-1}}\right)^{1/n}GM^{\frac{n-1}{n}}R^{\frac{3-n}{n}}
% \end{align}
% De esta forma, tendremos finalmente que:
% \begin{equation}
\Rightarrow\quad K&=\frac{1}{n+1}\left(\frac{4\pi}{x_1^{3-n} \left(x_1^2\left|\Theta'(x_1)\right|\right)^{n-1}}\right)^{1/n}GM^{1-\frac{1}{n}}R^{\frac{3}{n}-1}.\label{kpolitropico}
\end{align}

 \item La densidad central $\rho_{\rm c}$ presente en la normalización que condujo a la definición de la variable $\Theta(x)$, ya se determinó en términos de las variables conocidas en \eqref{densidad0politropica}.
% en se puede obtener fácilmente reemplazando el resultado anterior \eqref{kpolitropico} en \eqref{radiopolitropico} (escrita en términos de $n$):
% \begin{align}
% R&=\left(\frac{K(1+n)}{4\pi G}\right)^{1/2}\rho_{\rm c}^{\frac{1-n}{2n}}x_1\\
% R^{\frac{2n}{1-n}}&=\left(\frac{K(1+n)}{4\pi G}\right)^{\frac{n}{1-n}}x_1^{\frac{2n}{1-n}}\rho_{\rm c}\\
% R^{\frac{2n}{1-n}}&=\left(\frac{1+n}{4\pi G}\right)^{\frac{n}{1-n}}\left[\frac{1}{n+1}\left(\frac{4\pi}{x_1^{3-n} \left(x_1^2\left|\Theta'(x_1)\right|\right)^{n-1}}\right)^{\frac{1}{n}}GM^{1-\frac{1}{n}}R^{\frac{3}{n}-1}\right]^{\frac{n}{1-n}}x_1^{\frac{2n}{1-n}}\rho_{\rm c}\\
% R^{\frac{2n}{1-n}}&=4\pi\frac{x_1^2\left|\Theta'(x_1)\right|}{x_1^3}\frac{R^{\frac{3-n}{1-n}}}{M}\rho_{\rm c}
% \end{align}
% Es decir, obtenemos la densidad central en términos de cantidades conocidas:

\item La densidad $\rho(r)$ entonces queda completamente determinada al expresarla en función de la función de Lane-Emden $\Theta(x)$ por medio de \eqref{theta}, pues usando $\rho_{\rm c}$ dado en \eqref{densidad0politropica} podemos escribir:
\begin{align}
 \rho(r)&=\rho_{\rm c}\left[\Theta(x)\right]^n\label{rhoder}\\
&=\frac{1}{4\pi}\frac{x_1^3}{x_1^2\left|\Theta'(x_1)\right|}\frac{M}{R^3}\left[\Theta(x)\right]^n,
\end{align}
y considerando \eqref{laneemden-rax}:
\begin{equation}\label{densidadpolitropica}
 \rho(r)=\frac{1}{4\pi}\frac{x_1^3}{x_1^2\left|\Theta'(x_1)\right|}\frac{M}{R^3}\left[\Theta\left(x_1\frac{r}{R}\right)\right]^n.
\end{equation}

\item Usando la ecuación de estado \eqref{estadopolitropica}, podemos determinar directamente también la presión en el centro de la estrella $P_0$, pues evalúandola allí:
\begin{equation}
 P_0=K\rho_{\rm c}^{1+\frac{1}{n}}.
\end{equation}
Así, reemplazando \eqref{kpolitropico} y \eqref{densidad0politropica}, tenemos:
\begin{align}
%  P_0&=\frac{1}{n+1}\left(\frac{4\pi}{x_1^{3-n} \left(x_1^2\left|\Theta'(x_1)\right|\right)^{n-1}}\right)^{1/n}GM^{1-\frac{1}{n}}R^{\frac{3}{n}-1}\left[\frac{1}{4\pi}\frac{x_1^3}{x_1^2\left|\Theta'(x_1)\right|}\frac{M}{R^3}\right]^{1+\frac{1}{n}}\\
P_0&=\frac{1}{4\pi(n+1)}\frac{x_1^4}{\left(x_1^2\left|\Theta'(x_1)\right|\right)^2}\frac{GM^2}{R^4}\label{presionpolitropica}.
\end{align}
\item Análogamente, también podemos determinar la dependencia radial de la presión al interior de la estrella, usando \eqref{estadopolitropica} y \eqref{rhoder}:
\begin{align}
 P(r)=K\rho(r)^{1+\frac{1}{n}}=K\left(\rho_{\rm c}\left[\Theta(x)\right]^n\right)^{1+\frac{1}{n}}=P_0\left[\Theta(x)\right]^{1+n}.
\end{align}
Entonces, reemplazando \eqref{presionpolitropica}, tenemos que la forma explícita del perfil de presión en la estrella vendrá dado por:
\begin{equation}
 P(r)=\frac{1}{4\pi(n+1)}\frac{x_1^4}{\left(x_1^2\left|\Theta'(x_1)\right|\right)^2}\frac{GM^2}{R^4}\left[\Theta\left( x_1\frac{r}{R}\right)\right]^{1+n}.
\end{equation}






\end{enumerate}

\section{Solución exacta para gas de Fermi}\label{sec:fermi-exacta}

Un caso mucho más realista que los tratados anteriormente para resolver el sistema de ecuaciones de estructura estelar, es considerar la ecuación de estado exacta de un gas de Fermi completamente degenerado (a $T=0$), dada en forma implícita por las ecuaciones \eqref{presionfermi2} y \eqref{densidad_electrones-fermi}, que se pueden expresar respectivamente como:
\begin{align}\label{ec-estado-fermi}
 P&=Af(x),&\rho&=Bx^3,
\end{align}
en donde $x$ es un parámetro adimensional expresado en función del momentum de Fermi por \eqref{xrelativo}, en términos del cual se define:
\begin{equation}\label{ec-fermi-fx}
 f(x)=x\sqrt{1+x^2}\left(2x^2-3\right)+3\sinh^{-1}x.
\end{equation}
Además, $A$ es una constante dada por:
\begin{align}\label{ec-fermi-A}
 A_{e,n}&=\frac{\pi m_{e,n}^4 c^5}{3 h^3},
\end{align}
tanto para electrones ($e$) como neutrones ($n$), mientras que $B$ es otra constante que para electrones y neutrones toma respectivamente los valores:
\begin{align}\label{ec-fermi-B}
B_e=\frac{8\pi m_e^3 c^3}{3h^3}m_u\mu_e,\qquad\qquad B_n=\frac{8\pi m_n^4 c^3}{3h^3}&
\end{align}

\subsection{Ecuación de estructura de Fermi}
En principio tenemos las tres incógnitas, $P$, $\rho$ y $M$, por determinar, la última de las cuales se puede despejar siguiendo el procedimiento que condujo a la ecuación diferencial \eqref{eq-hidrostatico-sinm} de $P$ y $\rho$ en función de la coordenada radial $r$. Sustituyendo la ecuación de estado \eqref{ec-estado-fermi}, tenemos:
\begin{align}
\frac{1}{Br^2}\frac{d}{dr}\left(\frac{r^2}{x^3}\frac{dP}{dx}\frac{dx}{dr}\right)&=-4\pi G B x^3.
\end{align}
Notando que la derivada de la presión con respecto a $x$ se puede determinar directamente del integrando de \eqref{presionfermi1}, obtenemos:
\begin{align}
\frac{1}{Br^2}\frac{d}{dr}\left(\frac{r^2}{x^3}\frac{8Ax^4}{\sqrt{1+x^2}}\frac{dx}{dr}\right)&=-4\pi G B x^3,\\
\frac{1}{r^2}\frac{d}{dr}\left(r^2\frac{x}{\sqrt{1+x^2}}\frac{dx}{dr}\right)&=-\frac{\pi G B^2}{2A} x^3,\\
\frac{1}{r^2}\frac{d}{dr}\left(r^2\frac{d}{dr}\sqrt{1+x^2}\right)&=-\frac{\pi G B^2}{2A} x^3.
\end{align}
Haciendo la sustitución:
\begin{align}\label{sust-fermi}
 y:=\sqrt{1+x^2},\qquad\Rightarrow\qquad x=\sqrt{y^2-1},
\end{align}
tenemos que:
\begin{align}\label{ec-fermi1}
\frac{1}{r^2}\frac{d}{dr}\left(r^2\frac{dy}{dr}\right)&=-\frac{\pi G B^2}{2A} \left(y^2-1\right)^{3/2},
\end{align}
ecuación análoga a la de Lane-Emden \eqref{laneemden}, por lo que podemos usar técnicas similares para encontrar sus soluciones. En primer lugar, se escribe en términos de la coordenada radial adimensional $\eta$ y de la variable reescalada $\phi$, definidas por:
\begin{equation}\label{cambio-fermi-adim}
 r=a\eta,\qquad\text{y}\qquad y=y_0\phi,
\end{equation}
con $y_0$ constante, reminiscente de la densidad central de la estrella $\rho_{\rm c}$ (comparar con el cambio de variables de Lane-Emden \eqref{x} y \eqref{theta}). Para encontrar $a$, reemplazamos en \eqref{ec-fermi1}:
\begin{align}
\frac{1}{(a\eta)^2}\frac{d}{d(a\eta)}\left((a\eta)^2\frac{d(y_0\phi)}{d(a\eta)}\right)&=-\frac{\pi G B^2}{2A} \left((y_0\phi)^2-1\right)^{3/2},\\
\frac{1}{\eta^2}\frac{d}{d\eta}\left(\eta^2\frac{d\phi}{d\eta}\right)&=-a^2\frac{\pi G B^2 y_0^2}{2A}\left(\phi^2-\frac{1}{y_0^2}\right)^{3/2}.
\end{align}
Así, definiendo:
\begin{equation}\label{ec-fermi-a}
 a:=\sqrt{\frac{2A}{\pi G}}\frac{1}{B y_0}=\frac{l}{y_0},
\end{equation}
con $l$ una escala de longitud independiente de $y_0$, obtenemos \emph{la ecuación de estructura de Fermi}, dependiente del parámetro $y_0$ (comparar con \eqref{laneemden} que depende, en cambio, del parámetro $n$ ó $\gamma$):
\begin{align}\label{ec-fermi2}
\boxed{
\frac{1}{\eta^2}\frac{d}{d\eta}\left(\eta^2\frac{d\phi}{d\eta}\right)=-\left(\phi^2-\frac{1}{y_0^2}\right)^{3/2}.}
\end{align}
Las condiciones de borde son las mismas de Lane-Emden para la variable $\Theta$, ya que en el centro de la estrella $y(\eta=0)=y_0$, y así:
\begin{equation}\label{bcs-fermi}
\phi(\eta=0)=1\qquad\text{y}\qquad \phi'(\eta=0)=0,
\end{equation}
en donde la segunda condición se justifica de la misma forma que en \eqref{divergencia2}.

\subsection{Propiedades físicas de la ecuación de estructura de Fermi}

\subsubsection{Radio de la estrella}
Al igual que en el caso de la ecuación de Lane-Emden, tenemos que las soluciones $\phi=\phi(\eta)$ son monótonamente decrecientes. Como dicha variable está relacionada con $y$  de acuerdo a \eqref{cambio-fermi-adim}, ésta a su vez con $x$ a través de \eqref{sust-fermi}, que a su vez es parte de la ecuación de estado para la densidad $\rho$ \eqref{ec-estado-fermi}, vemos que:
\begin{equation}\label{rho-x-y-phi}
\begin{split}
 \rho=Bx^3=B\left(y^2-1\right)^{3/2}&=B\left[(y_0\phi)^2-1\right]^{3/2}\\
&=By_0^3\left(\phi^2-\frac{1}{y_0^2}\right)^{3/2}.
\end{split}
\end{equation}
Luego, es claro que la densidad también será monótonamente decreciente. Por lo tanto, se definirá el radio de la estrella, al igual que para Lane-Emden, como aquella coordenada radial adimensional $\eta_1$ en que la densidad se anule: $\rho(\eta_1)=0$. Como de la ecuación anterior esto equivale a que $x(\eta_1)=0$, vemos de \eqref{ec-estado-fermi} y \eqref{ec-fermi-fx} que la presión de la estrella también se anulará allí: $P(x=0)=0$, lo cual es consistente con lo que uno esperaría físicamente para el borde de una estrella. Ahora bien, esta condición implica, por \eqref{rho-x-y-phi}, que:
\begin{equation}
%  0=\rho(\eta_1)=B\left((y_0\phi(\eta_1))^2-1\right)^{3/2}\quad\Rightarrow\quad
\phi(\eta_1)=\frac{1}{y_0}\label{ec-fermi-raices}
\end{equation}
Consistentemente, tendremos de la definición de la variable $\eta$ en \eqref{cambio-fermi-adim}, que el radio físico $R$ de la estrella vendrá dado, al considerar \eqref{ec-fermi-a}, por:
\begin{equation}\label{ec-fermi-radio}
\boxed{
\begin{aligned}
 R&=a\eta_1=l\frac{\eta_1}{y_0}\\
&=\sqrt{\frac{2A}{\pi G}}\frac{1}{B}\frac{\eta_1}{y_0}.
\end{aligned}}
\end{equation}

Al igual que para el caso de Lane-Emden, el radio adimensional $\eta_1$ se obtiene, para cada $y_0$, resolviendo la ecuación de estructura de Fermi mediante métodos numéricos, que se describirán en la sección \ref{sec:fermi-numerico}.

\subsubsection{Masa de la estrella}

La masa de la estrella al interior del radio adimensional $\eta$ se obtendrá a partir de su definición en \eqref{masa1} y de \eqref{ec-fermi-dens}:
\begin{align}
 \mathcal{M}(\eta)&=4\pi\int\limits^R_{0}dr'r'^2\rho(r')=4\pi a^3\int\limits^{\eta}_{0}d\eta' \eta'^2\rho\\
&=4\pi B y_0^3 a^3\int\limits^{\eta}_{0} \left(\phi^2-\frac{1}{y_0^2}\right)^{3/2}\eta'^2d\eta'.
\end{align}
Pero, al igual que para Lane-Emden, notamos que de la ecuación de estructura de Fermi  \eqref{ec-fermi2}:
\begin{align}
 \eta^2\left(\phi^2-\frac{1}{y_0^2}\right)^{3/2}&=-\frac{d}{d\eta}\left(\eta^2\frac{d\phi}{d\eta}\right),\\
\Rightarrow\quad \int\limits^{\eta}_{0} \left(\phi^2-\frac{1}{y_0^2}\right)^{3/2}\eta'^2d\eta'&=\eta^2\left\Vert\phi'(\eta)\right\Vert,
\end{align}
en donde se toma el valor absoluto puesto que $\phi'(\eta)<0$ (soluciones decrecientes). Luego, la masa al interior de $\eta$ se podrá expresar usando también la definición de $a$ en \eqref{ec-fermi-a}, como:
\begin{align}
\mathcal{M}(\eta)&=4\pi B y_0^3 a^3\eta^2\left\Vert\phi'(\eta)\right\Vert\label{ec-fermi-masaparcial1}\\
&=4\pi\left(\frac{2A}{\pi G}\right)^{3/2}\frac{1}{B^2}\,\eta^2\left\Vert\phi'(\eta)\right\Vert\label{ec-fermi-masaparcial2}.
\end{align}
La masa total  de la estrella será toda aquella al interior de su radio adimensional $M:=\mathcal{M}(\eta_1)$, por lo que evaluando la expresión anterior en $\eta_1$, encontramos:
\begin{equation}\label{ec-fermi-masatotal}
\boxed{
 M=4\pi\left(\frac{2A}{\pi G}\right)^{3/2}\frac{1}{B^2}\eta_1^2\left\Vert\phi'(\eta_1)\right\Vert.}
\end{equation}
Note que aquí el parámetro $y_0$ de la ecuación de estructura sólo aparece implícitamente, a través de la solución $\phi$ y la cantidad característica $\eta_1^2\left\Vert\phi'(\eta_1)\right\Vert$, que se encuentran numéricamente (ver sección \eqref{sec:fermi-numerico}).

\subsubsection{Densidad central y media}
La densidad central de la estrella $\rho_{\rm c}$ se conseguirá evaluando \eqref{rho-x-y-phi} en $\eta=0$ y aplicando la condición de borde \eqref{bcs-fermi}:
\begin{align}
 \rho_{\rm c}:&=\rho(\eta=0)=B\left[(y_0\phi(\eta=0))^2-1\right]^{3/2}\\
&=B\left(y_0^2-1\right)^{3/2}=By_0^3\left(1-\frac{1}{y_0^2}\right)^{3/2}.\label{ec-fermi-byrho1}
\end{align}
De esta forma, tendremos que la densidad central es función únicamente del parámetro $y_0$ de la ecuación de estructura de Fermi:
\begin{equation}\label{ec-fermi-byrho2}
 \boxed{\rho_{\rm c}=B(y_0^2-1)^{3/2}}
\end{equation}

Por otra parte, el perfil de densidad de la estrella se puede expresar reescribiendo \eqref{rho-x-y-phi} y usando la relación anterior:
\begin{align}
 \rho(r)&=By_0^3\left(\phi(\eta)^2-\frac{1}{y_0^2}\right)^{3/2}\label{ec-fermi-dens}\\
&=\rho_{\rm c}\frac{y_0^3}{\left(y_0^2-1\right)^{3/2}}\left(\phi(\eta)^2-\frac{1}{y_0^2}\right)^{3/2}\\
&=\rho_{\rm c}\left(\frac{\phi(\eta)^2-\frac{1}{y_0^2}}{1-\frac{1}{y_0^2}}\right)^{3/2}.
\end{align}
También podemos establecer una relación similar a la de Lane-Emden entre la densidad media $\bar{\rho}$ y la densidad central $\rho_{\rm c}$ de la estrella. Para ello, notamos que la densidad media de materia dentro del radio $\eta$ se expresará en función de la masa al interior de dicho radio por medio de \eqref{ec-fermi-masaparcial1}:
\begin{align}
 \bar{\rho}(\eta)&=\frac{\mathcal{M}(r)}{\frac{4}{3}\pi r^3},\\
&=\frac{\cancel{4\pi} B y_0^3 \cancel{a^3}\eta^2\left\Vert\phi'(\eta)\right\Vert}{\frac{\cancel{4}}{3}\cancel{\pi} (\cancel{a}\eta)^3},
\end{align}
y expresando $B$ en términos de $\rho_{\rm c}$ por medio de \eqref{ec-fermi-byrho1}, tendremos la siguiente relación entre la densidad media parcial y la central\footnote{Consistemente, es fácilmente verificable, a partir de la relación mostrada, que $\lim_{\eta\to 0}\bar{\rho}(\eta)=\rho_{\rm c}$. En efecto, usando la regla de L'H\^opital y recordando la condición de borde $\phi'(0)=0$, vemos que:
\begin{align}
\lim_{\eta\to 0}\bar{\rho}=\frac{3\rho_{\rm c}}{\left(1-\frac{1}{y_0^2}\right)^{3/2}}\lim_{\eta\to 0}\frac{\left\Vert\phi'(\eta)\right\Vert}{\eta}
=
\frac{3\rho_{\rm c}}{\left(1-\frac{1}{y_0^2}\right)^{3/2}}\lim_{\eta\to 0}\frac{\left\Vert\phi''(\eta)\right\Vert}{1}\label{ec-fermi-limite-rho}
\end{align}
Luego, despejando $\phi''(\eta)$ de la ecuación de estructura \eqref{ec-fermi2} (lo que se hará explícitamente en \eqref{ec-fermi-phi''}), podemos reemplazarlo en el límite anterior y recordando la condición de borde $\phi(0)=0$, obtenemos la relación:
\begin{align}
\lim_{\eta\to 0}\phi''(\eta)&=\lim_{\eta\to 0}\frac{\phi'(\eta)}{\eta}=\lim_{\eta\to 0}-\left[2\frac{\phi'(\eta)}{\eta}+\left(\phi(\eta)-\frac{1}{y_0^2}\right)^{3/2}\right]\\
\Rightarrow\quad\lim_{\eta\to 0}3\frac{\phi'(\eta)}{\eta}&=-\left(\phi(0)-\frac{1}{y_0^2}\right)^{3/2}\\
\Rightarrow\quad\lim_{\eta\to 0}\left\Vert\phi''(\eta)\right\Vert&=\frac{1}{3}\left(1-\frac{1}{y_0^2}\right)^{3/2}
\end{align}
Así, reemplazando el límite anterior en \eqref{ec-fermi-limite-rho}, obtenemos finalmente que:
$\lim_{\eta\to 0}\bar{\rho}(\eta)=\rho_{\rm c}.$
}:
\begin{align}
\bar{\rho}(\eta)=\frac{3\rho_{\rm c}}{\left(1-\frac{1}{y_0^2}\right)^{3/2}}\frac{\eta^2\left\Vert\phi'(\eta)\right\Vert}{\eta^3},
\end{align}
y para la densidad media total $\bar{\rho}$ de la estrella, en $\eta=\eta_1$; obtenemos:
\begin{align}\label{ec-fermi-densidadmediaycentral}
\boxed{\bar{\rho}=\frac{3\rho_{\rm c}}{\left(1-\frac{1}{y_0^2}\right)^{3/2}}\frac{\eta_1^2\left\Vert\phi'(\eta_1)\right\Vert}{\eta_1^3},}
\end{align}
es decir, la densidad media será proporcional a la densidad central, pues los otros factores dependen sólo de la solución a la ecuación de estructura de Fermi para un $y_0$ dado (comparar con el resultado de Lane-Emden \eqref{laneemden-densidadmedia}). Y como la densidad media total de una estrella es $\bar{\rho}=M/(4\pi R^3/3)$, tendremos que la densidad central también se puede expresar en función de su radio y masa total:
\begin{equation}
\boxed{ \rho_{\rm c}=\left(1-\frac{1}{y_0^2}\right)^{3/2}\frac{1}{4\pi}\frac{\eta_1^2\left\Vert\phi'(\eta_1)\right\Vert}{\eta_1^3}\frac{M}{R^3}.}
\end{equation}



\subsection{Casos límite de las soluciones}\label{sec:ec-fermi-limites}

\subsubsection{Aproximación de bajas densidades}
De la ecuación de estado de Fermi \eqref{ec-estado-fermi} para la densidad en términos del parámetro $x$ , notamos que si esta cantidad evaluada en el centro de la estrella es peque\~na: $x(\eta=0)=x_0\ll1$, entonces la densidad central de la estrella, $\rho_{\rm c}=Bx_0^3$, también lo será ($\rho_{\rm c}\ll B$, en donde los valores numéricos de esta constante están dados en \eqref{densidad_electrones-fermi} y \eqref{densidad_neutrones-fermi}). Por lo tanto, en la aproximación de bajas densidades a analizar, se despreciarán términos de orden $\mathcal{O}(x_0^4)$, y así, el lado derecho de la ecuación de estructura de Fermi \eqref{ec-fermi2} se podrá escribir, al considerar \eqref{sust-fermi}, como:
\begin{align}
 -\left(\phi^2-\frac{1}{y_0^2}\right)^{3/2}&=-\left(\phi^2-\frac{1}{1+x_0^2}\right)^{3/2}\\
&\approx-\left(\phi^2-(1-x_0^2)\right)^{3/2}.
\end{align}
Definiendo
\begin{equation}\label{ec-fermi-sust-5/3-1}
 \widetilde{\Theta}:=\phi^2-(1-x_0^2),
\end{equation}
podemos expresar $\phi$ en términos de $\widetilde{\Theta}$,
\begin{align}
\phi=\sqrt{1+\widetilde{\Theta}-x_0^2}\approx1-\frac{1}{2}\left(x_0^2-\widetilde{\Theta}\right)\qquad\Rightarrow\quad \frac{d\phi}{d\eta}\approx\frac{1}{2}\frac{d\widetilde{\Theta}}{d\eta},\label{ec-fermi-sust-5/3}
\end{align}
y así, podemos cambiar la variable dependiente en la ecuación de estructura de Fermi \eqref{ec-fermi2} de $\phi$ a $\widetilde{\Theta}$:
\begin{align}\label{ec-fermi-ec5/3-1}
 \frac{1}{2}\frac{1}{\eta^2}\frac{d}{d\eta}\left(\eta^2\frac{d\widetilde{\Theta}(\eta)}{d\eta}\right)=-\left[\widetilde{\Theta}(\eta)\right]^{3/2},
\end{align}
por lo que cambiando la variable independiente a $\widetilde{x}=\sqrt{2}\eta$, tenemos:
\begin{align}\label{ec-fermi-ec5/3-2}
 \frac{1}{\widetilde{x}^2}\frac{d}{d\widetilde{x}}\left(\widetilde{x}^2\frac{d\widetilde{\Theta}(\widetilde{x})}{d\widetilde{x}}\right)=-\left[\widetilde{\Theta}(\widetilde{x})\right]^{3/2},
\end{align}
es decir, hemos recuperado la ecuación de Lane-Emden  \eqref{laneemden} de índice $n=3/2$, pero con condiciones de borde distintas, ya que en el origen $\eta=0$, la variable $\bar{\Theta}$  definida en \eqref{ec-fermi-sust-5/3-1} tomará el valor:
\begin{align}\label{ec-fermi-bcs5/3}
 \widetilde{\Theta}(0)=\left(\underbrace{\cancelto{1}{\phi(0)}^2}_{\text{\eqref{bcs-fermi}}}-(1-x_0^2)\right)=x_0^2,
\end{align}
a diferencia de la función de Lane-Emden $\Theta_{3/2}(x)$ que satisface $\Theta_{3/2}(x=0)=1$. Pero igualmente se puede aprovechar esta solución conocida, notando que la función definida por:
\begin{equation}\label{ec-fermi-thetatilde5/3}
 \widetilde{\Theta}(\widetilde{x})=x_0^2\Theta_{3/2}(\sqrt{x_0} \,\widetilde{x}),
\end{equation}
también será una solución de la misma ecuación de Lane-Emden\footnote{Esta es una característica general de la soluciones de la ecuación de Lane-Emden, denominada propiedad de \emph{homología}: Si $\Theta_n(x)$ es una solución de la ecuación de Lane-Emden \eqref{laneemden} de índice $n$ con la condición de borde usual $\Theta_n(0)=1$, entonces $\widetilde{\Theta}(x)=A^{\frac{2}{n-1}}\Theta_n(Ax)$ también será solución de la misma ecuación, pero con condición de borde $\widetilde{\Theta}(0)=A^{\frac{2}{n-1}}$.}. En efecto, reemplazando $\widetilde{\Theta}(x)$ en el lado izquierdo de \eqref{laneemden} con $n=3/2$:
\begin{align} \frac{1}{x^2}\frac{d}{dx}\left(x^2\frac{d\widetilde{\Theta}(x)}{dx}\right)&=\frac{x_0^2}{x^2}\frac{d}{dx}\left(x^2\frac{d}{dx}\Theta_{3/2}(\sqrt{x_0}x)\right),\\
&=x_0^2(\sqrt{x_0})^2\left[\frac{1}{(\sqrt{x_0}x)^2}\frac{d}{d(\sqrt{x_0}x)}\left((\sqrt{x_0}x)^2\frac{d}{d(\sqrt{x_0}x)}\Theta_{3/2}(\sqrt{x_0}x)\right)\right]\\
&=-x_0^3\left[\Theta_{3/2}(\sqrt{x_0}x)\right]^{3/2}\\
&=-\left[\widetilde{\Theta}(x)\right]^{3/2}.
\end{align}
Luego, $\widetilde{\Theta}(\widetilde{x})$ también será solución de la ecuación de Lane-Emden \eqref{ec-fermi-ec5/3-2}, pero con la condición de borde \eqref{ec-fermi-bcs5/3} apropiada para este caso. Entonces, podemos volver a la variable independiente $\eta$ de la ecuación de estructura de Fermi original, y así escribir la solución de ella (en la aproximación $x_0\ll1$) en términos de la función conocida de Lane-Emden $\Theta_{3/2}$ dada en \eqref{ec-fermi-thetatilde5/3}, al reemplazar en la definición de $\phi$  \eqref{ec-fermi-sust-5/3}, obteniendo:
\begin{align}
 \phi(\eta)&\approx1-\frac{1}{2}\left(x_0^2-\widetilde{\Theta}(\widetilde{x})\right)\\
&\approx1-\frac{x_0^2}{2}\left(1-\Theta_{3/2}(\sqrt{2x_0} \,\eta)\right).\label{ec-fermi-phi5/3}
\end{align}

El radio adimensional de la estrella $\eta_1$, en esta aproximación, también se puede expresar en términos del radio adimensional correspondiente a la solución de la ecuación de Lane-Emden, dado por la raíz $x_{1_{3/2}}$ de $\Theta_{3/2}$ (dada en la tabla \ref{tablalaneemden}). Para ello, notamos que el borde de la estrella en las variables de la ecuación de estructura de Fermi está dado por la condición \eqref{ec-fermi-raices}:
\begin{align}
 \phi(\eta_1)=\frac{1}{y_0}&=\frac{1}{\sqrt{1+x_0^2}}\approx1-\frac{x_0^2}{2}.
\end{align}
Entonces, al igualar con la solución $\phi$ en \eqref{ec-fermi-phi5/3} evaluada en $\eta_1$, obtenemos:
\begin{align}
 \phi(\eta_1)&=1-\frac{x_0^2}{2}\left(\cancel{1}-\Theta_{3/2}(2\sqrt{x_0} \,\eta_1)\right)\approx1-\cancel{\frac{x_0^2}{2}},\\
\Rightarrow\quad \Theta_{3/2}(\sqrt{2x_0} \,\eta_1)&=\Theta_{3/2}(x_{1_{3/2}})=0.
\end{align}
Por lo tanto, tendremos que el radio adimensional de la estrella para el caso de la ecuación de estructura de Fermi, en el límite de bajas densidades, vendrá dado por:
\begin{equation}\label{ec-fermi-eta1}
 \eta_1=\frac{x_{1_{3/2}}}{\sqrt{2x_0}}.
\end{equation}

Ahora podemos verificar que la aproximación realizada es consistente con el obtenido a partir de la ecuación de estado politrópica \eqref{fermi_norelativista} con $\gamma=5/3$ , que se obtiene en la aproximación no relativista \eqref{xrelativo} del parámetro adimensional relativo : $x\ll1$, y que por \eqref{xrelativo_electrones} y  \eqref{xrelativo_neutrones} es equivalente al límite de baja densidad central de la estrella: $\rho_{\rm c}\ll10^{9}[kg/m^3]$ para electrones ó $\rho_{\rm c}\ll6\cdot10^{18}[kg/m^3]$ para neutrones. En efecto, de la expresión para el radio físico $R$ de la estrella en el contexto de la ecuación de estructura de Fermi, \eqref{ec-fermi-radio}, con $a$ dado por \eqref{ec-fermi-a}, y usando \eqref{rho-x-y-phi} y \eqref{ec-fermi-eta1}, se puede escribir en forma equivalente:
\begin{align}
 R&=\sqrt{\frac{2A}{\pi G}}\frac{\eta_1}{B}\frac{1}{y_0}\\
&=\sqrt{\frac{2A}{\pi G} }\frac{1}{B^{5/6}}\left(\frac{1}{B^{1/6}}\right)\frac{1}{\sqrt{1+x_0^2}}\\
&=\sqrt{\frac{2A}{\pi G B^{5/3}}}\left(\frac{x_0^3}{\rho_{\rm c}}\right)^{1/6}\left(\frac{x_{1_{3/2}}}{\sqrt{2x_0}}\right)\left(1-\frac{x_0^2}{2}\right)\\
&\approx\sqrt{\frac{A}{\pi G B^{5/3}}}\rho_{\rm c}^{-1/6}x_{1_{3/2}},
\end{align}
en donde se han despreciado términos de orden $\mathcal{O}(x_0^2)$. Reemplazando los valores  de $A$ y $B$ dados por \eqref{ec-fermi-A} y \eqref{ec-fermi-B}, tendremos que si identificamos (tanto para electrones como neutrones):
\begin{align}
 K_{5/3}=\frac{8}{5}\frac{A}{B_{e,n}^{5/3}}=
\begin{cases}
 \frac{3^{2/3}\pi^{4/3}}{5}\frac{\hbar^2 }{m_e(m_u\mu_e)^{5/3}}\\
\frac{3^{2/3}\pi^{4/3}}{5}\frac{\hbar^2 }{m_n^{8/3}}
\end{cases},
\end{align}
entonces recuperamos la expresión ya encontrada para el radio politrópico con $\gamma=5/3$ \eqref{radio5/3}, en donde $K_{5/3}$ son precisamente las constantes politrópicas que surgen de las ecuaciones de estado \eqref{fermi_norelativista} para electrones y \eqref{fermi_norelativista2} para neutrones, que corresponden al caso límite de la ecuación de estado de Fermi en la aproximación mencionada.

A la misma conclusión se puede llegar considerando la ecuación para la masa de la estrella \eqref{ec-fermi-masatotal}, teniendo presente \eqref{ec-fermi-phi5/3}:
\begin{align}
\frac{d\phi}{d\eta}&=\frac{x_0^2}{2}\sqrt{2x_0}\frac{d\Theta_{3/2}(\widetilde{x})}{d\widetilde{x}},\\
\Rightarrow\quad \left\Vert\phi'(\eta_1)\right\Vert&=\frac{x_0^{3/2}}{\sqrt{2}}\left\Vert\Theta'(\sqrt{2x_0}\eta_1)\right\Vert,
\end{align}
y usando \eqref{ec-fermi-eta1}, tendremos la relación entre los parámetros característicos de las soluciones de la ecuación exacta de estructura de Fermi y de Lane-Emden.
\begin{align}
 \eta_1^2\left\Vert\phi'(\eta_1)\right\Vert=\left(\frac{x_0^2}{2}\right)x_{1_{3/2}}^2\left\Vert\Theta'(x_{1_{3/2}})\right\Vert.
\end{align}




\subsubsection{Aproximación de altas densidades: Masa límite}\label{sec:ec-fermi-masalimite}
Consideremos ahora el caso en que se tome el límite $y_0\to\infty$ del parámetro que determina la ecuación de estructura de Fermi \eqref{ec-fermi2}, con lo cual ésta tomará la forma:
\begin{equation}
\frac{1}{\eta^2}\frac{d}{d\eta}\left(\eta^2\frac{d\phi}{d\eta}\right)=-\phi^3,
\end{equation}
cuya solución es la correspondiente a la  ecuación de Lane-Emden \eqref{laneemden} de índice $n=3$: $\phi=\Theta_{3}$ (considerando que ambas ecuaciones poseen las mismas condiciones de borde). Esta solución es conocida y ya se discutió en la página \pageref{masa4/3}, donde se dedujo su característica más importante: la masa de una estrella con índice politrópico asociado  $n=3$ ó $\gamma=4/3$ (por \eqref{ngamma}) es independiente de su densidad central, así como también del radio de ésta. Estas propiedades las podemos recuperar y ver de forma más clara su procedencia mediante el siguiente análisis: considerando la expresión para el radio físico de la estrella, \eqref{ec-fermi-radio}, podemos notar que al tomar el límite anterior:
\begin{equation}
 \lim_{y_0\to\infty}R=\lim_{y_0\to\infty}l\,\frac{\eta_1}{y_0}=0,
\end{equation}
ya que $\eta_1$ es finito. Además, considerando la ecuación \eqref{ec-fermi-masatotal} para la masa total de la estrella, podemos notar que de la correspondencia mencionada con las soluciones $\Theta_3$ de Lane-Emden, tenemos:
\begin{align}
 \lim_{y_0\to\infty}M&= \lim_{y_0\to\infty}4\pi\left(\frac{2A}{\pi G}\right)^{3/2}\frac{1}{B^2}\eta_1^2\left\Vert\phi'(\eta_1)\right\Vert,\\
&=4\pi\left(\frac{2A}{\pi GB^{4/3}}\right)^{3/2}x_1^2\left\Vert\Theta_3'(x_1)\right\Vert<\infty,\label{ec-fermi-masachandra}
\end{align}
es decir, conforme $y_0\to\infty$, el radio de la estrella tiende a cero pero su masa se acerca a un límite finito. Si expresamos la siguiente cantidad en términos de variables físicas mediante las definiciones de $A$ y $B$ en \eqref{ec-fermi-A} y \eqref{ec-fermi-B}, tanto para electrones como neutrones:
\begin{align}\label{ec-fermi-k4/3}
 K_{4/3}=\frac{2A}{B_{e,n}^{4/3}}=\begin{cases}
\frac{3^{1/3}\pi^{2/3}}{4}\frac{\hbar c }{(m_u\mu_e)^{5/3}}\\
\frac{3^{1/3}\pi^{2/3}}{4}\frac{\hbar c }{m_n^{4/3}}
\end{cases},
\end{align}
podemos notar que $K_{4/3}$ corresponde al índice politrópico para $\gamma=4/3$ dado en \eqref{fermi_relativista} y \eqref{fermi_relativista2}, de modo tal que obtenemos el mismo resultado para $M$ que el hallado en \eqref{masa4/3}, en donde este valor entonces representa la \emph{Masa de Chandrasekhar}:
\begin{equation}\label{masachandra2}
\boxed{\lim_{y_0\to\infty}M=M_{ch}=4\pi\left(\frac{K}{\pi G}\right)^{3/2}x_1^2\left\Vert\Theta'(x_1)\right\Vert},
\end{equation}
con $K$ y $\Theta$ los correspondientes a $\gamma=4/3$ ó $n=3$.

La predicción de la existencia de una masa límite es el comportamiento que se observa para objetos compactos, tales como enanas blancas y estrellas de neutrones, ya que se pueden considerar compuestos de gases de fermiones (electrones y neutrones, respectivamente) completamente degenerados. Entonces, $M_{ch}$ representa el límite superior que puede tener la masa de estas estrellas en condiciones de equilibrio y sujetas a la ecuación de estado de Fermi, por lo que esta cantidad física se puede expresar de forma conveniente en unidades de esta masa límite al dividir \eqref{ec-fermi-masatotal} por \eqref{ec-fermi-masachandra}:
\begin{align}\label{ec-fermi-masa-masachandra}
 \boxed{M=\frac{\eta_1^2\left\Vert\phi'(\eta_1)\right\Vert}{x_1^2\left\Vert\Theta_3'(x_1)\right\Vert}M_{ch}.}
\end{align}

También podemos verificar, independientemente del resultado anterior para la masa, que este caso límite es consistente con el obtenido a partir de la ecuación de estado politrópica con $\gamma=4/3$ \eqref{fermi_relativista}, que se obtiene en la aproximación ultrarelativista del parámetro adimensional relativo \eqref{xrelativo} : $x\gg1$, y que por \eqref{xrelativo_electrones} y \eqref{xrelativo_neutrones} es equivalente al límite de alta densidad central de la estrella: $\rho_{\rm c}\gg10^{9}\,[kg/m^3]$ para electrones ó $\rho_{\rm c}\gg6\cdot10^{18}\,[kg/m^3]$ para neutrones. En efecto,  tenemos que si $y_0$ es grande, entonces por \eqref{sust-fermi} y \eqref{cambio-fermi-adim}:
\begin{align}
y_0\gg 1\quad\Rightarrow\quad x=\sqrt{(y_0\phi)^2-1}\gg1,
\end{align}
ya que $\phi$ es acotado para todo valor físicamente relevante de $y_0$. De esta forma,  la función $f(x)$ dada en \eqref{ec-fermi-fx} será en este límite:
\begin{equation}
 x\gg1\quad\Rightarrow\quad f(x)\to 2Ax^4,
\end{equation}
por lo que la ecuación de estado de Fermi \eqref{ec-estado-fermi} en función del parámetro $x$ tendrá la forma asintótica:
\begin{align}
 x\gg 1\qquad\Rightarrow\qquad &P\to 2Ax^4\qquad\text{y} \qquad\rho\to Bx^3.
\end{align}
Debe notarse que aquí se observa claramente la equivalencia entre $y_0\gg 1$ y la aproximación de densidades altas. Finalmente, podemos obtener la ecuación de estado en forma explícita $P=P(\rho)$ combinando las dos expresiones paramétricas anteriores y recordando \eqref{ec-fermi-k4/3}:,
\begin{align}
 P&=\frac{2A}{B^{4/3}}\rho^{4/3}=K_{4/3}\rho^{4/3},
\end{align}
que es una ecuación de estado politrópica \eqref{estadopolitropica} de índice $\gamma=4/3$ (por lo que se puede analizar en base a la ecuación de Lane-Emden), que consistentemente coincide con el caso límite de la ecuación de estado de Fermi en la aproximación mencionada: \eqref{fermi_relativista} y \eqref{fermi_relativista2}.

\subsection{Solución numérica de la ecuación de estructura de Fermi}\label{sec:fermi-numerico}
Al igual que para la solución numérica de Lane-Emden en \ref{sec:lane-numerico}, se resolverá la ecuación de estructura de Fermi \eqref{ec-fermi2} mediante el algoritmo de Runge-Kutta de 4${}^{\circ}$ orden. Por lo tanto, definiendo las variables
\begin{equation}
 Y_1=\phi(\eta)\qquad\text{y}\qquad Y_2=\frac{d\phi(\eta)}{d\eta},
\end{equation}
y notando que la ecuación de estructura de Fermi se puede escribir en la forma:
\begin{align}
\frac{1}{\eta^2}\left(\eta^2\frac{d^2\phi(\eta)}{d\eta^2}+2\eta
\frac{d\phi(x)}{d\eta}\right)&=-\left(\phi(\eta)^2-\frac{1}{y_0^2}\right)^{3/2},\\
\Rightarrow\quad \phi''(\eta)&=-\left[\frac{2}{\eta}\phi'(\eta)+\left(\phi(\eta)^2-y_0^{-2}\right)^{3/2}\right],\label{ec-fermi-phi''}
\end{align}
tenemos que el sistema de ecuaciones diferenciales de primer orden apropiado para utilizar Runge-Kutta es:
\begin{equation}
\boxed{
\begin{aligned}
 Y_1'&=Y_2,\\
Y_2'&=-\left[\frac{2}{\eta}Y_2+\left(Y_1^2-y_0^{-2}\right)^{3/2}\right],
\end{aligned}}
\end{equation}
sujeto a las mismas condiciones iniciales que Lane-Emden:
\begin{align}
 Y_1(0)=1\qquad\text{y}\qquad Y_2(0)=0.
\end{align}
Utilizando el mismo tama\~no de paso $\Delta\eta=1\cdot 10^{-3}$, se procede a la integración obteniendo los valores respectivos de $Y_1$ e $Y_2$. Sin embargo, a diferencia de Lane-Emden, la integración se detendrá en la coordenada $\eta_1$ en la que la función decreciente $Y_1$ alcance el valor $1/y_0$, debido a que allí se alcanza el borde de la estrella (ver \eqref{ec-fermi-raices}). En dicho punto, el integrador también proporcionará el valor de $Y_2$, con el que se determina la cantidad característica $\eta_1^2\left\Vert\phi'(\eta_1)\right\Vert$, y de este modo tendremos a disposición los parámetros necesarios para especificar la masa, radio y densidad de estrellas modeladas por esta ecuación. Los resultados de la integración numérica para la variable $\phi(\eta)$ se muestran en el siguiente gráfico, para los cinco valores representativos del parámetro $1/y_0$ indicados.

\begin{figure}[H]
\centering
\includegraphics[angle=0,width=0.7\textwidth]{fig/fig-fermi-soluciones.pdf}
\caption{Soluciones $\phi(\eta)$ de la ecuación de estructura de Fermi.}\label{grafico-fermi-soluciones}
\end{figure}



Una importante relación física que se mostrará en la siguiente sección es la existente entre las masas y radios de las estrellas que obedecen a esta ecuación de estado. Para obtenerla, notamos que la cota superior de las masas está dada por la de Chandrasekhar \eqref{ec-fermi-masa-masachandra}, que corresponde a $y_0\to\infty$ ó $1/y_0=0$. Debido a esto, seleccionamos un conjunto apropiado de valores para este parámetro, que en nuestro caso son 500 puntos distribuidos uniformemente desde $1/y_0=0$ hasta $1/y_0=0.94$. Esta elección se justifica si recordamos que la condición mínima admisible sería para $1/y_0=1.0$, que correspondería al mismo valor de la condición de borde en el origen $\phi(0)=1=1/y_0$, lo que implicaría que el radio de la estrella sería nulo: $\eta_1=0$. Luego, los valores de la solución numérica de la ecuación de estructura por cada uno de estos $y_0$ nos proporcionará la cantidad $\eta_1^2\left\Vert\phi'(\eta_1)\right\Vert$, que determina el conjunto de masas estelares admisibles a través de \eqref{ec-fermi-masa-masachandra}: desde un valor mínimo por conocer, para $y_{\rm 0,min}$, hasta la masa máxima de Chandrasekhar, para $y_0=\infty$. Por otra parte, del algoritmo de resolución obtendremos también $\eta_1$ por cada $y_0$, con lo que tendremos los valores del radio de la estrella a través de \eqref{ec-fermi-radio} por cada valor dado de la masa. Por lo tanto, por cada integración numérica de la ecuación de estructura de Fermi, obtendremos un sólo punto del diagrama masa-radio, requiriéndose unas 500 en nuestro caso para su determinación con un grado de precisión aceptable.

De forma análoga también se determina otra importante relación física que se mostrará en la siguiente sección: la relación masa-densidad central. En efecto, si mediante el método anterior obtenemos un conjunto de valores para la masa estelar por cada $y_0$, entonces podemos asociar a cada uno de dichos puntos una densidad central $\rho_{\rm c}$ dada por \eqref{ec-fermi-byrho2}, puesto que depende únicamente del parámetro $y_0$.


\section{Aplicación: Estrellas degeneradas}
En las dos secciones previas se llegó a soluciones explícitas para la masa y radio de estrellas newtonianas que satisfacen tanto la ecuación de estado de Fermi \eqref{ec-estado-fermi} como las ecuaciones de estado politrópicas \eqref{estadopolitropica}, que en los casos de $\gamma=5/3$ y $\gamma=4/3$ son aproximaciones de la primera. Ahora bien, hay dos casos simples de estrellas que adoptan naturalmente una ecuación de estado de estos tipos, apropiadas para describir materia degenerada en donde la temperatura no juega un rol relevante (se considera $T=0 K$): las enanas blancas y las estrellas de neutrones. En ellas, la presión de materia que contrarresta a la gravitatoria está dada por la presión de degeneración de sus electrones o neutrones, respectivamente.


\subsection{Enanas blancas}\label{sec:enanasblancas}
Cuando una estrella agota todo el combustible disponible para la fusión nuclear, entonces comienza a enfriarse y contraerse, debido a que la presión térmica no puede contrarrestar la presión gravitatoria. La contracción tiene como consecuencia que la materia dentro de la estrella se pueda considerar compuesta fundamentalmente de un plasma de núcleos atómicos y electrones degenerados que obedecen a la estadística de Fermi-Dirac. Este gas será más ideal, es decir, la interacción electromagnética entre los núcleos y los electrones será más despreciable (predominando únicamente la degeneración de electrones), cuanto más alta sea la densidad de la materia considerada. Por lo tanto, cuando la temperatura es suficientemente baja, éstos últimos pueden ser modelados en primera aproximación como un gas ideal de Fermi completamente degenerado, en donde la principal contribución a la densidad de masa proviene de los núcleos atómicos. En estas circunstancias, es válida la ecuación de estado de Fermi exacta y sus aproximaciones politrópicas  relativista y ultrarelativista (con índices politrópicos $\gamma=5/3$ y $\gamma=4/3$), desarrolladas en el apéndice \ref{sec:ecsdeestado}.

\subsubsection{Masas y Radios}
Por lo tanto, podemos reemplazar dichas ecuaciones de estado para electrones en las relaciones radio y masa con respecto a la densidad central, y también en la expresión masa-radio de los dos capítulos anteriores, además de los resultados numéricos  de Lane-Emden resumidos en las raíces $x_1$ y $x_1^2\left|\Theta(x_1')\right|$ de la tabla \ref{tablalaneemden}, obteniendo las siguientes expresiones que se compararán gráficamente\footnote{En las relaciones mostradas, se ha usado como normalización el radio solar $R_{\odot}=6.955\cdot 10^{8}\,[m]$ y la masa solar $M_{\odot}=1.99\cdot10^{30}\,[kg]$.}:

\begin{enumerate}

 \item \emph{Caso no relativista, baja densidad} ($\gamma=5/3$ y $\rho_{\rm c}\ll10^{9}\,[kg/m^3]$).
Aquí usamos la constante $K$ para electrones dada en \eqref{fermi_norelativista}
\begin{itemize}
 \item \emph{Radio}:
Reemplazando $K$ en \eqref{radio5/3}, tenemos la siguiente relación radio-densidad central:
\begin{align}
R&=\left(\frac{3^{2/3}\cdot\pi^{1/3}}{8}\frac{\hbar^2}{Gm_e m_u^{5/3}}\right)^{1/2}x_1\,\mu_e^{-5/6}\rho_{\rm c}^{-1/6}\\
% &\approx 3.5470\cdot10^{5}\cdot \left(\frac{\mu_e}{2}\right)^{-5/6}\rho_{\rm c}^{-1/6}\\
% &\approx 0.5010\cdot \left(\frac{\mu_e}{2}\right)^{-5/6}\rho_{\rm c}^{-1/6}\,R_{\odot},\\%0.90871
 &\approx1.1216\cdot10^{4}\left[\frac{\rho_{\rm c}}{10^{9}\,[kg/m^3]}\right]^{-1/6}\left(\frac{\mu_e}{2}\right)^{-5/6}\,[km].
\end{align}
\item \emph{Masa}:
Reemplazando $K$ en \eqref{masa5/3}, tenemos la siguiente relación masa-densidad central:
\begin{align}
M&=4\pi\left(\frac{3^{2/3}\pi^{1/3}}{8}\frac{\hbar^2}{Gm_e m_u^{5/3}}\right)^{3/2}x_1^2\left|\Theta(x_1')\right|\mu_e^{-5/2}\rho_{\rm c}^{1/2}\\
&\approx1.5687\cdot 10^{-5}\,\left(\frac{\mu_e}{2}\right)^{-5/2}\rho_{\rm c}^{1/2}\,M_{\odot}\label{enanablanca5/3-masadensidad}\\
&\approx0.4961\cdot\left[\frac{\rho_{\rm c}}{10^{9}\,[kg/m^3]}\right]^{1/2}\left(\frac{\mu_e}{2}\right)^{-5/2}\,M_{\odot}.
\end{align}
Y reemplazando $K$ en \eqref{masaradio5/3}, obtenemos la siguiente relación masa-radio:
\begin{align}
 M&=\frac{3^2\pi^2}{2^7}\left(\frac{\hbar^2}{Gm_em_u^{5/3}}\right)^3\,x_1^3\,x_1^2\left|\Theta(x_1')\right|\mu_e^{-5} R^{-3}\\
% \frac{M}{M_{\odot}}&=132.384\frac{3^2\pi^2}{2^7}\left(\frac{\hbar^2}{R_{\odot}Gm_em_u^{5/3}}\right)^3 M_{\odot}^{-1}\mu_e^{-5} \left(\frac{100R}{R_{\odot}}\right)^{-3}\\
&\approx7.001\cdot10^{11}\,\left(\frac{\mu_e}{2}\right)^{-5} \left[\frac{R}{1\,[km]}\right]^{-3}M_{\odot} \label{enanablanca5/3-masaradio}\\
% &\approx 2.0809\,\left(\frac{\mu_e}{2}\right)^{-5} \left(\frac{100R}{R_{\odot}}\right)^{-3}M_{\odot}, \label{enanablanca5/3-masaradio}\\
&\approx0.7001\cdot\left[\frac{R}{10^{4}\,[km]}\right]^{-3}\left(\frac{\mu_e}{2}\right)^{-5}\,M_{\odot}.
\end{align}
Recordemos que debido a la discusión de la sección \eqref{sec:casos-lane-emden}, en este caso tendremos que $MV=cte$, es decir, el volumen de una enana blanca es inversamente proporcional a su masa. Esta afirmación proviene del hecho que la estrella se soporta contra el colapso gravitatorio debido a la presión de degeneración de los electrones. éstos deben estar más cercanamente confinados (menor volumen) para generar una presión de degeneración más grande (de acuerdo con el principio de exclusión de Pauli), necesaria para soportar una estrella más masiva.
\end{itemize}

 \item \emph{Caso ultra-relativista, alta densidad} ($\gamma=4/3$ y $\rho_{\rm c}\gg10^{9}\,[kg/m^3]$). Aquí usamos la constante $K$ para electrones dada en \eqref{fermi_relativista}.
\begin{itemize}
 \item \emph{Radio}: Reemplazando $K$ en \eqref{radio4/3}, tenemos la siguiente relación radio-densidad central:
\begin{align}
R&=\left(\frac{3^{1/3}}{4\pi^{1/3}}\frac{\hbar c}{Gm_u^{4/3}}\right)^{1/2}x_1\,\mu_e^{-2/3}\,\rho_{\rm c}^{-1/3}\\
% &\approx3.3462\cdot10^{8}\cdot\left(\frac{\mu_e}{2}\right)^{-2/3}\,\rho_{\rm c}^{-1/3}\,[km]\\
% &\approx 48.112\cdot\left(\frac{\mu_e}{2}\right)^{-2/3}\,\rho_{\rm c}^{-1/3}R_{\odot}\\
 &\approx3.3461\cdot10^{4}\left[\frac{\rho_{\rm c}}{10^{9}\,[kg/m^3]}\right]^{-1/3}\left(\frac{\mu_e}{2}\right)^{-2/3}\,[km],
\end{align}
\item \emph{Masa}: Para este valor del índice politrópico, tenemos que la masa, denominada de Chandrasekhar, representa un valor límite (ver sección \eqref{sec:ec-fermi-masalimite}), es independiente tanto de la densidad central como del radio de la estrella, y su valor se obtendrá reemplazando $K$ en \eqref{masachandra2} ó \eqref{masa4/3}:
\begin{align}\label{masachandra}
M_{ch}&=\frac{\sqrt{3\pi}}{2}\left(\frac{\hbar c}{Gm_u^{4/3}}\right)^{3/2}x_1^2\left|\Theta(x_1')\right|\mu_e^{-2}\\
 &\approx1.4562\cdot\left(\frac{\mu_e}{2}\right)^{-2}\,M_{\odot},\label{enanablanca4/3-masa}
\end{align}
de donde vemos que la masa de Chandrasekhar depende únicamente de la composición de la enana blanca a través del peso molecular medio por electrón $\mu_e$. Para estrellas compuestas de helio o carbono, donde $\mu_e=2$, su valor es $M_{ch}\approx1.45M_{\odot}$.
\end{itemize}

\item \emph{Ec. de estado de Fermi exacta}. En este caso usamos los resultados obtenidos en la sección \eqref{sec:fermi-exacta}.
\begin{itemize}
\item \emph{Radio}: Reemplazando las constantes $A$ \eqref{ec-fermi-A} y $B$ \eqref{ec-fermi-B} para el caso de electrones en la definición de $a$ \eqref{ec-fermi-a}, y ésta en la expresión para el radio \eqref{ec-fermi-radio}, tenemos la siguiente relación del radio con el parámetro $y_0$ y el radio adimensional $\eta_1$:
\begin{align}
R&=\frac{1}{2}\sqrt{\frac{3\pi}{cG}}\frac{\hbar^{3/2}}{ m_e m_u \mu_e}\left(\frac{\eta_1}{y_0}\right)\\
&=3.8849\cdot10^3\left(\frac{\mu_e}{2}\right)^{-1}\left(\frac{\eta_1}{y_0}\right)\;[km].\label{enanablanca-exacta-radio}
% &=0.55858\left(\frac{\mu_e}{2}\right)^{-1}\left(\frac{\eta_1}{y_0}\right)\;\frac{R_{\odot}}{100}.\label{enanablanca-exacta-radio}
\end{align}
\item \emph{Masa}: Podemos expresar la masa de las enanas blancas en términos de la masa límite de Chandrasekhar \eqref{enanablanca4/3-masa} mediante la relación \eqref{ec-fermi-masa-masachandra}. Al reemplazar los valores correspondientes de la tabla \ref{tablalaneemden}, obtenemos la siguiente relación de la masa con el valor característico $\eta_1^2\left\Vert\phi'(\eta_1)\right\Vert$:
\begin{align}
M&=\frac{\eta_1^2\left\Vert\phi'(\eta_1)\right\Vert}{x_1^2\left\Vert\Theta_3'(x_1)\right\Vert}\,M_{ch}\\
&\approx0.72122\left(\frac{\mu_e}{2}\right)^{-2}\eta_1^2\left\Vert\phi'(\eta_1)\right\Vert\, M_{\odot}.\label{enanablanca-exacta-masa}
\end{align}
\item \emph{Densidad central}: Reemplazando el valor de $B$ dado por \eqref{ec-fermi-B} (para electrones) en \eqref{ec-fermi-byrho2}, obtenemos la siguiente relación de la densidad central con el parámetro $y_0$:
\begin{align}
\rho_{\rm c}&=\frac{m_e^3 c^3}{3\pi^2\hbar^3}m_u\mu_e(y_0^2-1)^{3/2}\\
&\approx1.9478\cdot10^{9}\left(\frac{\mu_e}{2}\right)\,(y_0^2-1)^{3/2}\,[kg/m^3].\label{enanablanca-exacta-densidad}
\end{align}
\end{itemize}


\end{enumerate}

\subsubsection{Gráficos y comparación entre modelos exactos y aproximados}
A continuación, puede apreciarse el gráfico \ref{graficomasa-radio}, donde se muestra la relación masa-radio para enanas blancas (con el valor característico para el peso molecular medio por electrón $\mu_e=2$), obtenida directamente por las relaciones provenientes de las ecuaciones de estado aproximadas \eqref{enanablanca5/3-masaradio} y \eqref{enanablanca4/3-masa}, además de la hallada a partir de la solución numérica de la ecuación de estado de Fermi exacta por medio del método descrito en la subsección \ref{sec:fermi-numerico}, donde se usan los valores de \eqref{enanablanca-exacta-masa} y \eqref{enanablanca-exacta-radio}.

\begin{figure}[H]
\centering
\includegraphics[angle=0,width=0.7\textwidth]{fig/fig-fermielectron-masa-radio.pdf}
\caption{Relación masa-radio para enanas blancas compuestas un gas de electrones completamente degenerado en los casos límite no relativista ($\gamma=5/3$) y ultra relativista ($\gamma=4/3$).}\label{graficomasa-radio}
\end{figure}

En primer lugar se debe notar que la masa y el radio obtenidas de la relación exacta para enanas blancas, son inversamente proporcional para todo valor del radio. Además, en la región de radios grandes, la curva exacta se superpone a la correspondiente al caso no relativista, por lo que ésta es una buena aproximación para radios de enanas blancas mayores a unos 15000 km. (2\% del radio solar), que corresponden a masas del orden de un 20\% de la correspondiente al Sol. Estos cantidades representan entonces los valores típicos para estos parámetros en este tipo de estrellas.

Por otra parte, cuando el radio de una enana blanca comienza a disminuir por debajo de unos 10000 km (1.5 \% del radio solar), la curva no relativista comienza a diferenciarse notoriamente del resultado exacto, indicando que se está entrando en el rango de validez de la aproximación ultra relativista. ésta es una buena aproximación para radios menores que unos 5000 km. (similar al radio de la Tierra), pues predice que la masa tiende al valor límite de la masa de Chandrasekhar $M_{ch}$ dado en \eqref{enanablanca4/3-masa}. Nótese que, mediante la curva exacta, se observa claramente lo mencionado en la sección \ref{sec:ec-fermi-masalimite}: la masa de Chandrasekhar sólo se alcanza cuando el radio estelar tiende a cero, que es cuando el parámetro $y_0$ de la ecuación de estructura de Fermi tiende a infinito.

La última afirmación del párrafo anterior implica que la densidad media de la estrella, y por ende la densidad central, (recordar que debido a \eqref{ec-fermi-densidadmediaycentral} ambas cantidades son proporcionales) tenderán a infinito, por lo que es relevante conocer qué relación existe entre la masa y densidad central de las enanas blancas. Esto se representa en el gráfico (\ref{graficomasa-densidad}), donde se muestra dicha relación obtenida directamente por las relaciones provenientes de las ecuaciones de estado aproximadas \eqref{enanablanca5/3-masadensidad} y \eqref{enanablanca4/3-masa}, además de la hallada a partir de la solución numérica de la ecuación de estado de Fermi exacta  por medio del método descrito en la subsección \ref{sec:fermi-numerico}, donde se usan los valores \eqref{enanablanca-exacta-masa} y \eqref{enanablanca-exacta-densidad}.

\begin{figure}[H]
\centering
\includegraphics[angle=0,width=0.7\textwidth]{fig/fig-fermielectron-masa-densidad.pdf}
\caption{Relación masa-densidad central para enanas blancas compuestas un gas de electrones completamente degenerado en los casos límite no relativista ($\gamma=5/3$) y ultra relativista ($\gamma=4/3$).}\label{graficomasa-densidad}
\end{figure}

La primera característica relevante que se observa de la curva exacta es que la masa es monótonamente creciente con respecto a la densidad central. Aún más importante es el hecho que aquí se pueden ver claramente los rangos de validez de las aproximaciones efectuadas: el caso no relativista se determinó haciendo la aproximación $\rho_{\rm c}\ll10^{9}\,[kg/m^3]$, mientras que el ultra relativista se obtuvo para $\rho_{\rm c}\gg10^{9}\,[kg/m^3]$, y consistemente, ambos casos límite se superponen a la curva exacta en sus dominios correspondientes. Sin embargo, la curva de la aproximación no relativista es ya muy cercana a la exacta tan sólo para un orden de magnitud menor de densidades centrales que la densidad crítica, mientras que la aproximación ultra relativista posee el mismo margen de exactitud para más de tres órdenes de magnitud por encima del mismo valor.

La razón física del comportamiento anterior se puede entender del siguiente modo: en el rango no relativista, es válida la relación masa volumen $MV={\rm cte}$ (\eqref{masavolumen5/3}), por lo que agregando cada vez más masa a enanas blancas de baja densidad, dichos astros se encogerían proporcionalmente, pero sin límite: eventualmente alcanzarían $R\to 0$ y $M\to\infty$ (ver gráfico \ref{graficomasa-radio}), resultando por tanto en una densidad central claramente divergente (ver gráfico \ref{graficomasa-densidad}). Pero este comportamiento no ocurre debido a que cuando las densidades en el interior de estas estrellas exceden los $10^{9}\,[kg/m^3]$, los electrones se encuentran tan cercanamente confinados que, debido al principio de exclusión de Pauli, estos obtendrían un momentum tal que daría como resultado velocidades cercanas a las de la luz. Por esta razón es que se entra en el rango relativista para densidades mayores que la crítica, en donde se observa (ver gráfico \ref{graficomasa-densidad}) que la masa se aproxima asintóticamente a la de Chandrasekhar sólo cuando  $\rho_{\rm c}\to\infty$, que es consistente con el hecho antes mencionado que el radio debe ser nulo para que se logre dicha condición.


\subsection{Estrellas de neutrones}

Según el argumento de la sección previa, a medida que la densidad de una enana blanca se incrementa en el dominio relativista, se debería obtener la masa límite de Chandrasekhar. Pero ésta nunca se alcanza, pues antes de ello las extremas densidades presentes en el núcleo estelar producen un tipo de reacción nuclear conocido como decaimiento beta inverso\footnote{Esta reacción nuclear es predominante cuando las densidades centrales de las enanas blancas superan los $4\cdot10^{14}\;[kg/m^3]$, mientras que la presión de degeneración es dominada por neutrones para densidades mayores a $4\cdot10^{15}\;[kg/m^3]$. Para detalles, ver capítulo 3 del texto de Shapiro \cite{Shapiro83}. }:
\begin{equation}
 p+e^{-}\quad\longrightarrow\quad n+\nu_{e},
\end{equation}
lo que cambia la composición de la enana blanca antes de su colapso total a una compuesta fundamentalmente por neutrones, donde el nuevo estado de equilibrio se alcanza debido a la presión de degeneración de estos fermiones, formándose así las estrellas de neutrones. Por lo tanto, éstas se pueden modelar en primera aproximación como un gas ideal de neutrones completamente degenerado, donde la masa total se debe únicamente sólo a ellos. En estas circunstancias, es válida la misma ecuación de estado de Fermi exacta y sus aproximaciones politrópicas no relativista ($\gamma=4/3$) y ultra relativista ($\gamma=5/3$), en donde sólo cambian las constantes referidas a neutrones en vez de a electrones.

\subsubsection{Masas y Radios}
Siguiendo el mismo desarrollo que para enanas blancas, podemos obtener las siguientes relaciones para los radios y masas de estrellas de neutrones, en sus dos aproximaciones características delimitadas por el valor de $x_F$ en \eqref{xrelativo_neutrones}:

\begin{enumerate}

\item \emph{Caso no relativista, baja densidad} ($\gamma=5/3$ y $\rho_{\rm c}\ll6\cdot10^{18}\,[kg/m^3]$).
Aquí usamos la constante $K$ para neutrones dada en \eqref{fermi_norelativista2}.
\begin{itemize}
 \item \emph{Radio}:
Reemplazando $K$ en \eqref{radio5/3}, tenemos la siguiente relación radio-densidad central:
\begin{align}
R&=\left(\frac{3^{2/3}\cdot\pi^{1/3}}{8}\frac{\hbar^2}{G m_n^{5/3}}\right)^{1/2}x_1\,\rho_{\rm c}^{-1/6}\\
% &\approx 0.02104\cdot \rho_{\rm c}^{-1/6}\,R_{\odot},\\%0.90871
 &\approx14.633\left[\frac{\rho_{\rm c}}{10^{18}\,[kg/m^3]}\right]^{-1/6}\,[km].
\end{align}
\item \emph{Masa}:
Reemplazando $K$ en \eqref{masa5/3}, tenemos la siguiente relación masa-densidad central:
\begin{align}
M&=4\pi\left(\frac{3^{2/3}\pi^{1/3}}{8}\frac{\hbar^2}{G m_n^{8/3}}\right)^{3/2}x_1^2\left|\Theta(x_1')\right|\rho_{\rm c}^{1/2}\\
&\approx1.1015\cdot 10^{-9}\,\rho_{\rm c}^{1/2}\,M_{\odot}\label{neutrones5/3-masadensidad}\\
&\approx1.1015\cdot\left[\frac{\rho_{\rm c}}{10^{18}\,[kg/m^3]}\right]^{1/2}\,M_{\odot}.
\end{align}
Y reemplazando $K$ en \eqref{masaradio5/3}, obtenemos la siguiente relación masa-radio:
\begin{align}
 M&=\frac{3^2\pi^2}{2^7}\left(\frac{\hbar^2}{Gm_n^{8/3}}\right)^3\,x_1^3\,x_1^2\left|\Theta(x_1')\right| R^{-3}\\
% \frac{M}{M_{\odot}}&=132.384\frac{3^2\pi^2}{2^7}\left(\frac{\hbar^2}{R_{\odot}Gm_em_u^{5/3}}\right)^3 M_{\odot}^{-1}\mu_e^{-5} \left(\frac{100R}{R_{\odot}}\right)^{-3}\\
% &\approx 10.2601\, \left(\frac{10^5 R}{R_{\odot}}\right)^{-3}M_{\odot}, \\
&\approx3.4518\cdot\left[\frac{R}{10\,[km]}\right]^{-3}\,M_{\odot}.\label{neutrones5/3-masaradio}
\end{align}

\end{itemize}

 \item \emph{Caso ultra-relativista, alta densidad} ($\gamma=4/3$ y $\rho_{\rm c}\gg6\cdot10^{18}\,[kg/m^3]$). Aquí usamos la constante $K$ para neutrones dada en \eqref{fermi_relativista2}
\begin{itemize}
 \item \emph{Radio}: Reemplazando $K$ en \eqref{radio4/3}, tenemos la siguiente relación radio-densidad central:
\begin{align}
R&=\left(\frac{3^{1/3}}{4\pi^{1/3}}\frac{\hbar c}{Gm_n^{4/3}}\right)^{1/2}x_1\,\rho_{\rm c}^{-1/3}\\
% &\approx 75.935\,\rho_{\rm c}^{-1/3}R_{\odot}\\
 &\approx52.813\left[\frac{\rho_{\rm c}}{10^{18}\,[kg/m^3]}\right]^{-1/3}\,[km].
\end{align}
\item \emph{Masa}: La masa de este caso límite, denominada de Chandrasekhar, se obtendrá reemplazando $K$ en \eqref{masachandra2} ó \eqref{masa4/3}:
\begin{align}\label{masachandra-neutrones}
M_{ch}&=\frac{\sqrt{3\pi}}{2}\left(\frac{\hbar c}{Gm_n^{4/3}}\right)^{3/2}x_1^2\left|\Theta(x_1')\right|\mu_e^{-2}\\
 &\approx5.7252\,M_{\odot},\label{neutrones4/3-masa}
\end{align}
de donde vemos que es aproximadamente 4 veces mayor que la correspondiente masa de Chandrasekhar para enanas blancas \eqref{enanablanca4/3-masa}, considerando $\mu_e=2$.
\end{itemize}

\item \emph{Ec. de estado de Fermi exacta}. Aquí usamos los resultados obtenidos en la sección \eqref{sec:fermi-exacta}.
\begin{itemize}
\item \emph{Radio}: Reemplazando las constantes $A$ \eqref{ec-fermi-A} y $B$ \eqref{ec-fermi-B} para el caso de neutrones en la definición de $a$ \eqref{ec-fermi-a}, y ésta en la expresión para el radio \eqref{ec-fermi-radio}, tenemos la siguiente relación del radio con el parámetro $y_0$ y el radio adimensional $\eta_1$:
\begin{align}
R&=\frac{1}{2}\sqrt{\frac{3\pi}{cG}}\frac{\hbar^{3/2}}{m_n^2}\left(\frac{\eta_1}{y_0}\right)\\
% &=0.60236\left(\frac{\eta_1}{y_0}\right)\;\frac{R_{\odot}}{10^{5}}.\label{neutrones-exacta-radio}
&=4.18945\left(\frac{\eta_1}{y_0}\right)\,[km].\label{neutrones-exacta-radio}
\end{align}
\item \emph{Masa}: Notando que aquí también es válido \eqref{enanablanca-exacta-masa}, pero usando el límite de Chandrasekhar para neutrones \eqref{neutrones4/3-masa}, podemos encontrar que:
\begin{align}
M&=\frac{\eta_1^2\left\Vert\phi'(\eta_1)\right\Vert}{x_1^2\left\Vert\Theta_3'(x_1)\right\Vert}\,M_{ch}.\\
&=2.83673\,\eta_1^2\left\Vert\phi'(\eta_1)\right\Vert\, M_{\odot}.\label{neutrones-exacta-masa}
\end{align}
\item \emph{Densidad central}: Reemplazando el valor de $B$ dado en \eqref{ec-fermi-B} para neutrones, en \eqref{ec-fermi-byrho2}, obtenemos la siguiente relación de la densidad central con el parámetro $y_0$:
\begin{align}
\rho_{\rm c}&=\frac{m_n^4 c^3}{3\pi^2\hbar^3}(y_0^2-1)^{3/2}\\
&=6.10656\cdot10^{18}\,(y_0^2-1)^{3/2}\,[kg/m^3].\label{neutrones-exacta-densidad}
\end{align}
\end{itemize}


\end{enumerate}

\subsubsection{Gráficos}
En el gráfico siguiente se muestra la relación masa-radio para estrellas de neutrones, obtenida directamente por las relaciones provenientes de las ecuaciones de estado aproximadas \eqref{neutrones5/3-masaradio} y \eqref{neutrones4/3-masa}, además de la hallada a partir de la solución numérica de la ecuación de estado de Fermi exacta por medio del método descrito en la subsección \ref{sec:fermi-numerico}, donde se usan los valores \eqref{neutrones-exacta-masa} y \eqref{neutrones-exacta-radio}.


\begin{figure}[H]
\centering
\includegraphics[angle=0,width=0.7\textwidth]{fig/fig-fermineutron-masa-radio.pdf}
\caption{Relación masa-radio para estrellas de neutrones compuestas un gas de neutrones completamente degenerado en los casos límite no relativista ($\gamma=5/3$) y ultra relativista ($\gamma=4/3$). Se destaca, además, el valor más nombrado en la literatura para la masa y el radio de estas estrellas (1 $M_\odot$, $10[km]$).}\label{graficomasa-radio-neutrones}
\end{figure}

Evidentemente, este gráfico corresponde sólo a un reescalamiento del correspondiente a enanas blancas \ref{graficomasa-radio}, siendo válidos los mismos comentarios hechos allí. Por tanto, lo único digno de mencionar es que, si bien las masas típicas de las estrellas de neutrones están en torno a una masa solar, sus radios típicos son unas 1000 veces menores que los correspondientes a las enanas blancas, siendo tan sólo de algunas decenas de kilómetros.

Por otra parte, en el siguiente gráfico, se muestra la relación masa-densidad obtenida directamente por las relaciones provenientes de las ecuaciones de estado aproximadas \eqref{neutrones5/3-masadensidad} y \eqref{neutrones4/3-masa}, además de la hallada a partir de la solución numérica de la ecuación de estado de Fermi exacta por medio del método descrito en la subsección \ref{sec:fermi-numerico}, donde se usan los valores de \eqref{neutrones-exacta-masa} y \eqref{neutrones-exacta-densidad}.

\begin{figure}[H]
\centering
\includegraphics[angle=0,width=0.7\textwidth]{fig/fig-fermineutron-masa-densidad.pdf}
\caption{Relación masa-densidad central estrellas de neutrones compuestas un gas de neutrones completamente degenerado en los casos límite no relativista ($\gamma=5/3$) y ultra relativista ($\gamma=4/3$)}\label{graficomasa-densidad-neutrones}
\end{figure}

Obviamente, este gráfico también corresponde a un reescalamiento del correspondiente al de las enanas blancas (ver figura \ref{graficomasa-densidad}). Los radios extremadamente peque\~nos que exhiben las estrellas de neutrones mostrados en la figura \ref{graficomasa-radio-neutrones}, implican necesariamente que su densidad debe ser gigantesca: aquí se puede ver claramente que sus densidades centrales típicas son del orden de $\rho_{\rm c}\approx10^{18}\,[kg/m^3]$, que corresponden al orden de magnitud de las densidades de los núcleos atómicos. Esto se puede entender al pensar que estos astros están compuestos prácticamente sólo de neutrones, tan cercanamente confinados que prácticamente no hay espacio entre ellos: es en este sentido que las estrellas de neutrones son como un núcleo atómico gigante.


%\section[Principio variacional]{Energía de una estrella y Principio variacional *}\label{sec:ppio_variacional_newton}
%
%\subsection{Energía de una estrella politrópica}\label{sec:energia_newton}
%
%En esta subsección se calculará explícitamente la energía interna $U$ y potencial gravitacional $V$, para una estrella politrópica que satisface \eqref{estadopolitropica}. La energía total de este sistema será simplemente la suma de ambas.
%
%\subsubsection{Energía potencial gravitacional}
%El cálculo se hará de dos formas distintas, las cuales se compararán posteriormente para determinar una integral que se utilizará para la energía interna:
%\begin{enumerate}
% \item \emph{Método 1: Usando la ecuación de equilibrio hidrostático \eqref{eqnewton}.}
%
%La energía potencial gravitatoria total de una estrella se definirá por:
%\begin{align}\label{energia_potencial_newton}
%V=-\int_0^R\frac{G\mathcal{M}(r)}{r}d\mathcal{M}.
%\end{align}
%Usando \eqref{masa1} para determinar $d\mathcal{M}(r)$ y la ecuación de equilibrio hidrostático \eqref{eqnewton} para introducir la presión $P$, tenemos que:
%\begin{align}
% V&=-\int\limits_0^R\frac{G\mathcal{M}(r)}{r}\left[4\pi r^2\rho(r)\,dr\right]=-4\pi\int\limits_0^R  r\left[G\mathcal{M}(r)\rho(r)\right]\,dr \label{energia_potencial_newton1}\\
%&=4\pi\int\limits_0^R r\left[-r^2\frac{dP}{dr}\right]\,dr=4\pi\int\limits_0^R r^3\,dP.
%\end{align}
%Integrando por partes, podemos escribir:
%\begin{align}
% V&=4\pi\int\limits_0^R r^3\,dP=4\pi\left[\int\limits_0^R d(Pr^3)-\int\limits_0^R P\,d(r^3)\right]\\
%&=4\pi\left[\cancelto{0}{P(R)}R^3-\lim_{r\to 0}(P(r)\cancelto{0}{r^3})-3\int\limits_0^R P\,r^2\,dr\right]\\
%&=-3\int\limits_0^R P\,4\pi r^2\,dr.\label{energia_potencial1}
%\end{align}
%Si ahora multiplicamos y dividimos por la densidad $\rho$, podemos usar la expresión para la masa $\mathcal{M}$ \eqref{masa2}, e integrar nuevamente por partes:
%\begin{align}
% V&=-3\int\limits_0^R \frac{P}{\rho}\,\left[4\pi \rho r^2 \,dr\right]=-3\int\limits_0^R\frac{P}{\rho}d\mathcal{M}\\
%&=-3\int\limits_0^R\left[d\left(\frac{P}{\rho}\mathcal{M}\right)-\mathcal{M}d\left(\frac{P}{\rho}\right)\right]\\
%&=-3\left[\cancelto{0}{\left(\lim_{r\to R}\frac{P}{\rho}\right)}\mathcal{M}-\left(\lim_{r\to 0}\frac{P}{\rho}\cancelto{0}{\mathcal{M}}\right)\right]+3\int\limits_0^R\mathcal{M}\,d\left(\frac{P}{\rho}\right).\label{energia_potencial_2}
%\end{align}
%El primer límite se anula debido a que, por la ecuación de estado politrópica, tendremos la relación siguiente
%\begin{align}
% \frac{P}{\rho}=K\rho^{\gamma-1}\quad\Rightarrow\quad \lim_{r\to R}\left(\frac{P}{\rho}\right)=K[\cancelto{0}{\rho(R)}]^{\gamma-1}=0,
%\end{align}
%para todos los valores físicamente aceptables del índice politrópico ($\gamma>6/5$). Con el objetivo de seguir simplificando $V$, primero notamos que
%\begin{align}
% P=K\rho^{\gamma}\qquad\Rightarrow\qquad \rho=\left(\frac{P}{K\rho}\right)^{1/(\gamma-1)}.
%\end{align}
%Con esta relación, es posible expresar el diferencial $d\left(P/\rho\right)$ en términos del diferencial $dP$, pues:
%\begin{align}
% dP&=d(K\rho^{\gamma})=K\gamma\rho^{\gamma-1}d\rho,\\
% &=K\gamma\rho^{\gamma-1}d\left(\frac{P}{K\rho}\right)^{1/(\gamma-1)}\\
%&=\frac{\gamma}{\gamma-1}\rho^{\gamma-1}\left(\frac{P}{K\rho}\right)^{1/(\gamma-1)-1}\,d\left(\frac{P}{\rho}\right)\\
%&=\frac{\gamma}{\gamma-1}\rho^{\gamma-1}\left(\rho^{\gamma-1} \right)^{(2-\gamma)/(\gamma-1)} \,d\left(\frac{P}{\rho}\right)\\
%&=\frac{\gamma}{\gamma-1}\;\rho\;d\left(\frac{P}{\rho}\right).
%\end{align}
%Así, reemplazando la relación anterior en \eqref{energia_potencial_2} y usando nuevamente la ecuación de equilibrio \eqref{eqnewton}, tenemos:
%\begin{align}
% V&=3\int\limits_0^R\mathcal{M}\,\frac{\gamma-1}{\gamma}\frac{dP}{\rho}=3\frac{\gamma-1}{\gamma}\int\limits_0^R\mathcal{M}\,\left[\frac{1}{\rho}\frac{dP}{dr}\right]\,dr\\
%&=3\frac{\gamma-1}{\gamma}\int\limits_0^R\mathcal{M}\,\left[-\frac{G\mathcal{M}}{r^2}\right]\,dr\\
%&=3G\frac{\gamma-1}{\gamma}\int\limits_0^R\mathcal{M}^2\,d\left(\frac{1}{r}\right).\label{relacion_dp}
%\end{align}
%Integrando por partes, podemos escribir:
%\begin{align}
% V&=3G\frac{\gamma-1}{\gamma}\int\limits_0^R\left[d\left(\frac{\mathcal{M}^2}{r}\right)-d\left(\mathcal{M}^2\right)\frac{1}{r}\right]\\
%&=3G\frac{\gamma-1}{\gamma}\left[\frac{[\mathcal{M}(R)]^2}{R}-\lim_{r\to 0}\cancelto{0}{\left(\frac{\mathcal{M}^2}{r}\right)}-2\int\limits_0^R\,\frac{\mathcal{M}}{r}d\mathcal{M}\right],\\
%&=3\frac{\gamma-1}{\gamma}\left[\frac{GM^2}{R}-2\int\limits_0^R\,\frac{G\mathcal{M}}{r}d\mathcal{M}\right].
%\end{align}
%Vemos que la última integral la podemos identificar con la expresión original para $V$, es decir,
%\begin{align}
% V&=3\frac{\gamma-1}{\gamma}\left[\frac{GM^2}{R}+2V\right],
%\end{align}
%y despejando dicha variable, encontramos una expresión explícita para la energía potencial gravitatoria en términos de la masa $M$, radio $R$ e índice politrópico $\gamma$ de la estrella:
%\begin{equation}\label{energia_potencial_newton:r1}
% \boxed{V=-3\left(\frac{\gamma-1}{5\gamma-6}\right)\frac{GM^2}{R}.}
%\end{equation}
%Para aplicaciones posteriores es útil reexpresar la relación anterior eliminando $R$ e introduciendo la densidad central $\rho_{\rm c}$, además de la función adimensional de Lane-Emden $\Theta$ que depende de $\gamma$. Con este objetivo se debe usar la relación \eqref{densidad0politropica}, despejando $1/R$:
%\begin{equation}
% \frac{1}{R}=\left(\frac{4\pi \rho_{\rm c}}{M}\frac{x_1^2\left\Vert\Theta'(x_1)\right\Vert}{x_1^3}\right)^{1/3},
%\end{equation}
%la cual se puede reemplazar en \eqref{energia_potencial_newton:r1}, obteniendo:
%\begin{equation}\label{energia_potencial_newton:r2}
% \boxed{V=-3\left(\frac{\gamma-1}{5\gamma-6}\right)\frac{\left(4\pi x_1^2\left\Vert\Theta'(x_1)\right\Vert\right)^{1/3}}{x_1}GM^{5/3}\rho_{\rm c}^{1/3}.}
%\end{equation}
%\item\emph{Forma 2: Sin asumir la ecuación de equilibrio hidrostático newtoniana \eqref{eqnewton}, ó en términos de variables adimensionales de Lane-Emden.}\\
%Partimos nuevamente de la definición de energía potencial gravitatoria como la integral \eqref{energia_potencial_newton1} (en donde sólo se ha usado la ecuación de masa), pero ahora escribiéndola en términos de las variables adimensionales de Lane-Emden $x$ y $\Theta$ dadas en \eqref{x} y \eqref{theta}. Además, $\mathcal{M}$ se puede reescribir usando \eqref{masapoli-en-r} como:
%\begin{equation}
% \mathcal{M}=\frac{x^2\left\Vert\Theta'(x)\right\Vert}{x_1^2\left\Vert\Theta'(x_1)\right\Vert}M.
%\end{equation}
%Luego, tendremos que:
%\begin{align}
%V&=-4\pi G\int_0^R \mathcal{M}\,\rho\,r\,dr\\
%&=-4\pi G\int_0^{x_1}\left[\frac{x^2\left\Vert\Theta'(x)\right\Vert}{x_1^2\left\Vert\Theta'(x_1)\right\Vert}M\right]\,\left[\rho_{\rm c}\,\Theta^{1/(\gamma-1)}\right](ax)\,d(ax)\\
%&=\frac{4\pi G M\rho_{\rm c}}{x_1^2\left\Vert\Theta'(x_1)\right\Vert}\,a^2\int\limits_0^{x_1}x^3\Theta'(x)\Theta(x)^{1/(\gamma-1)}\,dx.
%\end{align}
%Pero podemos usar \eqref{masaparcial-laneemden} (evaluada en $x=x_1$) para encontrar una expresión para la escala de longitud $a$ en términos de las variables antes se\~naladas:
%\begin{equation}
% a=\left(\frac{M}{4\pi\rho_{\rm c} x_1^2\left|\Theta'(x_1)\right|}\right)^{1/3}.
%\end{equation}
%De este modo, encontramos que,
%\begin{align}
% V=\left(4\pi\rho_{\rm c}\right)^{1/3}\frac{GM^{5/3}}{\left(x_1^2\left\Vert\Theta'(x_1)\right\Vert\right)^{5/3}}\int\limits_0^{x_1}x^3\Theta'\Theta^{1/(\gamma-1)}\,dx.
%\end{align}
%Integrando por partes la integral anterior es posible expresarla de una forma más compacta:
%\begin{align}
% \int\limits_0^{x_1}x^3\Theta'\Theta^{1/(\gamma-1)}\,dx&=\int\limits_0^{x_1}\frac{x^3}{1/(\gamma-1)+1}\frac{d}{dx}\left(\Theta^{1/(\gamma-1)+1}\right)\,dx\\
%&=\frac{\gamma-1}{\gamma}\left[\int\limits_0^{x_1}d\left(\Theta^{\gamma/(\gamma-1)}x^3\right)-\int\limits_0^{x_1}\Theta^{\gamma/(\gamma-1)}d(x^3)\right]\\
%&=\frac{\gamma-1}{\gamma}\left[[\cancelto{0}{\Theta(x_1)}x_1^3]^{\gamma/(\gamma-1)}-\lim_{x\to 0}\left(\Theta^{\gamma/(\gamma-1)}\cancelto{0}{x}^3\right)-3\int\limits_0^{x_1}x^2\Theta^{\gamma/(\gamma-1)}\,dx\right].
%\end{align}
%Por lo tanto, encontramos que el potencial gravitatorio se puede escribir también en la forma:
%\begin{align}\label{energia_potencial_politropicac}
% \boxed{V=-\left(4\pi\rho_{\rm c}\right)^{1/3}\frac{GM^{5/3}}{\left(x_1^2\left\Vert\Theta'(x_1)\right\Vert\right)^{5/3}}\,3\frac{\gamma-1}{\gamma}\int\limits_0^{x_1}x^2\Theta^{\gamma/(\gamma-1)}\,dx.}
%\end{align}
%También podemos expresar lo anterior en una forma que será útil en el siguiente capítulo, separando la parte dependiente de $\gamma$ de los demás parámetros:
%\begin{equation}\label{energia_potencial_newton-c}
%\boxed{V=-C_2\,M^{5/3}\,\rho_{\rm c}^{1/3}=-G\mathcal{C}_2(\gamma)\,M^{5/3}\,\rho_{\rm c}^{1/3}}
%\end{equation}
%en donde $C_2:=G\mathcal{C}_2$ y
%\begin{equation}\label{energia_coef_c2}
% \mathcal{C}_2(\gamma):=3\frac{\left(4\pi\right)^{1/3}}{x_1^2\left\Vert\Theta'(x_1)\right\Vert}\,\frac{\gamma-1}{\gamma}\int\limits_0^{x_1}x^2\Theta^{\gamma/(\gamma-1)}\,dx.
%\end{equation}
%
%
%Será útil también determinar el valor de la integral que aparece en la última ecuación en términos de funciones de Lane-Emden simples. Para ello, simplemente se igualan las relaciones para $V$ antes halladas en \eqref{energia_potencial_newton:r2} y \eqref{energia_potencial_politropicac}, obteniendo:
%\begin{align}
%%  V&=-\left(4\pi\rho_{\rm c}\right)^{1/3}\frac{GM^{5/3}}{x_1^2\left\Vert\Theta'(x_1)\right\Vert}\,3\frac{\gamma-1}{\gamma}\int\limits_0^{x_1}x^2\Theta^{\gamma/(\gamma-1)}\,dx=-3\left(\frac{\gamma-1}{5\gamma-6}\right)\frac{\left(4\pi x_1^2\left\Vert\Theta'(x_1)\right\Vert\right)^{1/3}}{x_1}GM^{5/3}\rho_{\rm c}^{1/3}\\
%\int\limits_0^{x_1}x^2\Theta^{\gamma/(\gamma-1)}\,dx=\frac{\gamma}{5\gamma-6}\frac{\left(x_1^2\left\Vert\Theta'(x_1)\right\Vert\right)^2}{x_1}.\label{integral_laneemden}
%\end{align}
%
%
%
%\end{enumerate}
%
%\subsubsection{Energía interna}
%Si definimos la energía interna por unidad de volumen (ó densidad de energía interna) como $u(r)=dU(r)/dV$, tendremos que la energía interna total de la estrella será:
%\begin{align}\label{energia_interna_newton}
% U&=\int\limits_0^R \,u(r)\,dV=\int\limits_0^R 4\pi r^2\,u(r)\,dr.
%\end{align}
%Reemplazando la ecuación de estado politrópica \eqref{estadopolitropica_alterna}, que relaciona $u$  con $P$,  en la relación anterior, tenemos
%\begin{align}\label{energia_interna1}
%U&=\frac{1}{\gamma-1}\int\limits_0^R P\, 4\pi r^2 \,dr.
%\end{align}
%
%\begin{enumerate}
% \item \emph{Forma 1: Usando la definición de energía potencial gravitacional}\\
%En \eqref{energia_interna1}, podemos identificar la expresión para la energía potencial gravitacional dada en \eqref{energia_potencial1}, de donde obtenemos la relación entre ambos tipos de energías:
%\begin{align}
%U&=-\frac{V}{3(\gamma-1)},\label{energia_int_pot}
%\end{align}
%con la cual, de la expresión explícita para $V$ dada en \eqref{energia_potencial_newton:r1}, encontramos finalmente para la energía interna de una estrella politrópica:
%\begin{equation}\label{energia_interna_newton:r1}
% \boxed{U=\frac{1}{5\gamma-6}\frac{GM^2}{R}.}
%\end{equation}
%
%\item \emph{Forma 2: Sin asumir la ecuación de equilibrio hidrostático, ó en términos de variables adimensionales.}\\
%Podemos expresar \eqref{energia_interna1} en función de $x$ y $\Theta$ mediante sus definiciones \eqref{x}, \eqref{theta} y la ecuación de estado politrópica:
%\begin{align}
% U&=\frac{1}{\gamma-1}\int\limits_0^R K\rho^{\gamma}\, 4\pi (ax)^2 \,d(ax)=\frac{4\pi a^3 K}{\gamma-1}\int\limits_0^{x_1}\left(\rho_{\rm c}\Theta^{1/(\gamma-1)}\right)^{\gamma}\,x^2\,dx\\
%&=\frac{\left(4\pi\rho_{\rm c}\, a^3\right)}{\gamma-1}K\rho_{\rm c}^{\gamma-1}\int\limits_0^{x_1}x^2\,\Theta^{\gamma/(\gamma-1)}\,dx.
%\end{align}
%Usando \eqref{masaparcial-laneemden} evaluada en $x=x_1$, es posible reescribir la cantidad entre paréntesis anterior en términos de $M$, y la integral se puede calcular en función de \eqref{integral_laneemden}, de modo que:
%\begin{align}
% U=\frac{M}{x_1^2\left|\Theta'(x_1)\right|}\frac{K\rho_{\rm c}^{\gamma-1}}{\gamma-1}\left[\frac{\gamma}{5\gamma-6}\frac{(x_1^2\left|x_1^2\Theta'(x_1)\right|)^2}{x_1}\right].
%\end{align}
%Entonces, tenemos que la energía interna total será:
%\begin{align}\label{energia_interna_newton:r2}
% \boxed{U=\frac{\gamma/(\gamma-1)}{5\gamma-6}\,\frac{x_1^2\left|\Theta'(x_1)\right|}{x_1}\, KM\,\rho_{\rm c}^{\gamma-1}.}
%\end{align}
%
%De la misma forma que en el caso anterior, podemos separar esta expresión en una parte dependiente de $\gamma$ y otra de los parámetros:
%\begin{equation}\label{energia_interna_newton-c}
% \boxed{U=C_1\,M\,\rho_{\rm c}^{\gamma-1}=K\,\mathcal{C}_1(\gamma)\,M\,\rho_{\rm c}^{\gamma-1}.}
%\end{equation}
%en donde $C_1:=K\mathcal{C}_1(\gamma)$ y
%\begin{equation}\label{energia_coef_c1}
% \mathcal{C}_1(\gamma):=\frac{\gamma/(\gamma-1)}{5\gamma-6}\,\frac{x_1^2\left|\Theta'(x_1)\right|}{x_1}.
%\end{equation}
%
%\end{enumerate}
%
%
%\subsubsection{Energía total}
%De las relaciones anteriores \eqref{energia_interna_newton:r1} y \eqref{energia_potencial_newton:r1} podremos determinar una expresión para una estrella newtoniana que obedece la ecuación de estado \eqref{estadopolitropica} con índice politrópico $\gamma$:
%\begin{align}\label{energia_total_newtoniana}
% \boxed{E=U+V=(4-3\gamma)U=-\frac{3\gamma-4}{5\gamma-6}\frac{GM^2}{R}.}
%\end{align}
%De aquí es posible deducir que la energía de la estrella será negativa si $\gamma>4/3$, mientras que $E>0$ si $6/5<\gamma<4/3$\footnote{Recordar que el caso $\gamma<6/5$ no es físicamente relevante, pues la estrella tendría radio infinito.}.
%
%La energía total también se puede obtener sumando las expresiones alternas para las energías interna y potencial gravitatoria \eqref{energia_interna_newton-c} y \eqref{energia_potencial_newton-c}:
%\begin{equation}\label{energia_newton}
%\boxed{ E=U+V=C_1\, M\rho_{\rm c}^{\gamma-1}-C_2\,M^{5/3}\rho_{\rm c}^{1/3}.}
%\end{equation}
%
%\subsection{Ecuación de equilibrio hidrostático de un principio variacional}
%
%
%
%En esta sección, se probará que el hecho que la energía $E$ de una estrella compuesta de un fluido ideal (con composición química y entropía por barión constante) sea extremal (ó un punto crítico, ó estacionario), es equivalente a la condición de equilibrio hidrostático dada por la ecuación \eqref{eqnewton}. Este resultado se aplicará en la siguiente sección para obtener configuraciones de equilibrio a partir de este método variacional. Para ello, debemos considerar la energía total de la estrella $E$ en términos de la coordenada radial como función de la masa parcial al interior de dicho radio, es decir, $r=r(\mathcal{M})$, un supuesto válido debido al hecho que dichas cantidades tienen una relación 1-1 (en particular, $\mathcal{M}$ es monónotamente creciente con $r$). De esta forma, se debe tomar la variación $\delta$ de la energía total manteniendo constante $\mathcal{M}$, para luego sumar las expresiones generales de $U$ y $V$ dadas en \eqref{energia_interna_newton} y \eqref{energia_potencial_newton}, respectivamente:
%\begin{align}
% \delta E&=\delta U+\delta V\\
%&=\delta \int_0^M u\;dV-G\int\limits_0^M \mathcal{M}(r)\,d\mathcal{M}\,\delta\left(\frac{1}{r}\right)\\
%&=\int_0^M \delta \left[\frac{u(r)}{\rho}\right]\;d\mathcal{M}+G\int\limits_0^M \frac{\mathcal{M}}{r^2}\,\delta r\,d\mathcal{M}.\label{energia_gral_0}
%\end{align}
%Usando la definición de presión \eqref{def_presion} de la primera ley de la Termodinámica (pero con variaciones $\delta$  en vez de diferenciales $d$), y recordando que en el caso newtoniano $\rho\approx\rho_0=m_B\, n$, podemos reescribir la integral para $\delta U$ como:
%\begin{align}
%\delta U&=\frac{1}{m_B}\int_0^M \delta \left[\frac{u}{n}\right]\;d\mathcal{M}\\
%&=\frac{1}{m_B}\int_0^M \left[-P\delta\left(\frac{1}{n}\right)\right]\;d\mathcal{M}\\ &=\int_0^M \left[-P\delta\left(\frac{1}{\rho}\right)\right]\;d\mathcal{M}.
%\end{align}
%La variación $\delta \rho^{-1}$ se puede simplificar de la ecuación de masa \eqref{masa1}, ya que:
%\begin{equation}
% \frac{d\mathcal{M}}{dr}=4\pi r^2\rho\quad\Rightarrow\quad \frac{1}{\rho}=4\pi r^2\frac{dr}{d\mathcal{M}},
%\end{equation}
%y así es posible separar la integral para $U$ en dos términos:
%\begin{align}
%\delta U&=-\int_0^M P\delta\left(4\pi r^2\frac{dr}{d\mathcal{M}}\right)\;d\mathcal{M}\\
%&=-\int_0^M 8\pi P\frac{dr}{d\mathcal{M}}r\delta r\,d\mathcal{M}-\int_0^{M}4\pi P r^2\delta\left(\frac{dr}{d\mathcal{M}}\right)d\mathcal{M}\\
%&=\int_0^M 4\pi \delta r\left(-2Pr\frac{dr}{d\mathcal{M}}\right)\,d\mathcal{M}-\int_0^{M}4\pi P r^2\,d\left(\frac{\delta r}{d\mathcal{M}}\right)d\mathcal{M}
%\end{align}
%en donde en el último término se han intercambiado el diferencial $d$ con la variación $\delta$ (derivada funcional). Integrando por partes este último término, y recordando que en el borde $r=R$ de la estrella se debe tener $P(R)=0$, tenemos que:
%\begin{align}
% \delta U&=\int_0^M 4\pi \delta r\left(-2Pr\frac{dr}{d\mathcal{M}}\right)\,d\mathcal{M}-\int_0^{M}4\pi\left\{ \frac{d}{d\mathcal{M}}\cancelto{0}{\left(P r^2\,\delta r\right)}d\mathcal{M} -\frac{d}{d\mathcal{M}}\left(Pr^2\right)\delta r\,d\mathcal{M}\right\}\\
%&=\int_0^M 4\pi \delta r\,d\mathcal{M}\left\{-2Pr\frac{dr}{d\mathcal{M}}+\frac{d}{d\mathcal{M}}\left(Pr^2\right)\right\}\\
%&=\int_0^M 4\pi r^2 \frac{dP}{d\mathcal{M}}\delta r\,d\mathcal{M}.\label{energia_gral_1}
%\end{align}
%Usando la ecuación de masa parcial \eqref{masa1}, podemos ver que la derivada $dP/d\mathcal{M}$ se puede reescribir como:
%\begin{equation}
% \frac{d\mathcal{M}}{dr}=4\pi r^2\rho\quad\Rightarrow\quad\frac{dP}{d\mathcal{M}}=\frac{dP}{dr}\frac{dr}{d\mathcal{M}}=\frac{dP}{dr}\frac{1}{4\pi r^2 \rho},
%\end{equation}
%y de este modo, la energía interna \eqref{energia_gral_1} queda expresada como:
%\begin{align}
% \delta U&=\int_0^M\frac{1}{\rho}\frac{dP}{dr}\,\delta r\,d\mathcal{M}.
%\end{align}
%Entonces, reemplazando lo anterior en la relación original para la variación de la energía de la estrella \eqref{energia_gral_0}, obtenemos:
%\begin{align}
%\delta E&=\int_0^M\frac{1}{\rho}\frac{dP}{dr}\,\delta r\,d\mathcal{M}+G\int\limits_0^M \frac{\mathcal{M}}{r^2}\,\delta r\,d\mathcal{M}\\
%&=\int_0^M\left\{\frac{1}{\rho}\frac{dP}{dr}+\frac{G\mathcal{M}}{r^2}\right\}\,\delta r\,d\mathcal{M}.
%\end{align}
%Por consiguiente, la condición $\delta E=0,\;\;\forall \delta r$, implica
%\begin{align}
%\frac{dP}{dr}&=-\frac{G\mathcal{M}\rho}{r^2}.
%\end{align}
%Así, vemos que la relación anterior es idéntica a la ecuación de equilibrio hidrostático newtoniano \eqref{eqnewton}.
%
%
%
%
%
%\section{Condiciones de estabilidad *}
%
%Todas las soluciones de las ecuaciones de estructura estelar antes halladas representan estados de equilibrio, pero son físicamente relevantes sólo si son estables. Ahora bien, para determinar si una determinada configuración estelar es estable, existen varios métodos y criterios que se detallarán a continuación.
%
%
%
%\subsection{Criterio de transición entre configuraciones estables e inestables}\label{sec:criterio_transicion}
%El método más directo consiste en hallar el comportamiento de una configuración de equilibrio frente a peque\~nas perturbaciones (lineales), determinando las frecuencias propias$\omega_n$ de sus modos normales. Aún sin efectuar el análisis detallado (que se mostrará en la siguiente sección), es posible establecer un criterio de estabilidad frente a oscilaciones radiales que involucra únicamente la solución de equilibrio. Para ello, supondremos que hemos construido una secuencia de configuraciones estelares de equilibrio con la misma ecuación de estado politrópica, pero a diferentes densidades centrales $\rho_{\rm c}$ (es decir, dependientes de un sólo parámetro).
%
%\emph{Una estrella compuesta de un fluido ideal, con composición química y entropía por nucleón constante, pasa de estabilidad a inestabilidad con respecto a un modo normal radial (es decir, una perturbación en la densidad del tipo $\delta \rho=\delta\rho(r,t)$) para un valor de la densidad central $\rho_{\rm c}$ en que la energía total $E$ y el número de bariones $N$ sean estacionarios}\footnote{En rigor, esto es válido si se desprecian reacciones nucleares, viscosidad, conducción de calor y transferencia de energía radiativa}:
%\begin{equation}\label{criterio_transicion}
%\boxed{ \frac{\partial E}{\partial \rho_{\rm c}}=0\quad\text{y}\quad \frac{\partial N}{\partial \rho_{\rm c}}=0. }
%\end{equation}
%Esto puede entenderse debido al hecho que las ecuaciones dinámicas que gobiernan las perturbaciones radiales (que son una linealización de las ecuaciones de estructura) son invariantes bajo inversión temporal $t\to-t$ (no hay fuerzas disipativas), por lo que las frecuencias de los modos normales que se obtienen quedan en términos de $\omega_n^2$, correspondiente al modo normal $n$. Ahora, si la ecuación de estado politrópica es conocida (con el índice $\gamma$ constante), cada modelo estelar es parametrizado por la densidad central, por lo que tendremos que la frecuencia es una función real continua de dicho parámetro:
%\begin{equation}
% \omega_n^2=\omega_n^2(\rho_{\rm c}).
%\end{equation}
%Así, tenemos dos situaciones posibles:
%\begin{enumerate}
% \item Si $\omega_n^2>0$, entonces las frecuencias serán reales y la estrella oscilará con una dependencia armónica en el tiempo del tipo:
%\begin{equation}\label{modelo-estable}
%\delta\rho\propto\exp(-i\omega_n t)=\exp(\pm i|\omega_n|t),
%\end{equation}
%es decir, este caso corresponde a una configuración \emph{estable}.
%\item Si $\omega_n^2<0$, entonces las frecuencias serán complejas y conjugadas. El modo con parte imaginaria positiva tendrá una dependencia de la forma:
%\begin{equation}
%\delta\rho\propto\exp(-i\omega_n)=e^{-i\mathfrak{Re}(\omega_n)t} e^{+|\mathfrak{Im}(\omega_n)|t},
%\end{equation}
%lo que implica un crecimiento exponencial en el tiempo, claramente correspondiendo a una inestabilidad. En cambio, el modo con parte imaginaria positiva tendrá una dependencia de la forma
%\begin{equation}
%\delta\rho\propto\exp(-i\omega_n)=e^{-i\mathfrak{Re}(\omega_n)t} e^{-|\mathfrak{Im}(\omega_n)|t},
%\end{equation}
%lo que implica una oscilación exponencialmente amortiguada. Ahora, en una solución general siempre estarán presentes ambos signos de la parte compleja de $\omega_n$, si bien con una condición dada particular el sistema físico se determinará por una de ellas. Por lo tanto, al considerar el conjutno de todas estas posibilidades, se dirá que el caso con $\omega_n^2<0$ representa una configuración estelar \emph{inestable}.
%
%\end{enumerate}
%Luego, por continuidad, la transición de una solución estable a una inestable corresponderá al valor de $\rho_{\rm c}$ en que $\omega_n^2=0$. Ahora bien, un valor de $\omega_n$ cercano a $0$ implicará una frecuencia de oscilación ó tasa de crecimiento de dicho modo muy largo, por lo que el estado dinámico de la configuración cerca del punto de cambio de estabilidad a inestabilidad se puede considerar cuasi-adiabático (los cambios en las variables ocurren muy lentamente). Así, si la configuración $\rho(r)$ es de equilibrio, $\rho(r)+\delta\rho(r)$ también lo será, teniendo esta última la misma energía $E$ y número de bariones $N$ que la primera, por las leyes de conservación respectivas. Entonces, el único parámetro que varía entre las configuraciones anteriores es la densidad central $\rho_C$, pues si se tuviera $\delta\rho_C=0$ implicaría el resultado trivial $\delta\rho(r)=0$ y no habrían oscilaciones. De esta forma, en el punto de transición de estabilidad a inestabilidad hay configuraciones vecinas de equilibrio con \emph{distintos} valores de $\rho_C$ pero con los \emph{mismos} $E$ y $N$, que equivale a la afirmación que ambas cantidades son estacionarias allí (ver \eqref{criterio_transicion}).
%
%Para el caso newtoniano, tendremos que si $m_u$ representa la masa típica de un barión, entonces la masa total $M$ de la estrella vendrá dada por $M=m_u N$, por lo que la segunda condición del criterio \eqref{criterio_transicion} equivale simplemente a que la función $M=M(\rho_{\rm c})$ tenga un punto crítico en la transición:
%\begin{equation}
%\boxed{\frac{\partial M}{\partial \rho_{\scriptscriptstyle C}}=0.}
%\end{equation}
%
%\subsection{Criterio de estabilidad estático}
%Si consideramos, al igual que en la subsección anterior, una familia de modelos estelares parametrizados por la densidad central de modo tal que $M=M(\rho_{\rm c})$ y $R=R(\rho_{\rm c})$, entonces es posible demostrar que una condición \emph{necesaria} para la estabilidad de un modelo estelar, en el sentido de la ecuación \eqref{modelo-estable}), es:
%\begin{align}\label{criterio-estabilidad-estatico}
%\frac{\partial M}{\partial \rho_{\rm c}}>0,
%\end{align}
%mientras que la desigualdad contraria \emph{siempre} implica la inestabilidad de una configuración. Esto es plausible si pensamos que el agregar masa a una estrella provoca su contracción, lo que implica una presión mayor para que contrarreste la fuerza gravitacional incrementada y así pueda seguir en un estado de equilibrio. Y lógicamente, una mayor presión implicará también una densidad central más grande, de donde es claro que la masa total $M$ de la estrella debe ser creciente con respecto a $\rho_{\rm c}$.
%
%Existe otro criterio similar para diagramas masa-radio que se obtiene a partir del análisis de perturbaciones Lagrangianas.
%
%\subsection{Criterio para el índice adiabático}\label{sec:criterio_indicegamma}
%Ahora aplicaremos el criterio \eqref{criterio_transicion} para determinar si una estrella con ecuación de estado politrópica es estable. De la relación \eqref{energia_total_newtoniana} para $E=E(\rho_{\rm c})$ y \eqref{masalaneemden} para $M=M(\rho_{\rm c})$, podemos ver que sus puntos críticos simultáneos satisfacerán:
%\begin{align}
% \frac{\partial E}{\partial \rho_{\rm c}}=\left(\frac{5\gamma-6}{2}\right)\rho_{\rm c}^{\frac{5\gamma-6}{2}-1}=0\quad\Leftrightarrow\quad 5\gamma-6=0,\\
%\frac{\partial M}{\partial \rho_{\rm c}}=\left(\frac{3\gamma-4}{2}\right)\rho_{\rm c}^{\frac{3\gamma-4}{2}-1}=0\quad\Leftrightarrow\quad 3\gamma-4=0,
%\end{align}
%de donde se ve que no hay solución para ningún valor de $\rho_{\rm c}$. Esto implica que la estrella no pasará de estabilidad a inestabilidad para ningún valor de $\rho_{\rm c}$, sino que será o estable o inestable para todo valor de densidad central. Ahora bien, esto sólo podrá depender de la ecuación de estado, o en forma más precisa, del valor del índice politrópico $\gamma$. Para determinarlo, primero se encontrará el punto donde la energía $E$ se hace estacionaria con respecto a $\rho_{\rm c}$ manteniendo $M$ constante (aplicando el resultado de la sección \ref{sec:ppio_variacional_newton}), para lo cual usamos la expresión antes hallada en \eqref{energia_newton} para $E$.
%% en el caso simple en que la densidad es \emph{constante}: $\rho=\rho_C=cte$. Así, en primer lugar, debemos usar la ecuación de estado \eqref{estadopolitropica_alterna} que relaciona linealmente $P$ con $\rho$ (por lo que la presión también será constante), en la definición de energía interna \eqref{energia_interna_newton}:
%% \begin{equation}\label{energia_interna2}
%%  U=\frac{4\pi}{3(\gamma-1)}\int\limits_0^R P\,r^2\,dr=\frac{4\pi}{3(\gamma-1)}PR^3=\frac{4\pi K}{3(\gamma-1)}\rho^{\gamma}R^3
%% \end{equation}
%% Por otra parte, podemos expresar la masa al interior del radio $r$: $\mathcal{M}(r)$ en función de $\rho=cte$, ya que que por su definición \eqref{masa1}:
%% \begin{equation}
%%  \mathcal{M}(r)=\int\limits_0^r 4\pi r'^2\rho dr'=\frac{4\pi}{3}\rho r^3,
%% \end{equation}
%%  y así podemos encontrar la energía potencial gravitatoria definida por \eqref{energia_potencial_newton1} en términos de $\rho$ y $R$:
%% \begin{align}\label{energia_potencial2}
%%  V:=-4\pi G\int_0^R r'\left(\frac{4\pi}{3}\rho r'^3\right)\rho\,dr=-\frac{16\pi^2}{15}G\rho^2 R^5
%% \end{align}
%% Pero la idea es encontrar un extremo de la energía bajo la restricción de un número de bariones $N$ constante. Esto implica que la masa de la estrella sea constante, de donde es posible expresar el radio $R$ en términos de $\rho$ y $M$ de la siguiente forma:
%% \begin{align}
%%  M=\rho V=\frac{4\pi}{3}\rho R^3=m_u N=cte\qquad\Rightarrow\qquad R=\left(\frac{3M}{4\pi\rho}\right)^{1/3}
%% \end{align}
%% De esta forma, sumando \eqref{energia_interna2} y \eqref{energia_potencial2} y eliminando $R$ mediante la relación anterior, podemos obtener la energía total sólo en términos del parámetro $\rho$:
%% \begin{align}
%%  E&=U+V\\
%% &=\frac{4\pi K}{3(\gamma-1)}\rho^{\gamma}\left(\frac{3M}{4\pi\rho}\right)-\frac{16\pi^2}{15}G\rho^2 \left(\frac{3M}{4\pi\rho}\right)^{5/3}\\
%% &=\underbrace{\frac{KM}{\gamma-1}}_{:=a}\rho^{\gamma-1}-\underbrace{\frac{3}{5}\left(\frac{4\pi}{3}\right)^{1/3}GM^{5/3}}_{:=b}\rho^{1/3}\label{energia_defs_ayb}
%% \end{align}
%Sus puntos críticos con respecto a $\rho_C$ y $M=cte$ serán:
%\begin{align}
% \left.\frac{\partial E}{\partial \rho_{\rm c}}\right|_{\rho_{\rm c,eq}}&=(\gamma-1)C_1 M\rho_{\rm c,eq}^{\gamma-2}-\frac{1}{3}C_2M^{5/3}\rho_{\rm c,eq}^{-2/3}=0 \label{energia_pto_critico1}\\
%\Rightarrow\quad\rho_{\rm c,eq}&=\left(\frac{C_2 M^{2/3}}{3(\gamma-1)C_1}\right)^{1/(\gamma-4/3)}.\label{energia_pto_critico2}
%\end{align}
%Este valor de la densidad central representa una configuración de equilibrio. Para determinar si es estable o no, debemos evaluar el signo de la segunda derivada de $E=E(\rho_{\rm c})$ en dicho punto. Así, usando \eqref{energia_pto_critico1}, tenemos:
%\begin{align}
% \left.\frac{\partial^2 E}{\partial^2 \rho_{\rm c}}\right|_{\rho_{\rm c,eq}}&=(\gamma-1)(\gamma-2)C_1M\rho_{\rm c,eq}^{\gamma-3}+\frac{1}{3}\frac{2}{3}\,C_2M^{5/3}\rho_{\rm c,eq}^{-5/3}\\
%&=(\gamma-2)\left(\frac{1}{3}C_2M^{5/3}\rho_{\rm c,eq}^{-5/3}\right)+\frac{2}{3}\left(\frac{1}{3}C_2M^{5/3}\rho_{\rm c,eq}^{-5/3}\right)\\
%&=\left(\gamma-\frac{4}{3}\right)\left(\frac{1}{3}C_2M^{5/3}\right)\rho_{\rm c,eq}^{-5/3}\\
%&=\left(\gamma-\frac{4}{3}\right)\left(\frac{1}{3}C_2M^{5/3}\right)\left(\frac{C_2 M^{2/3}}{3(\gamma-1)C_1}\right)^{(5/3)/(4/3-\gamma)}.
%\end{align}
%Por lo tanto, dado que los dos últimos paréntesis del lado derecho son definidos positivos en el caso físicamente relevante de $\gamma>6/5$, tendremos que los distintos casos para el punto de equilibrio $\rho_{\rm c,eq}$ se pueden resumir en el siguiente criterio de estabilidad:
%\begin{itemize}
% \item Si $\gamma>4/3$, entonces $\left.\frac{\partial^2 E}{\partial^2 \rho_{\rm c}}\right|_{\rho_{\rm c,eq}}>0$ y $\rho_{\rm c,eq}$ será un mínimo de la energía, con lo que la configuración de equilibrio será estable.
%\item Si $\gamma<4/3$, entonces $\left.\frac{\partial^2 E}{\partial^2 \rho_{\rm c}}\right|_{\rho_{\rm c,eq}}<0$ y $\rho_{\rm c,eq}$ será un máximo de la energía, con lo que el equilibrio será inestable.
%\item $\gamma=4/3$ representará el punto de transición entre estabilidad e inestabilidad, pues $\left.\frac{\partial^2 E}{\partial^2 \rho_{\rm c}}\right|_{\rho_{\rm c,eq}}=0$. En este caso, la expresión original para la energía \eqref{energia_newton} se reducirá a
%\begin{equation}
% E=(C_1M-C_2M^{5/3})\rho^{1/3},
%\end{equation}
%con lo que el extremar obtendremos:
%\begin{align}
% \frac{\partial E}{\partial \rho_{\rm c}}&=\frac{1}{3}(C_1M-C_2M^{5/3})\rho_{\rm c}^{-2/3}=0\\
%\Rightarrow\quad C_1&=C_2 M^{2/3},
%\end{align}
%es decir, el equilibrio será neutral para todo valor de la densidad central. Además, la última ecuación predice un valor determinado para la masa de una estrella en equilibrio con $\gamma=4/3$, cuyo valor se puede determinar usando los valores de $C_1$ y $C_2$ dados en \eqref{energia_coef_c1} y \eqref{energia_coef_c2}, respectivamente:
%\begin{align}
% \label{masacritica_chandra}M&=\left(\frac{C_1}{C_2}\right)^{3/2}=\left[\frac{\frac{(4/3)/(4/3-1)}{5(4/3)-6}\,\frac{x_1^2\left|\Theta'(x_1)\right|}{x_1}\;K}{3\left(\frac{4/3-1}{5(4/3)-6}\right)\frac{\left(4\pi x_1^2\left\Vert\Theta'(x_1)\right\Vert\right)^{1/3}}{x_1}\;G}\right]^{3/2},\\
%&=4\pi\,x_1^2\left\Vert\Theta'(x_1)\right\Vert\left(\frac{K}{\pi G}\right)^{3/2}=M_{\rm ch},
%\end{align}
%en donde se ha encontrado nuevamente la masa de Chandrasekhar \eqref{masa4/3}, avalando la validez de este método variacional
%%
%% \footnote{La  última condición $a=b$ implica, en términos de variables físicas definidas \eqref{energia_defs_ayb}, la siguiente condición sobre la masa:
%% \begin{align}
%%  3KM&=\frac{3}{5}\left(\frac{4\pi}{3}\right)^{1/3}GM^{5/3}\\
%% \Rightarrow\quad M&=\left(\frac{5K}{G}\right)^{3/2}\left(\frac{4\pi}{3}\right)^{-1/2},
%% \end{align}
%% que es una aproximación a la masa límite de Chandrasekhar dada en \eqref{masachandra2}.}
%\end{itemize}
%Por otra parte, la densidad central en equilibrio obtenida en \eqref{energia_pto_critico2} nos permite encontrar otra justificación del criterio de estabilidad estático \eqref{criterio-estabilidad-estatico}, aunque válido sólo para estrellas politrópicas. En efecto, de dicha ecuación podemos verificar que obtenemos para la masa de una estrella politrópica el mismo resultado que el encontrado directamente de Lane-Emden \eqref{masalaneemden}, cuya dependencia con la densidad central es:
%\begin{align}
% M=cte(\gamma)\cdot\rho_{\rm c}^{\frac{3\gamma-4}{2}}.
%\end{align}
%% que tiene la misma dependencia con la densidad central que el resultado exacto $M_{ex}$ para estrellas politrópicas \footnote{De hecho, la razón entre la masa obtenida por este método, $M_{var}$, y el resultado anterior es:
%% \begin{equation}
%%  \frac{M_{var}}{M_{ex}}=\frac{15(\gamma-1)/\gamma}{3x_1^2\left\Vert\Theta'(x_1)\right\Vert},
%% \end{equation}
%% que es cercano a uno para valores de $\gamma$ físicamente relevantes}
%Por lo tanto, derivando con respecto a la densidad central, obtendremos que:
%\begin{equation}
% \frac{\partial M}{\partial \rho_{\rm c}}=\frac{\left(3\gamma-4\right)}{2}\cdot cte(\gamma)\cdot\rho^{(3\gamma-6)/2}.
%\end{equation}
%Así, dado que sabemos que $\gamma>4/3$ implica equilibrio estable, por la ecuación anterior esta condición equivaldrá al criterio antes mencionado \eqref{criterio-estabilidad-estatico}: $M$ debe ser creciente con respecto a $\rho_{\rm c}$.
%
%
%
%
%\section{Pulsación estelar newtoniana}
%Del mismo modo que el que se desarrollará en la sección de pulsación estelar relativista, es posible modelar newtonianamente las oscilaciones radiales de una estrella de masa $M$ en su modo fundamental, la que se considera compuesta de un fluido donde es válida una ecuación de estado politrópica. Se probará que el desplazamiento radial $\delta R$ obedece a una ecuación tipo oscilador armónico,
%\begin{equation}
% M\delta \ddot{R}=-K\delta R,
%\end{equation}
%en donde $K$ depende del índice politrópico promedio $\overline{\gamma} $. Mediante un análisis detallado, es posible resolver el problema anterior, encontrando que la frecuencia del modo fundamental $\omega$ está dada por:
%\begin{equation}
% \omega^2=\frac{K}{M}=3\left(\overline{\gamma}-\frac{4}{3}\right)\frac{|\Omega|}{I},
%\end{equation}
%con $\Omega$ la energía potencial gravitatoria de la estrella e $I$ su momento de inercia. Entonces, es claramente deducible que el límite entre configuraciones estables e inestables vendrá dado por $\bar{\gamma}=4/3$, coincidiendo con el criterio general para el índice politrópico establecido en la sección previa.
