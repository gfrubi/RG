\chapter[Nociones de Termodinámica]{Nociones básicas de Termodinámica relativista}\label{cap:termo}
\section{Primera ley de la Termodinámica}
El primer objetivo de esta sección será escribir la generalización relativista de la primera ley de la termodinámica newtoniana (en equilibrio),
\begin{equation}\label{1termoa}
dU=-PdV+TdS,
\end{equation}
en el contexto de Relatividad General. Aquí, $dU$ es la \textit{energía interna} total en el elemento de volumen $dV$ conteniendo un número (constante) $\mathcal{N}$ bariones, $P$ es la presión, $T$ la temperatura y $S$ la entropía total en dicho elemento de volumen. Para escribir la generalización apropiada de \eqref{1termoa} referimos las cantidades a un \textit{SR comóvil} con el elemento de fluido $dV$. Con este objetivo, definimos la \textit{densidad numérica propia de bariones} (los que contribuyen principalmente a la masa total de una estrella, por ejemplo) definida en este sistema de referencia:
\begin{equation}\label{densidadbarionica}
n=\frac{d\mathcal{N}}{dV}.
\end{equation}
Si la \textit{densidad de energía interna} total es $\epsilon$, 
%relacionada con la \textit{densidad de masa} total de la materia $\rho$ por
%\begin{equation}
%\rho\,c^2=\epsilon,
%\end{equation}
entonces es conveniente escribir todas las cantidades que aparecen en la ley de conservación de energía \eqref{1termoa} en función de $n$, ya que es una cantidad conservada. Es decir, en  términos de la \textit{densidad de energía interna por barión} $\epsilon/n$, el inverso de densidad de bariones $n^{-1}$ (volumen por barión), y la entropía por barión $s=S/\mathcal{N}$, tendremos que \eqref{1termoa} se generaliza a:

\begin{align}\label{1ley_termo_b}
\boxed{d\left(\frac{\epsilon}{n}\right)=-Pd\left(\frac{1}{n}\right)+Tds.}
\end{align}
% \begin{equation}
% u=\frac{dU}{dV}\quad\Rightarrow\quad U=\frac{u A}{n},
% \end{equation}
% Por lo tanto, si la entropía por barión es $s$ la ecuación \ref{1termoa} se puede expresar como:
Si separamos explícitamente los diferenciales y agrupamos, notamos que
\begin{align}
\frac{d\epsilon}{n}+\epsilon d\left(\frac{1}{n}\right)+Pd\left(\frac{1}{n}\right)&=Tds\\
\frac{d\epsilon}{n}+(\epsilon +P)\left(-\frac{1}{n^2}\,dn\right)=Tds.
\end{align}
Así, encontramos la siguiente relación proveniente de la primera ley de la Termodinámica relativista:
\begin{equation}\label{1leytermo}
\boxed{d\epsilon=\frac{\epsilon+P}{n}dn+nT\,ds.}
\end{equation}
La ecuación \eqref{1ley_termo_b} también se puede reescribir en términos de la densidad de energía interna $u$. Para ello, primero definimos la \textit{densidad de masa en reposo} $\rho$ en términos de la densidad numérica de bariones:
\begin{equation}\label{masa_reposo}
\rho_0=m_B\,n
\end{equation}
en donde $m_B$ es la masa en reposo característica de un barión.
%, que para efectos prácticos, se puede considerar igual a la unidad de masa atómica $m_u=1.66\cdot 10^{-27}\,[kg]$. 
De este modo, dado que la densidad de masa total $\rho$ incluye tanto la energía interna como la energía en reposo, tendremos:
\begin{equation}\label{energia_interna_y_densidad}
\boxed{u=\rho_0 c^2+\epsilon \quad\Leftrightarrow\quad \epsilon=u-\rho_0c^2=u-m_B nc^2}
\end{equation}
Entonces, dado que
\begin{equation}
d\left(\frac{\epsilon}{n}\right)=d\left(\frac{u-m_B\,n\,c^2}{n}\right)=d\left(\frac{u}{n}\right)-\cancelto{0}{d(m_B\,c^2)},
\end{equation}
tendremos que la ley \eqref{1ley_termo_b} equivale a:
\begin{align}
% d\left(\frac{u \cancel{A}}{n}\right)&=-Pd\left(\frac{\cancel{A}}{n}\right)+Td(\cancel{A}s)\\
\boxed{d\left(\frac{u}{n}\right)=-Pd\left(\frac{1}{n}\right)+Tds.}\label{1ley_termo}
\end{align}
De lo anterior es posible encontrar el análogo de \eqref{1leytermo}, pero usando $u$ en lugar de $\rho$:
\begin{align}
  du=\frac{u+P}{n}dn + nT\,ds.
\end{align}
También podemos notar de las dos formas de la primera ley en \eqref{1ley_termo} y \eqref{1ley_termo_b}, que la presión y temperatura se pueden expresar como:
\begin{align}
 P&:=-\frac{\partial (\epsilon/n)}{\partial(1/n)}=-\frac{\partial (u/n)}{\partial(1/n)} \label{def_presion},\\
T&:=\frac{\partial (\epsilon/n)}{\partial s}=\frac{\partial (u/n)}{\partial s}\label{def_temperatura}.
\end{align}
% Esta ecuación es conveniente interpretarla del siguiente modo, en vista de la discusión sobre ecuaciones de estado desarrolladas en la sección (FALTA)
% \begin{equation}
% \rho(n,s)=\rho[P(n,s),T(n,s)].
% \end{equation}
% Es decir, se puede deducir $P$ y $T$ a partir de $\rho$

\section{Ecuación de estado adiabática o politrópica}
Una ecuación de estado adiabática o politrópica es aquella en que la presión $P$ y la densidad numérica de bariones $n$ (ó la densidad de masa en reposo $\rho_0$) se relacionan a través de
\begin{equation}\label{estadopolitropica_general}
 P=K'n^{\gamma}\qquad\Leftrightarrow\qquad P=K\rho_0^{\gamma},
\end{equation}
en donde $K$ y $\gamma$ son constantes, siendo esta última el \textit{índice adiabático ó politrópico}. Si $f$ es el número de grados de libertad de las partículas constituyentes del fluido del que se compone la estrella, entonces $\gamma=(f+2)/f$. 

Es posible mostrar que esta relación equivale a una ecuación de estado que relacione la energía interna $u$ con la presión $P$, sólo suponiendo que sean proporcionales en la forma
\begin{equation}\label{estadopolitropica_alterna}
\boxed{ u=\frac{P}{\gamma-1},}
\end{equation}
y además, que la entropía por barión $s$ sea constante. En efecto, consideremos la primera ley de la termodinámica en la forma \eqref{1ley_termo} y reemplacemos la relación anterior:
\begin{align}
 \frac{1}{\gamma-1}\left[d\left(\frac{P}{n}\right)\right]+P\,d\left(\frac{1}{n}\right)&=0,\\
\frac{1}{\gamma-1}\left[\left(\frac{1}{n}\right)dP+P\,d\left(\frac{1}{n}\right)\right]+P\,d\left(\frac{1}{n}\right)&=0,\\
\frac{1}{\gamma-1}\left[\left(\frac{1}{n}\right)dP+\gamma P\, d\left(\frac{1}{n}\right)\right]&=0.
\end{align}
Multiplicando por $1/n^{\gamma-1}$, podemos escribir
\begin{align}
\frac{1}{n^{\gamma}}dP+\frac{\gamma P}{n^{\gamma-1}}\,d\left(\frac{1}{n}\right)&=0,\\
d\left(\frac{P}{n^{\gamma}}\right)&=0,
\end{align}
que, tal como se quería probar, equivale a \eqref{estadopolitropica_general}. En el caso newtoniano, se tiene que la energía interna es mucho menor que la densidad de masa en reposo $\epsilon\ll \rho_0 c^2$, por lo que de \eqref{energia_interna_y_densidad} tendremos
\begin{equation}
u\approx \rho_0c^2=m_Bn\,c^2,
\end{equation}
de donde recuperamos la forma usual de la ecuación de estado politrópica
\begin{equation}\label{estadopolitropica}
 \boxed{P=K\rho^{\gamma}.}
\end{equation}

\section{Involucrando la Temperatura y Entropía}

En principio, el requisito mínimo necesario para una ecuación de estado del tipo $P=P(\rho)$ (usada para resolver el sistema de ecuaciones de estructura estelar), se basa en conocer cómo se relacionan la presión y la densidad a partir de un parámetro, que por simplicidad se escoge como la densidad numérica de bariones $n$, pues en dicho caso recuperamos (\ref{estado}):
\begin{equation*}
 P=P(n),\qquad\rho=\rho(n)\quad\Rightarrow\quad n=n(\rho)\qquad\Rightarrow\quad P=P(n(\rho))=P(\rho).
\end{equation*}
Sin embargo, no se puede deducir en general $P$ ni $\rho$ a partir únicamente de un conocimiento de $n$, sino que se requiere además la temperatura, $T$, o la entropía por barión, $s$. De este modo, de las leyes de la termodinámica es posible determinar todas las cantidades termodinámicas restantes, a través de las ecuaciones de estado más generales:
\begin{align}\label{estadogeneral}
 &P=P(n,s),\qquad\rho=\rho(n,s)\qquad\text{ó}\\
&P=P(n,T),\qquad\rho=\rho(n,T).
\end{align}

Sin embargo, para poder pasar de las ecuaciones anteriores, que involucran cantidades desconocidas como la temperatura  y la entropía, a la ecuación estándar (\ref{estado}), se necesita información sobre las propiedades térmicas de la estrella. Por ejemplo, se puede considerar los efectos térmicos en la ecuación de estado, al no  despreciar la temperatura de una estrella, de modo que:
\begin{equation}\marginnote{Efectos de la temperatura}
 P(r)=P(\rho(r),T(r)).
\end{equation}
Pero ahora necesitamos otra ecuación adicional que ligue el campo escalar temperatura en función del resto de las variables. Usualmente, se escoge para este propósito la ecuación de energía $E$, de modo de obtener la relación deseada entre $P$ y $\rho$,
\begin{align}
 E(r)&=E(\rho(r),T(r)),
\end{align}
con $E$ conocida. Entonces, podemos escribir:
\begin{align}
T(r)&=T(E(r),\rho(r))=T(\rho(r)),\\
P(r)&=P(\rho(r),T(\rho(r)))=P(\rho(r)).
\end{align}

Pero esta última ecuación de estado también será útil para modelar estrellas que están en la aproximación \marginnote{Tipos de estrellas donde es válido $P=P(\rho)$}del cero absoluto, en las cuales la temperatura no juega ningún rol relevante, tales como enanas blancas o estrellas de neutrones. Para este caso, de acuerdo al \textit{teorema de Nerns't}, la entropía por nucleón $s$ será constante por toda la estrella, lo que es consistente con la reducción de las ecuaciones de estado generales (\ref{estadogeneral}) a (\ref{estado}):
\begin{equation}
 P=P(n,s=cte)=P(n),\qquad\rho=\rho(n,s=cte)=\rho(n)\qquad\Rightarrow P=P(\rho).
\end{equation}
Puede probarse\footnote{Ver Weinberg \cite{Weinberg72}.} también que la condición de entropía constante, que según la relación anterior siempre conducirá a una ecuación de estado que relacione la presión y la densidad, se da también en estrellas en \textit{equilibrio convectivo}, en las cuales el mecanismo más eficiente para transferencia de energía al interior de la estrella es convección. Generalmente, esta condición se produce en estrellas supermasivas.

\section[Ecuaciones de estado de Fermi]{Ecuaciones de estado para un gas de Fermi completamente degenerado}\label{sec:ecsdeestado}
\subsection{Definiciones estadísticas}
En esta sección se usarán resultados de la teoría cinética para encontrar ecuaciones de estado apropiadas para modelar dos tipos de estrellas en las etapas finales de su evolución: \textit{enanas blancas} y \textit{estrellas de neutrones}.

Consideremos en primer lugar la \textit{densidad numérica en el espacio fase} 6-D para cada especie de partícula:
\begin{equation}
 \frac{d\mathcal{N}}{d^3x\,d^3p}=\frac{g}{h^3}f(\vec{x},\vec{p},t)
\end{equation}
en donde $f(\vec{x},\vec{p},t)$ es la \textit{función de distribución adimensional en el espacio fase} que da el número de ocupación promedio en una celda de dicho espacio, $h$ es la constante de Planck de modo que $h^3$ sea el volumen de una celda en el espacio fase, y $g$ es el peso estadístico  (número de estados de una partícula con un valor dado de momentum $\vec{p}$). Para partículas masivas, $g=2S+1$ ($S$ es el spin), para fotones $g=2$ y para neutrinos $g=1$.

Con esta definición, la densidad numérica $n$ de cada especie de partículas será:
\begin{equation}\label{densidad_dist}
 n=\int\frac{d\mathcal{N}}{d^3x\,d^3p}d^3p=\frac{g}{h^3}\int f\,d^3p,
\end{equation}
en donde la integral es sobre todo el espacio de los momenta $\vec{p}$. Por otra parte, la densidad de energía $\epsilon$ estará dada por
\begin{equation}\label{energia_dist}
 u=\int E\frac{d\mathcal{N}}{d^3x\,d^3p}d^3p=\frac{g}{h^3}\int \left(p^2c^2+m^2c^4\right)^{1/2}f\,d^3p,
\end{equation}
en donde $E=\sqrt{p^2 c^2+m^2 c^4}$ es la energía relativista de las partículas con masa en reposo $m$. Por otra parte, la presión $P$ para una \textit{distribución isotrópica de momenta} será:
\begin{align}
 P&=\frac{1}{3}\int pv\left(\frac{d\mathcal{N}}{d^3x\,d^3p}\right)d^3p=\frac{g}{3h^3}\int \frac{p^2c^2}{E}f\,d^3p
 =\frac{g}{3h^3}\int p\frac{dE}{dp}f\,d^3p\label{presion_dist2}
\end{align}
en donde $v=pc^2/E$ y el factor $\frac{1}{3}$ proviene de la isotropía considerada y el principio de equipartición. Finalmente, la densidad de masa en reposo $\rho$ se puede definir de dos formas para los casos de interés tratados:
\begin{enumerate}
 \item \emph{Electrones (en enanas blancas)}:
\begin{equation}\label{densidad_electrones}
\rho_0=\mu_e  m_u n_e,
\end{equation}
en donde $n=n_e$ es la \textit{densidad numérica de electrones}, $m_u=1.66\cdot10^{-27}\,[kg]$ es la unidad de masa atómica (e.d., de protones y de neutrones, ver apéndice \ref{app:constantes}), y $\mu_e$ es el \textit{peso molecular medio por electrón}. La expresión (\ref{densidad_electrones}) considera que la masa y por tanto la densidad del fluido considerado se debe a los bariones de los núcleos atómicos, despreciándose la contribución de los electrones:
\begin{align}
 \mu_e&=\frac{\text{masa total}}{m_u}\frac{1}{N^{\circ}\text{ total de electrones}},\\
%&=\frac{m_B}{m_u Y_e}=\left(\frac{\sum_i n_i m_i}{\sum_i n_i A_i}\right)\frac{1}{m_u}\frac{1}{Y_e},\\
&\approx \frac{1}{Y_e}=\frac{N^{\circ}\text{ total de bariones}}{N^{\circ}\text{ total de electrones}}=\frac{Z}{A},
\end{align}
en donde $m_B\approx m_u$ es la \textit{masa media de bariones}, $Y_e$ es el \textit{número medio de electrones por barión}, $A$ es el peso atómico y $Z$ el número atómico de la especie considerada (e igual al número de protones en el núcleo).  Para muchos de los elementos de los que usualmente está compuesta una enana blanca,  en donde esta relación es válida, tal como ${}^4\textrm{He}$, ${}^{12}\textrm{C}$ y ${}^{24}\textrm{Mg}$, el peso molecular medio por electrón es $\mu_e=2$. Una excepción es ${}^{56}\textrm{Fe}$, que tiene $\mu_e\approx2.15$.

\item \emph{Neutrones (en estrellas de neutrones)}:
\begin{equation}\label{densidad_neutrones}
\rho_0=  m_n n_n,
\end{equation}
en donde $n=n_n$ es la densidad numérica de neutrones y $m_n$ es la masa de un neutrón. Esto se debe a que en las estrellas de neutrones existen prácticamente sólo estos bariones, por lo que la masa total será debido únicamente a ellos.
\end{enumerate}

\subsection{Función de distribución de Fermi}
En general, de la estadística de Fermi-Dirac, sabemos que la función de distribución de un gas ideal de Fermi (describiendo fermiones, partículas de spin semi-entero) estará dada, como función de la energía, por:
\begin{equation}
 f(E)=\frac{1}{e^{\frac{E-\mu}{kT}}+1},
\end{equation}
en donde $\mu$ es el \textit{potencial químico}. Es posible probar que la relación anterior se reduce a la conocida función de distribución de Maxwell-Boltzmann para densidades bajas y temperaturas altas. Por otra parte, cuando las temperaturas son bajas, como ocurre con la materia presente en las estrellas analizadas, los fermiones se irán a los niveles de energía más bajos disponibles, denominándose \textit{gas de Fermi completamente degenerado} en el límite $T\to 0$. Para este caso, el potencial químico $\mu=E_f$ pasa a denominarse energía de Fermi, y la función de distribución se convierte en una función escalón:
\begin{equation}\label{fermidegenerada}
f(E)=\begin{cases}
   1& \text{si } E \leq E_F ,\\
   0& \text{si } E > E_F.
  \end{cases}
\end{equation}
En esta situación, todos los fermiones al estar ocupando el nivel más bajo, tendrán $\left|\vec{p}\right|\le p_F$, en donde $p_F$ se denomina momentum de Fermi, estando relacionados con la energía de Fermi mediante la relación
\begin{equation}\label{energiafermi}
E_F=\sqrt{p_F^2 c^2+m_e^2 c^4}.
\end{equation}

\subsection{Ecuación de estado de Fermi exacta}

\subsubsection{Densidad numérica y densidad propia de masa}
Usando la función de distribución para un gas ideal de Fermi completamente degenerado \eqref{fermidegenerada} de electrones (con subíndice $e$) o neutrones (con subíndice $n$)  en la definición \eqref{densidad_dist}, obtenemos para la densidad numérica de las partículas.
\begin{equation}
 n_{e,n}=\frac{2}{h^3}\int_0^{p_F}4\pi p^2 dp=\frac{8\pi}{3h^3}\,p_F^3,
\end{equation}
en donde se ha considerado $g=2$ debido a que el spin de los fermiones considerados, electrones y neutrones, es $S=1/2$. Definiendo el \textit{parámetro adimensional de momentum relativo} $x$:
\begin{equation}\label{xrelativo}
 x:=\frac{p}{m_{e,n}c}\qquad\Rightarrow\qquad x_F=\frac{p_F}{m_{e,n} c},
\end{equation}
podemos escribir la relación anterior como:
\begin{equation}\label{densidadfermi1}
 n_{e,n}=\frac{8\pi c^3}{3 h^3}m_{e,n}^3x_F^3.
\end{equation}
De este modo, podemos expresar la densidad propia de masa para el caso de electrones usando \eqref{densidad_electrones} como:
\begin{equation}\label{densidad_electrones-fermi}
\boxed{ \rho_0= \frac{8\pi c^3}{3 h^3}\mu_em_u m_e^3x_F^3\approx9.7393\cdot10^8\,\mu_e\, x_F^3\;[kg/m^3],}
\end{equation}
mientras que para neutrones usamos \eqref{densidad_neutrones}, obteniendo:
\begin{equation}\label{densidad_neutrones-fermi}
\boxed{ \rho_0= \frac{8\pi c^3}{3 h^3} m_n^4x_F^3\approx6.1066\cdot10^{18}\,x_F^3\;[kg/m^3].}
\end{equation}

\subsubsection{Calculando la presión y densidad de energía}
Por otra parte, para la presión usamos la definición \eqref{presion_dist2}, obteniendo la integral
\begin{equation}
 P_{e,n}=\frac{1}{3}\frac{2}{h^3}\int_0^{p_F}\frac{p^2c^2}{(p^2 c^2+m_{e,n}^2 c^4)^{1/2}}4\pi p^2\,dp=\frac{8\pi c^2}{3h^3}\int\limits_0^{p_F}\frac{p^4\,dp}{(p^2 c^2+m_e^2c^4)^{1/2}},
\end{equation}
y en función del parámetro relativo $x$ \eqref{xrelativo}, podemos escribir la integral de presión en la forma:
\begin{equation}\label{presionfermi1}
P_{e,n}=\frac{8\pi m_{e,n}^4c^5}{3h^3}\int\limits_0^{x_F}\frac{x^4\,dx}{(1+x^2)^{1/2}}.
\end{equation}
Además, la segunda forma en que se ha escrito la integral de presión \eqref{presion_dist2}  permite encontrar una relación directa con la integral de energía \eqref{energia_dist}, puesto que integrando por partes con la función de distribución considerada \eqref{fermidegenerada}, podemos escribir:
\begin{align}
 P=\frac{2}{3h^3}\int\limits_0^{p_F} p\frac{dE}{dp}\,d^3p&=\frac{8\pi}{3h^3}\int\limits_0^{p_F} p^3\frac{dE}{dp}\,dp\\
&=\frac{8\pi}{3h^3}\left\{\int\limits_0^{p_F} \frac{d}{dp}\left(p^3 E\right)\,dp-\int\limits_0^{p_F} 3p^2E\,dp\right\}\\
&=\left.\frac{8\pi}{3h^3}p^3E\right|_0^{pf}-\frac{8\pi}{h^3}\int\limits_0^{p_F}Ep^2\,dp\\
&=\frac{8\pi}{3h^3}p_F^3E_F-u\label{integral-presion-energia-fermi}.
\end{align}
En términos del parámetro relativo $x_F$, tenemos para la densidad de energía, que:
\begin{align}
u_{e,n}&=\frac{8\pi}{3h^3}p_F^3E_F-P_{e,n}\\
&=\frac{8\pi}{3h^3}\left(\frac{p_F}{m_{e,n}c}\right)^3\left(m_{e,n}c\right)^3\sqrt{p_F^2c^2+m_{e,n}^2c^4}-P_{e,n}\\
&=\frac{8\pi m_{e,n}^4c^5}{3h^3}x_F^3\sqrt{1+x_F^2}-P_{e,n}.\label{fermi-relacion-presion-energia}
\end{align}
Por lo tanto, sólo basta determinar la integral de presión \eqref{presionfermi1} para obtener de la expresión anterior la densidad de energía $\epsilon$, no requiriendo calcular directamente \eqref{energia_dist}.
% Aunque por completitud, igual se dará la forma explícita de la integral de energía:
% \begin{align}
%  \epsilon_{e,n}&=\frac{2}{h^3}\int\limits_0^{p_F}\left(p^2 c^2+m_{e,n}^2 c^4\right)4\pi p^2\,dp\\
% &=\frac{8\pi m_{e,n}^4c^5}{h^3}\int\limits_0^{x_F}x^2\sqrt{x^2+1}\,dx.\label{energiafermi1}
% \end{align}
Entonces, para calcular \eqref{presionfermi1}, conviene efectuar la sustitución hiperbólica,
\begin{align}\label{sust-hiperbolica}
 x&=:\sinh\theta\quad\Rightarrow\quad dx=\cosh\theta d\theta,\qquad \left(1+x^2\right)^{1/2}=\cosh\theta,\\
\theta_F&=\sinh^{-1}x_F=\ln\left|x_F+\sqrt{1+x_F^2}\right|,
\end{align}
con la cual la integral de presión queda
\begin{align}\label{presionfermi-hiperbolica1}
 P_{e,n}=\frac{8\pi m_{e,n}^4c^5}{3h^3}\int\limits_0^{\theta_F}\sinh^4\theta\, d\theta.
\end{align}
Para determinarla, usamos algunas identidades hiperbólicas\footnote{$\cosh^2\theta-\sinh^2\theta\equiv 1$, $\sinh2\theta\equiv 2\sinh\theta\cosh\theta$, $\cosh2\theta\equiv \sinh^2\theta+\cosh^2\theta\equiv 1+2\sinh^2\theta$, $\sinh^2\theta\equiv \left(\cosh2\theta-1\right)/2$.} de modo que el integrando se pueda reescribir como:
\begin{align}
 \sinh^4\theta &= \sinh^2\theta\left(\cosh^2\theta-1\right) \\
&=\left(\frac{\sinh 2\theta}{2}\right)^2-\sinh^2\theta,\\
&=\frac{1}{4}\left(\frac{\cosh4\theta-1}{2}\right)-\left(\frac{\cosh2\theta-1}{2}\right),\\
&=\frac{1}{8}\left(\cosh4\theta-4\cosh2\theta+3\right)\label{fermi-senh4},
\end{align}
con lo cual se puede calcular directamente la integral \eqref{presionfermi-hiperbolica1}, ya que al reemplazar allí \eqref{fermi-senh4}, obtenemos:
\begin{align}
 P_{e,n}&=\frac{\pi m_{e,n}^4c^5}{3h^3}\int\limits_0^{\theta_F}\left(\cosh4\theta-4\cosh2\theta+3\right)\, d\theta\\
&=\frac{\pi m_{e,n}^4c^5}{3h^3}\left(\frac{1}{4}\sinh4\theta_F-2\sinh2\theta_F+3\theta_F\right) .\label{presionfermi-hiperbolica2}
\end{align}
Esta expresión se puede escribir de varias formas equivalentes, las que serán útiles según las circunstancias:
\begin{itemize}
 \item Factorizando directamente \eqref{presionfermi-hiperbolica2} por 1/4,
\begin{align}
 P_{e,n}=\frac{1}{3}\frac{\pi m_{e,n}^4c^5}{4h^3}\left[\sinh\left(4\theta_F\right)-8\sinh\left(\frac{4\theta_F}{2}\right)+3\left(4\theta_F\right)\right],
\end{align}
e introduciendo el parámetro $t$ definido por
\begin{align}\label{fermi-parametro-t}
 t:=4\theta_F&=4\sinh^{-1}x_F=4\ln\left|x_F+\sqrt{1+x_F^2}\right|,
\end{align}
obtenemos la forma paramétrica para la presión dada por Oppenheimer \cite{Oppenheimer39enero}:
\begin{align}\label{fermi-presion-OV}
 \boxed{P_{e,n}=\frac{1}{3}\frac{\pi m_{e,n}^4c^5}{4h^3}\left(\sinh t-8\sinh\left(\frac{t}{2}\right)+3t\right).}
\end{align}
Para encontrar la densidad de energía, usamos la relación \eqref{fermi-relacion-presion-energia}, notando que de la sustitución \eqref{sust-hiperbolica} podemos escribir
\begin{align}
 x_F^3\sqrt{1+x_F^2}&=\sinh^3\theta_F\cosh\theta_F=\left(\sinh^2\theta_F\right)\left(\sinh\theta_F \cosh\theta_F\right)\\
&=\left(\frac{\cosh 2\theta_F-1}{2}\right)\left(\frac{\sinh 2\theta_F}{2}\right)
% \frac{1}{4}\left(\cosh 2\theta_F\sinh 2\theta_F\right)-\frac{1}{4\sinh 2\theta_F}\\
=\frac{1}{8}\sinh4\theta_F-\frac{1}{4}\sinh2\theta_F.
\end{align}
Luego, obtenemos que:
\begin{align}
u_{e,n}&=\frac{8\pi m_{e,n}^4c^5}{3h^3}\left[\frac{1}{8}\sinh4\theta_F-\frac{1}{4}\cancel{\sinh2\theta_F}\right]-\frac{1}{3}\frac{\pi m_{e,n}^4c^5}{4h^3}\left[\sinh\left(4\theta_F\right)-8\cancel{\sinh\left(\frac{4\theta_F}{2}\right)}+3\left(4\theta_F\right)\right],\\
&=\frac{\pi m_{e,n}^4c^5}{3h^3}\left[\frac{3}{4}\sinh 4\theta_F-3\theta_F\right],
\end{align}
y en términos del parámetro $t$ \eqref{fermi-parametro-t}:
\begin{align}\label{fermi-energia-OV}
 \boxed{u_{e,n}=\frac{\pi m_{e,n}^4c^5}{4h^3}(\sinh t-t).}
\end{align}

\item También podemos expresar \eqref{presionfermi-hiperbolica2} directamente en términos del parámetro $x_F$, para lo cual se reescribe dicha expresión en la forma:
\begin{align}
P_{e,n}&=\frac{\pi m_{e,n}^4c^5}{3h^3}\left(\frac{1}{2}\sinh2\theta_F\cosh2\theta-2\sinh2\theta_F+3\theta_F\right),\\
&=\frac{\pi m_{e,n}^4c^5}{3h^3}\left(\frac{1}{2}\sinh2\theta_F\left(1+2\sinh^2\theta_F\right)-2\sinh2\theta_F+3\theta_F\right),\\
&=\frac{\pi m_{e,n}^4c^5}{3h^3}\left(\sinh\theta_F\cosh\theta_F\left(2\sinh^2\theta_F-3\right)+3\theta_F\right),
\end{align}
y reexpresando en términos de $x_F$ mediante \eqref{sust-hiperbolica}, obtenemos para la presión
\begin{equation}\label{presionfermi2}
\boxed{ P_{e,n}=\frac{\pi m_{e,n}^4c^5}{3h^3}\left[x_F\sqrt{1+x_F^2}\left(2x_F^2-3\right)+3\ln\left|x_F+\sqrt{1+x_F^2}\right|\right],}
\end{equation}
\end{itemize}
y mediante \eqref{fermi-relacion-presion-energia}, obtenemos directamente para la densidad de energía la siguiente relación:
\begin{equation}\label{energiafermi2}
 \boxed{u_{e,n}=\frac{\pi m_{e,n}^4c^5}{h^3}\left[x_F\sqrt{1+x_F^2}\left(1+2x_F^2\right)-\ln\left|x_F+\sqrt{1+x_F^2}\right|\right],}
\end{equation}
que son las formas para estas variables dadas por Chandrasekhar \cite{Chandra39} y Shapiro \cite{Shapiro83}.


\subsubsection{Obtención de la ecuación de estado explícita}

De esta forma, las ecuaciones \eqref{densidad_electrones} ó \eqref{densidad_neutrones}, \eqref{presionfermi2} y \eqref{energiafermi2} proporcionarán una forma paramétrica para la ecuación de estado de Fermi completamente degenerada (y exacta) en función de $x_F$: $\rho_0=\rho_0(x_F)$, $P=P(x_F)$ y $u=u(x_F)$.

Ahora bien, para determinar una ecuación de estado del tipo \eqref{estado}, se despeja $x_F$ de \eqref{densidad_electrones-fermi}, de modo de obtener una dependencia con la densidad propia de masa del tipo $x_F=x_F(\rho)$, que para el caso de electrones es:
\begin{equation}\label{xrelativo_electrones}
 x_F=\left(\frac{3h^3}{8\pi c^3 m_u m_e^3}\right)^{1/3}\left(\frac{\rho_0}{\mu_e}\right)^{1/3}\approx\left(\frac{\rho_0/\mu_e}{9.7393\cdot10^8\,[kg/m^3]}\right)^{1/3},
\end{equation}
y para neutrones sería:
\begin{equation}\label{xrelativo_neutrones}
 x_F=\left(\frac{3h^3}{8\pi c^3 m_n^4}\right)^{1/3}\left(\rho_0\right)^{1/3}\approx\left(\frac{\rho_0/\mu_e}{6.1066\cdot10^{18}\,[kg/m^3]}\right)^{1/3}.
\end{equation}

Luego, una ecuación de estado que relacione la presión con la densidad de masa se obtendrá sustituyendo lo anterior en \eqref{presionfermi2}, obteniendo $P=P(x_F)=P(x_F(\rho_0))=P(\rho_0)$. Del mismo modo, se puede encontrar la densidad de energía en función de la densidad propia de masa, usando \eqref{energiafermi2}: $u=\epsilon(x_F)=u(x_F(\rho_0))=u(\rho_0)$.

\subsection{Ecuaciones de estado de Fermi aproximadas}
Las expresiones resultantes para las ecuaciones de estado de Fermi exactas son poco manejables analíticamente, por lo que su solución recae en métodos numéricos, tal como se hizo para las ecuaciones de estructura estelar en la sección \ref{sec:fermi-exacta}. Por esta razón, y con el objetivo de simplificar la ecuación de estado obtenida, consideraremos dos casos extremos para el parámetro $x_F$ dado por \eqref{xrelativo}:

\begin{enumerate}
 \item $x_F\ll1$. Es posible encontrar directamente una serie de potencias para la presión \eqref{presionfermi2} y densidad de energía \eqref{energiafermi2} a paritr de dichas expresiones. Sin embargo, es más fácil expandir primero el integrando de \eqref{presionfermi1}, ya que en este límite:
\begin{align}
P_{e,n}&\approx\frac{8\pi m_{e,n}^4c^5}{3h^3}\int\limits_0^{x_F}x^4\left[1-\frac{1}{2}x^2+\left(-\frac{1}{2}\right)\left(-\frac{3}{2}\right)\left(\frac{1}{2!}\right)x^4+\cdots\right]\,dx,\\
&=\frac{8\pi m_{e,n}^4c^5}{3h^3}\left[\frac{x_F^5}{5}-\frac{1}{2}\frac{x_F^7}{7}+\frac{3}{8}\frac{x_F^9}{9}+\cdots\right],
\end{align}
y también, usando \eqref{fermi-relacion-presion-energia} y expandiendo:
\begin{align}
 u_{e,n}&\approx\frac{8\pi m_{e,n}^4c^5}{3h^3}\left[x_F^3\left(1+\frac{1}{2}x_F^2+\left(\frac{1}{2}\right)\left(-\frac{1}{2}\right)\left(\frac{1}{2!}\right)x_F^4+\cdots\right)\right]-P_{e,n}\\
&=\frac{8\pi m_{e,n}^4c^5}{3h^3}\left\{\left[x_F^3+\frac{1}{2}x_F^5-\frac{1}{8}x_F^7+\cdots\right]-\left[\frac{x_F^5}{5}-\frac{1}{2}\frac{x_F^7}{7}+\frac{3}{8}\frac{x_F^9}{9}+\cdots\right]\right\}\\
&=\frac{8\pi m_{e,n}^4c^5}{3h^3}\left[x_F^3+\frac{3}{10}x_F^5-\frac{3}{56}x_F^7+\cdots \right]
\end{align}
De este modo\footnote{Note que comparando ambas expansiones con \eqref{presionfermi2} y \eqref{energiafermi2}, es posible encontrar las siguientes relaciones va'lidas para $x_F\ll1$:
\begin{align}
 x_F\sqrt{1+x_F^2}\left(2x_F^2-3\right)+3\ln\left|x_F+\sqrt{1+x_F^2}\right|&=\frac{8}{5}\left(x_F^5-\frac{5}{14}x_F^7+\frac{5}{24}x_F^9+\cdots\right),\\
 x_F\sqrt{1+x_F^2}\left(1+2x_F^2\right)-\ln\left|x_F+\sqrt{1+x_F^2}\right|&=\frac{8}{3}\left(x_F^3+\frac{3}{10}x_F^5-\frac{3}{56}x_F^7+\cdots \right).
\end{align}
}, dejando sólo el primer término en la expresión para la presión:
\begin{equation}\label{presionfermi1-asintotico}
P_{e,n}=\frac{8\pi m_{e,n}^4c^5}{3h^3}\frac{x_F^5}{5}
\end{equation}
obtenemos una ecuación de estado politrópica \eqref{estadopolitropica} con índice $\gamma=5/3$, ya que $x_F$ es proporcional a $\rho_0^{1/3}$. Su expresión explícita dependerá del tipo de partícula involucrada:
\begin{enumerate}
\item \emph{Electrones no relativistas} ($p_F\ll m_{e}c$). Equivale por  \eqref{xrelativo_electrones} a que las densidades típicas de las enanas blancas sean bajas, del orden de $\rho_0\ll10^{9}\,[kg/m^3]$. Reemplazando dicho $x_F$ en la expansión asintótica de la presión \eqref{presionfermi1-asintotico}, tenemos:
\begin{equation}\label{fermi_norelativista}
 \boxed{P_e=\frac{3^{2/3}\pi^{4/3}}{5}\frac{\hbar^2}{m_e(m_u\mu_e)^{5/3}}\rho_0^{5/3}\approx1.00359\cdot 10^{7}\,\left(\frac{\rho_0}{\mu_e}\right)^{5/3}\,MKS.}
\end{equation}

\item \emph{Neutrones no relativistas} ($p_F\ll m_{n}c$). Equivale por  \eqref{xrelativo_neutrones} a que las densidades típicas de las estrellas de neutrones sean bajas, del orden de $\rho\ll6\cdot10^{18}\,[kg/m^3]$.  Reemplazando dicho $x_F$ en la expansión asintótica de la presión \eqref{presionfermi1-asintotico}, tenemos:
\begin{equation}\label{fermi_norelativista2}
 \boxed{P_n=\frac{3^{2/3}\pi^{4/3}}{5}\frac{\hbar^2}{m_n^{8/3}}\rho_0^{5/3}\approx5.3803\cdot 10^{3}\,\left(\rho_0\right)^{5/3}\,MKS.}
\end{equation}

\end{enumerate}
 \item $x_F\gg1$. Al igual que para el caso anterior, en vez de expandir directamente la serie de potencias para la presión \eqref{presionfermi2} y densidad de energía \eqref{energiafermi2}, es más conveniente desarrollar primero el integrando de \eqref{presionfermi1}, que en este límite es:
\begin{align}
P_{e,n}&\approx\frac{8\pi m_{e,n}^4c^5}{3h^3}\int\limits_0^{x_F}\frac{x^4\,dx}{x\left(1+\frac{1}{x^2}\right)^{1/2}}\\
&=\frac{8\pi m_{e,n}^4c^5}{3h^3}\int\limits_0^{x_F}x^3
\left[1-\frac{1}{2}\frac{1}{x^2}+\left(-\frac{1}{2}\right)\left(-\frac{3}{2}\right)\left(\frac{1}{2!}\right)\frac{1}{x^4}+\cdots\right]\,dx,\\
&=\frac{8\pi m_{e,n}^4c^5}{3h^3}\left[\frac{x_F^4}{4}-\frac{1}{2}\frac{x_F^2}{2}+\frac{3}{8}\ln(x_F)+\cdots\right],
\end{align}
y también, usando \eqref{fermi-relacion-presion-energia} y expandiendo:
\begin{align}
 u_{e,n}&\approx\frac{8\pi m_{e,n}^4c^5}{3h^3}\left[x_F^4\left(1+\frac{1}{x_F^2}\right)^{1/2}\right]-P_{e,n}\\
&=\frac{8\pi m_{e,n}^4c^5}{3h^3}\left[x_F^4\left(1+\frac{1}{2}\frac{1}{x_F^2}+\left(\frac{1}{2}\right)\left(-\frac{1}{2}\right)\left(\frac{1}{2!}\right)\frac{1}{x_F^4}+\cdots\right)\right.\\
&\left.\quad-\left(\frac{x_F^4}{4}-\frac{1}{2}\frac{x_F^2}{2}+\frac{3}{8}\ln(x_F)+\cdots\right)\right]\\
&=\frac{8\pi m_{e,n}^4c^5}{3h^3}\left[\frac{3}{4}x_F^4+\frac{3}{4}x_F^2-\frac{3}{8}\ln(x_F)+\cdots\right]\label{energia_expansion_fermi_relativista}
\end{align}
De este modo\footnote{Note que comparando ambas expansiones con \eqref{presionfermi2} y \eqref{energiafermi2}, es posible encontrar las siguientes relaciones válidas para $x_F\gg1$:
\begin{align}
 x_F\sqrt{1+x_F^2}\left(2x_F^2-3\right)+3\ln\left|x_F+\sqrt{1+x_F^2}\right|&=2\left(x_F^4-x_F^2+\frac{3}{2}\ln(x_F)+\cdots\right),\\
 x_F\sqrt{1+x_F^2}\left(1+2x_F^2\right)-\ln\left|x_F+\sqrt{1+x_F^2}\right|&=2\left(x_F^4+x_F^2-\frac{1}{2}\ln(x_F)+\cdots \right).
\end{align}
}, dejando sólo el primer término en la expresión para la presión:
\begin{equation}\label{presionfermi2-asintotico}
P_{e,n}=\frac{8\pi m_{e,n}^4c^5}{3h^3}\frac{x_F^4}{4}
\end{equation}
obtenemos una ecuación de estado politrópica \eqref{estadopolitropica} con índice $\gamma=4/3$, ya que $x_F$ es proporcional a $\rho_0.^{1/3}$. Su expresión explícita dependerá del tipo de partícula involucrada:

\begin{enumerate}
\item \emph{Electrones ultra-relativistas} ($p_F\gg m_{e}c$). Equivale por  \eqref{xrelativo_electrones} a que las densidades típicas de las enanas blancas sean altas, del orden de $\rho_0\gg10^{9}\,[kg/m^3]$. Reemplazando dicho $x_F$ en la expansión asintótica de la presión \eqref{presionfermi2-asintotico}, tenemos:
\begin{equation}\label{fermi_relativista}
 \boxed{P_e=\frac{3^{1/3}\pi^{2/3}}{4}\frac{\hbar c}{(m_u\mu_e)^{4/3}}\rho_0^{4/3}\approx1.2435\cdot 10^{10}\,\left(\frac{\rho_0}{\mu_e}\right)^{4/3}\,MKS.}
\end{equation}

\item \emph{Neutrones ultra-relativistas} ($p_F\gg m_{n}c$). Equivale por  \eqref{xrelativo_neutrones} a que las densidades típicas de las estrellas de neutrones sean altas, del orden de $\rho_0\gg6\cdot10^{18}\,[kg/m^3]$.Reemplazando dicho $x_F$ en la expansión asintótica de la presión \eqref{presionfermi2-asintotico}, tenemos:
\begin{equation}\label{fermi_relativista2}
 \boxed{P_n=\frac{3^{1/3}\pi^{2/3}}{4}\frac{\hbar c}{m_n^{4/3}}\rho_0^{4/3}\approx1.2293\cdot 10^{10}\,\left(\rho_0\right)^{4/3}\,MKS.}
\end{equation}

\end{enumerate}
\end{enumerate}

La diferencia en el exponente (índice politrópico $\gamma$) de las ecuaciones de estado obtenidas tiene importancia fundamental en la estabilidad de enanas blancas y estrellas de neutrones, tanto aplicando la teoría newtoniana de gravitación como Relatividad General.

Notar que en las aplicaciones del texto principal en que se usen resultados de este apéndice, se omitirá el subíndice $F$ en el parámetro relativo $x_F$, a fin de no recargar la notación.
